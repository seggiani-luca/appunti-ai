
\documentclass[a4paper,11pt]{article}
\usepackage[a4paper, margin=8em]{geometry}

% usa i pacchetti per la scrittura in italiano
\usepackage[french,italian]{babel}
\usepackage[T1]{fontenc}
\usepackage[utf8]{inputenc}
\frenchspacing 

% usa i pacchetti per la formattazione matematica
\usepackage{amsmath, amssymb, amsthm, amsfonts}

% usa altri pacchetti
\usepackage{gensymb}
\usepackage{hyperref}
\usepackage{standalone}

% imposta il titolo
\title{Appunti Ricerca Operativa}
\author{Luca Seggiani}
\date{2024}

% disegni
\usepackage{pgfplots}
\pgfplotsset{width=10cm,compat=1.9}

% imposta lo stile
% usa helvetica
\usepackage[scaled]{helvet}
% usa palatino
\usepackage{palatino}
% usa un font monospazio guardabile
\usepackage{lmodern}

\renewcommand{\rmdefault}{ppl}
\renewcommand{\sfdefault}{phv}
\renewcommand{\ttdefault}{lmtt}

% disponi il titolo
\makeatletter
\renewcommand{\maketitle} {
	\begin{center} 
		\begin{minipage}[t]{.8\textwidth}
			\textsf{\huge\bfseries \@title} 
		\end{minipage}%
		\begin{minipage}[t]{.2\textwidth}
			\raggedleft \vspace{-1.65em}
			\textsf{\small \@author} \vfill
			\textsf{\small \@date}
		\end{minipage}
		\par
	\end{center}

	\thispagestyle{empty}
	\pagestyle{fancy}
}
\makeatother

% disponi teoremi
\usepackage{tcolorbox}
\newtcolorbox[auto counter, number within=section]{theorem}[2][]{%
	colback=blue!10, 
	colframe=blue!40!black, 
	sharp corners=northwest,
	fonttitle=\sffamily\bfseries, 
	title=Teorema~\thetcbcounter: #2, 
	#1
}

% disponi definizioni
\newtcolorbox[auto counter, number within=section]{definition}[2][]{%
	colback=red!10,
	colframe=red!40!black,
	sharp corners=northwest,
	fonttitle=\sffamily\bfseries,
	title=Definizione~\thetcbcounter: #2,
	#1
}

% disponi problemi
\newtcolorbox[auto counter, number within=section]{problem}[2][]{%
	colback=green!10,
	colframe=green!40!black,
	sharp corners=northwest,
	fonttitle=\sffamily\bfseries,
	title=Problema~\thetcbcounter: #2,
	#1
}

% disponi codice
\usepackage{listings}
\usepackage[table]{xcolor}

\lstdefinestyle{codestyle}{
		backgroundcolor=\color{black!5}, 
		commentstyle=\color{codegreen},
		keywordstyle=\bfseries\color{magenta},
		numberstyle=\sffamily\tiny\color{black!60},
		stringstyle=\color{green!50!black},
		basicstyle=\ttfamily\footnotesize,
		breakatwhitespace=false,         
		breaklines=true,                 
		captionpos=b,                    
		keepspaces=true,                 
		numbers=left,                    
		numbersep=5pt,                  
		showspaces=false,                
		showstringspaces=false,
		showtabs=false,                  
		tabsize=2
}

\lstdefinestyle{shellstyle}{
		backgroundcolor=\color{black!5}, 
		basicstyle=\ttfamily\footnotesize\color{black}, 
		commentstyle=\color{black}, 
		keywordstyle=\color{black},
		numberstyle=\color{black!5},
		stringstyle=\color{black}, 
		showspaces=false,
		showstringspaces=false, 
		showtabs=false, 
		tabsize=2, 
		numbers=none, 
		breaklines=true
}

\lstdefinelanguage{javascript}{
	keywords={typeof, new, true, false, catch, function, return, null, catch, switch, var, if, in, while, do, else, case, break},
	keywordstyle=\color{blue}\bfseries,
	ndkeywords={class, export, boolean, throw, implements, import, this},
	ndkeywordstyle=\color{darkgray}\bfseries,
	identifierstyle=\color{black},
	sensitive=false,
	comment=[l]{//},
	morecomment=[s]{/*}{*/},
	commentstyle=\color{purple}\ttfamily,
	stringstyle=\color{red}\ttfamily,
	morestring=[b]',
	morestring=[b]"
}

% disponi sezioni
\usepackage{titlesec}

\titleformat{\section}
	{\sffamily\Large\bfseries} 
	{\thesection}{1em}{} 
\titleformat{\subsection}
	{\sffamily\large\bfseries}   
	{\thesubsection}{1em}{} 
\titleformat{\subsubsection}
	{\sffamily\normalsize\bfseries} 
	{\thesubsubsection}{1em}{}

% disponi alberi
\usepackage{forest}

\forestset{
	rectstyle/.style={
		for tree={rectangle,draw,font=\large\sffamily}
	},
	roundstyle/.style={
		for tree={circle,draw,font=\large}
	}
}

% disponi algoritmi
\usepackage{algorithm}
\usepackage{algorithmic}
\makeatletter
\renewcommand{\ALG@name}{Algoritmo}
\makeatother

% disponi numeri di pagina
\usepackage{fancyhdr}
\fancyhf{} 
\fancyfoot[L]{\sffamily{\thepage}}

\makeatletter
\fancyhead[L]{\raisebox{1ex}[0pt][0pt]{\sffamily{\@title \ \@date}}} 
\fancyhead[R]{\raisebox{1ex}[0pt][0pt]{\sffamily{\@author}}}
\makeatother

\begin{document}

% sezione (data)
\section{Lezione del 02-10-24}

% stili pagina
\thispagestyle{empty}
\pagestyle{fancy}

% testo
\subsection{Trasporto}
Poniamo il seguente problema:

\begin{problem}{Trasporto}
	Due centrali del latte di Firenze producono	rispettivamente 50 e 60 mila litri di latte al giorno.
	Le centrali servono tre quartieri, che consumano rispettivamente 30, 30 e 20 mila litri di latte al giorno.
	Si conosce il costo necessario per portare un migliaio di litri di latte da ogni centrale a ogni quartiere, riportato nella seguente tabella:
	
	\center{} \rowcolors{2}{green!10}{green!40!black!20}
	\begin{tabular} { | c || c | c | c | }
		\hline
		& \bfseries Novoli & \bfseries Statuto & \bfseries Rifredi \\ 
		\hline
		\bfseries Centrale A & 6 & 8 & 4 \\
		\bfseries Centrale B & 7 & 3 & 9 \\
		\hline
	\end{tabular}

	\par\bigskip

	Vogliamo capire quanto latte deve spedire ogni centrale ad ogni quartiere.

	\raggedright
	\par\smallskip

	\tiny{Nota simpatica: secondo l'indagine INRAN-SCAI 2005-06, l'italiano medio consuma $0.115 \mathrm{g}$ di latte al giorno, che per un peso specifico di circa $1.040 \mathrm{kg}/\mathrm{L}$ fanno $0.11 \mathrm{L}$ di latte al giorno. Al 2024, il comune di Firenze ha $364\ 073$ abitanti, ergo dovrebbe avere bisogno di approssimativamente $40\ 258 \mathrm{L}$ di latte al giorno. I fiorentini nell'esempio devono avere le ossa veramente forti!}

\end{problem}

Possiamo esprimere il problema dell'esempio come un problema LP.
Abbiamo innanzitutto che i costi di trasporto formano una matrice:
$$
C_{matr} =
\begin{pmatrix}
6 & 8 & 4 \\
7 & 3 & 9 \\
\end{pmatrix}
$$
che possiamo linearizzare, come avevamo fatto nei problemi di assegnamento di costo minimo, in un vettore costo:
$$
C = (6, 8, 4, 7, 3, 9)
$$

Questo vettore moltiplica il vettore delle variabili decisionali, che è la linearizzazione della matrice:
$$
x_{matr} =
\begin{pmatrix}
	x_{13} & x_{14} & x_{15} \\ 
	x_{23} & x_{24} & x_{25}
\end{pmatrix}
$$

Questa matrice non rappresenta altro che quanto latte mandare ad ogni quartiere.

A questo punto, possiamo stabilire i vincoli.
Innanzitutto, non si può avere più latte di quanto viene prodotto, ergo:
\[
	\begin{cases}
		x_{13} + x_{14} + x_{15} \leq 50 \\ 
		x_{23} + x_{14} + x_{15} \leq 60
	\end{cases}
\]
inoltre, si vuole fornire ad ogni quartiere il fabbisogno richiesto, ergo:
\[
	\begin{cases}
		x_13 + x_{23} \geq 30 \\	
		x_14 + x_{24} \geq 30	\\
		x_15 + x_{25} \geq 20	
	\end{cases}
\]

Questo è un problema di programmazione lineare.

In generale, quindi, un problema di trasporto minimizza la funzione obiettivo data da una matrice di costo in $n \times m$ variabili, con $m$ vincoli di riga sul vettore $o_j$ dei limiti di produzione, e $n$ vincoli di colonna sul vettore $d_j$ della domanda, in forma:
\[
	\begin{cases}
		\min{\sum^m_{i=1} \sum^n_{j=1} c_{ij}x_{ij}} = C^T \cdot x \\
		\sum^m_{i=1} x_{ij} \geq d_{j} \quad \forall j = 1,...,n \\ 
		\sum^n_{j=1} x_{ij} \leq o_{i} \quad \forall i = 1,...,m \\ 
		x \geq 0
	\end{cases}
\]

Non ci sono soluzioni se la domanda supera l'offerta, cioè sé:
$$
\sum_{j=1}^m d_j \geq \sum_{i=1}^m o_i
$$

Mentre in caso di eccessi di produzione, potremmo trasformare le diseguaglianze in eguaglianze, e aggiugnere un carico "fittizio" con costo zero dove deviare il surplus di produzione.

Inoltre, come avevamo detto per i problemi di assegnamento di costo minimo, anche qui potremmo scegliere di distinguere fra trasporti divisibili (nel campo $x = \mathbb{R}^n$) e indivisibili (col vincolo aggiunto $x = \mathbb{Z}^n$).

\subsection{Forma duale standard}
Avevamo definito la forma primale standard:
\[
	\begin{cases}
		\max{C^T \cdot x} \\
		Ax \leq b
	\end{cases}
\]

Introduciamo adesso la forma duale standard:
\begin{definition}{Forma duale standard}
	Un problema di programmazione lineare si dice in forma duale standard quando è espresso in forma:
	
	\[
		\begin{cases}
			\min(c^T \cdot x) \\
			Ax = b \\
			x \geq 0
		\end{cases}
	\]

\end{definition}

\subsubsection{Vertici del duale}
Sulle forme duali è semplice il calcolo dei vertici. 
Possiamo infatti avere, come avevamo fatto sulla primale:
\begin{definition}{Soluzione di base duale}
	Sia dato un problema LP $\mathcal{P}$ in forma duale standard.
	Sia $B \subseteq \{ 1, ..., n \}$ un sottoinsieme di indici di variabili decisionali tale che $\mathrm{card}(B) = m$.
	Chiamiamo $x_B$ l'insieme delle variabili decisionali individuate da $B$, e $x_N$ l'insieme delle $n - m$ variabili decisionali rimanenti:
	$$ x = \{x_B, x_N\}$$
	Impostiamo quindi tutte le $x_N$ a 0: avremo un sistema di $m$ variabili in $m$ equazioni, quindi determinato.
	La soluzione di quel sistema è detta soluzione di base duale di $\mathcal{P}$.
\end{definition}

Indichiamo spesso questo vertice come $(bA_B^{-1}, 0)$.
Questa definizione porta ad una caratterizzazione dei vertici del tutto analoga a quella dichiarata sui problemi in forma primale standard:

\begin{theorem}{Caratterizzazione dei vertici duali}
	Su un problema in forma duale standard, un punto $x$ del poliedro $P$ è un vertice di $P$ se e solo se è una soluzione di base duale ammissibile, ovvero:
	$$ 
	x \in \mathrm{vert}(P) \Leftrightarrow \text{$x$ è soluzione di base duale}
	$$
\end{theorem}

\subsubsection{Soluzioni di base duali degeneri}
Possiamo ricavare il concetto di soluzione degenere (e anche di soluzione ammissibile) sui vertici del poliedro del duale. Si ha:

\begin{theorem}{Caratterizzazione delle soluzioni di base duali degeneri}
	Se una soluzione è di base, ergo scelto $B = \{ 1, ..., n \}$ con $\mathrm{card}(B) = m$ è data da $(bA_B^{-1}, 0)$, possiamo dire che è pure degenere quando $\exists i \in B$ tale che almeno una componente si annulla. 
\end{theorem}
e riguardo l'ammissibilità:
\begin{theorem}{Caratterizzazione delle soluzioni di base duali ammissibili}
	Se una soluzione è di base, ergo scelto $B = \{ 1, ..., n \}$ con $\mathrm{card}(B) = m$ è data da $(bA_B^{-1}, 0)$, possiamo dire che è ammissibile quando il vettore soluzione è $\geq 0$. 
\end{theorem}

\end{document}
