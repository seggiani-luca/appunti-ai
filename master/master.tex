
\documentclass[a4paper,11pt]{article}
\usepackage[a4paper, margin=8em]{geometry}

% usa i pacchetti per la scrittura in italiano
\usepackage[french,italian]{babel}
\usepackage[T1]{fontenc}
\usepackage[utf8]{inputenc}
\frenchspacing 

% usa i pacchetti per la formattazione matematica
\usepackage{amsmath, amssymb, amsthm, amsfonts}

% usa altri pacchetti
\usepackage{gensymb}
\usepackage{hyperref}
\usepackage{standalone}

% cose fluttuanti
\usepackage{float}

% imposta il titolo
\title{Appunti Ricerca Operativa}
\author{Luca Seggiani}
\date{2024}

% disegni
\usepackage{pgfplots}
\pgfplotsset{width=10cm,compat=1.9}

% imposta lo stile
% usa helvetica
\usepackage[scaled]{helvet}
% usa palatino
\usepackage{palatino}
% usa un font monospazio guardabile
\usepackage{lmodern}

\renewcommand{\rmdefault}{ppl}
\renewcommand{\sfdefault}{phv}
\renewcommand{\ttdefault}{lmtt}

% disponi il titolo
\makeatletter
\renewcommand{\maketitle} {
	\begin{center} 
		\begin{minipage}[t]{.8\textwidth}
			\textsf{\huge\bfseries \@title} 
		\end{minipage}%
		\begin{minipage}[t]{.2\textwidth}
			\raggedleft \vspace{-1.65em}
			\textsf{\small \@author} \vfill
			\textsf{\small \@date}
		\end{minipage}
		\par
	\end{center}

	\thispagestyle{empty}
	\pagestyle{fancy}
}
\makeatother

% disponi teoremi
\usepackage{tcolorbox}
\newtcolorbox[auto counter, number within=section]{theorem}[2][]{%
	colback=blue!10, 
	colframe=blue!40!black, 
	sharp corners=northwest,
	fonttitle=\sffamily\bfseries, 
	title=Teorema~\thetcbcounter: #2, 
	#1
}

% disponi definizioni
\newtcolorbox[auto counter, number within=section]{definition}[2][]{%
	colback=red!10,
	colframe=red!40!black,
	sharp corners=northwest,
	fonttitle=\sffamily\bfseries,
	title=Definizione~\thetcbcounter: #2,
	#1
}

% disponi problemi
\newtcolorbox[auto counter, number within=section]{problem}[2][]{%
	colback=green!10,
	colframe=green!40!black,
	sharp corners=northwest,
	fonttitle=\sffamily\bfseries,
	title=Problema~\thetcbcounter: #2,
	#1
}

% disponi codice
\usepackage{listings}
\usepackage[table]{xcolor}

\lstdefinestyle{codestyle}{
		backgroundcolor=\color{black!5}, 
		commentstyle=\color{codegreen},
		keywordstyle=\bfseries\color{magenta},
		numberstyle=\sffamily\tiny\color{black!60},
		stringstyle=\color{green!50!black},
		basicstyle=\ttfamily\footnotesize,
		breakatwhitespace=false,         
		breaklines=true,                 
		captionpos=b,                    
		keepspaces=true,                 
		numbers=left,                    
		numbersep=5pt,                  
		showspaces=false,                
		showstringspaces=false,
		showtabs=false,                  
		tabsize=2
}

\lstdefinestyle{shellstyle}{
		backgroundcolor=\color{black!5}, 
		basicstyle=\ttfamily\footnotesize\color{black}, 
		commentstyle=\color{black}, 
		keywordstyle=\color{black},
		numberstyle=\color{black!5},
		stringstyle=\color{black}, 
		showspaces=false,
		showstringspaces=false, 
		showtabs=false, 
		tabsize=2, 
		numbers=none, 
		breaklines=true
}

\lstdefinelanguage{javascript}{
	keywords={typeof, new, true, false, catch, function, return, null, catch, switch, var, if, in, while, do, else, case, break},
	keywordstyle=\color{blue}\bfseries,
	ndkeywords={class, export, boolean, throw, implements, import, this},
	ndkeywordstyle=\color{darkgray}\bfseries,
	identifierstyle=\color{black},
	sensitive=false,
	comment=[l]{//},
	morecomment=[s]{/*}{*/},
	commentstyle=\color{purple}\ttfamily,
	stringstyle=\color{red}\ttfamily,
	morestring=[b]',
	morestring=[b]"
}

% disponi sezioni
\usepackage{titlesec}

\titleformat{\section}
	{\sffamily\Large\bfseries} 
	{\thesection}{1em}{} 
\titleformat{\subsection}
	{\sffamily\large\bfseries}   
	{\thesubsection}{1em}{} 
\titleformat{\subsubsection}
	{\sffamily\normalsize\bfseries} 
	{\thesubsubsection}{1em}{}

% disponi alberi
\usepackage{forest}

\forestset{
	rectstyle/.style={
		for tree={rectangle,draw,font=\large\sffamily}
	},
	roundstyle/.style={
		for tree={circle,draw,font=\large}
	}
}

% disponi algoritmi
\usepackage{algorithm}
\usepackage{algorithmic}
\makeatletter
\renewcommand{\ALG@name}{Algoritmo}
\makeatother

% disponi numeri di pagina
\usepackage{fancyhdr}
\fancyhf{} 
\fancyfoot[L]{\sffamily{\thepage}}

\makeatletter
\fancyhead[L]{\raisebox{1ex}[0pt][0pt]{\sffamily{\@title \ \@date}}} 
\fancyhead[R]{\raisebox{1ex}[0pt][0pt]{\sffamily{\@author}}}
\makeatother

\begin{document}

\pagestyle{fancy}
\thispagestyle{empty}
\renewcommand{\thispagestyle}[1]{}

\maketitle
\documentclass[a4paper,11pt]{article}
\usepackage[a4paper, margin=8em]{geometry}

% usa i pacchetti per la scrittura in italiano
\usepackage[french,italian]{babel}
\usepackage[T1]{fontenc}
\usepackage[utf8]{inputenc}
\frenchspacing 

% usa i pacchetti per la formattazione matematica
\usepackage{amsmath, amssymb, amsthm, amsfonts}

% usa altri pacchetti
\usepackage{gensymb}
\usepackage{hyperref}
\usepackage{standalone}

% imposta il titolo
\title{Appunti Ricerca Operativa}
\author{Luca Seggiani}
\date{23-09-24}

% imposta lo stile
% usa helvetica
\usepackage[scaled]{helvet}
% usa palatino
\usepackage{palatino}
% usa un font monospazio guardabile
\usepackage{lmodern}

\renewcommand{\rmdefault}{ppl}
\renewcommand{\sfdefault}{phv}
\renewcommand{\ttdefault}{lmtt}

% disponi teoremi
\usepackage{tcolorbox}
\newtcolorbox[auto counter, number within=section]{theorem}[2][]{%
	colback=blue!10, 
	colframe=blue!40!black, 
	sharp corners=northwest,
	fonttitle=\sffamily\bfseries, 
	title=Teorema~\thetcbcounter: #2, 
	#1
}

% disponi definizioni
\newtcolorbox[auto counter, number within=section]{definition}[2][]{%
	colback=red!10,
	colframe=red!40!black,
	sharp corners=northwest,
	fonttitle=\sffamily\bfseries,
	title=Definizione~\thetcbcounter: #2,
	#1
}

% disponi problemi
\newtcolorbox[auto counter, number within=section]{problem}[2][]{%
	colback=green!10,
	colframe=green!40!black,
	sharp corners=northwest,
	fonttitle=\sffamily\bfseries,
	title=Problema~\thetcbcounter: #2,
	#1
}

% disponi codice
\usepackage{listings}
\usepackage[table]{xcolor}

\lstdefinestyle{codestyle}{
		backgroundcolor=\color{black!5}, 
		commentstyle=\color{codegreen},
		keywordstyle=\bfseries\color{magenta},
		numberstyle=\sffamily\tiny\color{black!60},
		stringstyle=\color{green!50!black},
		basicstyle=\ttfamily\footnotesize,
		breakatwhitespace=false,         
		breaklines=true,                 
		captionpos=b,                    
		keepspaces=true,                 
		numbers=left,                    
		numbersep=5pt,                  
		showspaces=false,                
		showstringspaces=false,
		showtabs=false,                  
		tabsize=2
}

\lstdefinestyle{shellstyle}{
		backgroundcolor=\color{black!5}, 
		basicstyle=\ttfamily\footnotesize\color{black}, 
		commentstyle=\color{black}, 
		keywordstyle=\color{black},
		numberstyle=\color{black!5},
		stringstyle=\color{black}, 
		showspaces=false,
		showstringspaces=false, 
		showtabs=false, 
		tabsize=2, 
		numbers=none, 
		breaklines=true
}

\lstdefinelanguage{javascript}{
	keywords={typeof, new, true, false, catch, function, return, null, catch, switch, var, if, in, while, do, else, case, break},
	keywordstyle=\color{blue}\bfseries,
	ndkeywords={class, export, boolean, throw, implements, import, this},
	ndkeywordstyle=\color{darkgray}\bfseries,
	identifierstyle=\color{black},
	sensitive=false,
	comment=[l]{//},
	morecomment=[s]{/*}{*/},
	commentstyle=\color{purple}\ttfamily,
	stringstyle=\color{red}\ttfamily,
	morestring=[b]',
	morestring=[b]"
}

% disponi sezioni
\usepackage{titlesec}

\titleformat{\section}
	{\sffamily\Large\bfseries} 
	{\thesection}{1em}{} 
\titleformat{\subsection}
	{\sffamily\large\bfseries}   
	{\thesubsection}{1em}{} 
\titleformat{\subsubsection}
	{\sffamily\normalsize\bfseries} 
	{\thesubsubsection}{1em}{}

% disponi alberi
\usepackage{forest}

\forestset{
	rectstyle/.style={
		for tree={rectangle,draw,font=\large\sffamily}
	},
	roundstyle/.style={
		for tree={circle,draw,font=\large}
	}
}

% disponi algoritmi
\usepackage{algorithm}
\usepackage{algorithmic}
\makeatletter
\renewcommand{\ALG@name}{Algoritmo}
\makeatother

% disponi numeri di pagina
\usepackage{fancyhdr}
\fancyhf{} 
\fancyfoot[L]{\sffamily{\thepage}}

\makeatletter
\fancyhead[L]{\raisebox{1ex}[0pt][0pt]{\sffamily{\@title \ \@date}}} 
\fancyhead[R]{\raisebox{1ex}[0pt][0pt]{\sffamily{\@author}}}
\makeatother

% disegni
\usepackage{pgfplots}
\pgfplotsset{width=10cm,compat=1.9}

\begin{document}
% sezione (data)
\section{Lezione del 23-09-24}

% stili pagina
\thispagestyle{empty}
\pagestyle{fancy}

% testo
\subsection{Introduzione}

\subsubsection{Programma del corso}
Il corso di ricerca operativa si divide in 4 parti:

\begin{enumerate}
	\item Modello di Programmazione Lineare;
	\item Programmazione Lineare su reti, ergo programmazione lineare su grafi;
	\item Programmazione Lineare intera, ergo programmazione lineare col vincolo $x \in \mathbb{Z}^n$;
	\item Programmazione Non Lineare.
\end{enumerate}

Le prime 3 parti hanno come prerequisiti l'algebra lineare: in particolare operazioni matriciali, prodotti scalari, sistemi lineari, teorema di Rouché-Capelli.
La quarta parte richiede invece conoscenze di Analisi II.

\subsubsection{Un problema di programmazione lineare}

La ricerca operativa si occupa di risolvere problemi di ottimizzazione con variabili decisionali e risorse limitate.
Poniamo un problema di esempio:

\begin{problem}{Produzione}
Una ditta produce due prodotti: \textbf{laminato A} e \textbf{laminato B}.
Ogni prodotto deve passare attraverso diversi reparti: il reparto \textbf{materie prime}, il reparto \textbf{taglio}, il reparto \textbf{finiture A} e il reparto \textbf{finiture B}.
Il guadagno è rispettivamente di 8.4 e 11.2 (unità di misura irrilevante) per ogni tipo di laminato.

Ora, nel reparto materie prime, il laminato A occupa 30, ore, e lo B 20 ore.
Nel reparto taglio il laminato A occupa 10 ore e lo B 20 ore.
Il laminato A occupa poi 20 ore nel reparto finiture A, mentre il laminato B occupa 30 ore nel reparto finiture B.
I reparti hanno a disposizione, rispettivamente, 120, 80, 62 e 105 ore.
Possiamo porre queste informazioni in forma tabulare:

	\center \rowcolors{2}{green!10}{green!40!black!20}
	\begin{tabular} { | c || c | c | c | }
		\hline
		\bfseries Reparto & \bfseries Capienza & \bfseries Laminato A & \bfseries Laminato B \\
		\hline 
		Materie prime & 120 & 30 & 20 \\
		Taglio & 80 & 10 & 20 \\
		Finiture A & 62 & 20 & / \\
		Finiture B & 105 & / & 30 \\
		\hline
		\textbf{Guadagno} & & 8.4 & 11.2 \\
		\hline
	\end{tabular}

	\par\bigskip

Quello che ci interessa è chiaramente massimizzare il guadagno.
\end{problem}

Decidiamo di modellizzare questa situazione con un modello matematico.

Il guadagno che abbiamo dai laminati rappresenta una \textbf{funzione obiettivo}, ovvero la funzione che vogliamo ottimizzare.
Ottimizzare significa trovare il modo migliore di massimizzare o minimizzare i valori della funzione agendo sulle variabili decisionali.
La funzione obiettivo va ottimizzata rispettando determinati \textbf{vincoli}, che modellizzano il fatto che le risorse sono limitate.
Una \textbf{soluzione ammissibile} è una qualsiasi soluzione che rispetta i vincoli del problema.
Chiamiamo quindi \textbf{regione ammissibile} l'insieme di tutte le soluzioni ammissibili.
All'interno della regione ammissibile c'è la soluzione che cerchiamo, ovvero la \textbf{soluzione ottima}.

Decidiamo quindi le \textbf{variabili decisionali}, ed esplicitiamo la funzione obiettivo e i vincoli.

In questo caso le variabili decisionali saranno le quantità di laminato A e B da produrre, che individuano un punto in $ \mathbb{R}^2 $ denominato $ ( x_A, x_B ) $. 
Decidere di usare la soluzione $ (1,1) $ significa decidere di produrre 1 unità di laminato A e 1 unità di laminato B, per un guadagno complessivo di $ 8.4 + 11.2 = 19.6 $.

La funzione obiettivo sarà quindi:

$$ f(x_A, x_B) = 8.4 x_A + 11.2 x_B, \quad f: \mathbb{R}^2 \rightarrow \mathbb{R} $$

lineare, e noi saremo interessati a:

$$ \max(f(x_A, x_B)) $$

rispettando i vincoli, ergo nella regione ammissibile.
Per esprimere questi vincoli, cioè il tempo limitato all'interno di ogni reparto, introduciamo il sistema di disequazioni:

\[
	\begin{cases}
		30 x_A + 20 x_B \leq 120 \\
		10 x_A + 20 x_B \leq 80	\\
		20 x_A + 0 x_B \leq 62 \\	
		0 x_A + 30 x_B \leq 105 \\
		- x_A \leq 0 \\
		- x_B \leq 0 \\
	\end{cases}
\]

dove notiamo le ultime due disequazioni indicano la positività di $x_A$ e $x_B$, in forma $ f(x_A, x_B) \leq b $.
Questo sistema non indica altro che la regione ammissibile.

Possiamo riscrivere questo modello usando la notazione dell'algebra lineare.
La funzione obiettiva e i vincoli diventano semplicemente:

\[
	\begin{cases}
		\max(c^T \cdot x) \\
		A \cdot x \leq b	
	\end{cases}
\]

dove $c$ rappresenta il vettore dei costi, $A$ rappresenta la matrice dei costi a $b$ il vettore dei vincoli.
$c$ è trasposto per indicare prodotto fra vettori.

Possiamo scrivere $A$, $b$ e $c$ per esteso:

$$
A:
\begin{pmatrix}
	30 & 20 \\
	10 & 20 \\
	20 & 0 \\
	0 & 30 \\
	-1 & 0 \\
	0 & -1
\end{pmatrix}, \quad
b:
\begin{pmatrix}
	120 \\
	80 \\
	62 \\ 
	105 \\ 
	0 \\ 
	0 
\end{pmatrix}, \quad 
c:
\begin{pmatrix}
	8.4 \\
	11.2 \\
\end{pmatrix}
$$

Notiamo come $A$ e $b$ hanno dimensione verticale $ 4 + 2 = 6 $, dai 4 vincoli superiori e i 2 vincoli inferiori.

A questo punto, possiamo disegnare la regione ammissibile come l'intersezione dei semipiani individuati da ogni singola disuguaglianza.
Si riporta un grafico:

\begin{tikzpicture}
\begin{axis}[
    axis lines = middle,
    xlabel = {$x_A$},
    ylabel = {$x_B$},
    xmin=0, xmax=6,
    ymin=0, ymax=6,
    domain=0:10,
    samples=100,
    width=10cm, height=10cm,
    legend pos=north east
  ]

% regione ammissibile

	\addplot[fill=gray, opacity=0.4] 
    coordinates {
			(0, 0)
			(3.1, 0)
			(3.1, 1.35)
			(2,3)
			(1, 3.5)
			(0, 3.5)
		};

% rette

\addplot[domain=2:3.1, thick, blue] {6 - 1.5*x}; 
\addlegendentry{$ 30 x_A + 20 x_B \leq 120 $}

\addplot[domain=1:2, thick, green] {4 - 0.5*x}; 
\addlegendentry{$ 10 x_A + 20 x_B \leq 80 $}

\addplot[thick, purple] coordinates {(3.1, 0) (3.1, 1.35)};
\addlegendentry{$ 20 x_A + 0 x_B \leq 62 $}

\addplot[domain=0:1, thick, red] {3.5}; 
\addlegendentry{$ 0 x_A + 30 x_B \leq 105 $}
	
\end{axis}
\end{tikzpicture}

In diversi colori sono riportate i margini delle disequazioni, mentre in grigio è evidenziata la regione ammissibile.

\par\smallskip

Il modello finora descritto prende il nome di modello di programmazione lineare, e permette di formulare problemi di programmazione lineare (LP).

\begin{definition}{Problema di programmazione lineare (1)}
Un problema di programmazione lineare (LP) riguarda l'ottimizzazione di una funzione lineare in più variabili
soggetta a vincoli di tipo $ =, \ \leq $ e $ \geq $, ovvero in forma:
\[
	\begin{cases}
			\min / \max(c^T \cdot x) \\
			A x \leq b \\
			... \\
			B x \geq d \\
			... \\
			C x = e \\
			...
	\end{cases}
\]
\end{definition}

"Programmazione" qui non ha alcun legame col concetto di programmazione informatica, ma si riferisce al fatto che il modello è effettivamente \textit{programmabile}.

"Lineare" si riferisce alla linearità del problema (e quindi del modello).




\end{document}

\documentclass[a4paper,11pt]{article}
\usepackage[a4paper, margin=8em]{geometry}

% usa i pacchetti per la scrittura in italiano
\usepackage[french,italian]{babel}
\usepackage[T1]{fontenc}
\usepackage[utf8]{inputenc}
\frenchspacing 

% usa i pacchetti per la formattazione matematica
\usepackage{amsmath, amssymb, amsthm, amsfonts}

% usa altri pacchetti
\usepackage{gensymb}
\usepackage{hyperref}
\usepackage{standalone}

% imposta il titolo
\title{Appunti Ricerca Operativa}
\author{Luca Seggiani}
\date{24-09-24}

% imposta lo stile
% usa helvetica
\usepackage[scaled]{helvet}
% usa palatino
\usepackage{palatino}
% usa un font monospazio guardabile
\usepackage{lmodern}

\renewcommand{\rmdefault}{ppl}
\renewcommand{\sfdefault}{phv}
\renewcommand{\ttdefault}{lmtt}

% disponi teoremi
\usepackage{tcolorbox}
\newtcolorbox[auto counter, number within=section]{theorem}[2][]{%
	colback=blue!10, 
	colframe=blue!40!black, 
	sharp corners=northwest,
	fonttitle=\sffamily\bfseries, 
	title=Teorema~\thetcbcounter: #2, 
	#1
}

% disponi definizioni
\newtcolorbox[auto counter, number within=section]{definition}[2][]{%
	colback=red!10,
	colframe=red!40!black,
	sharp corners=northwest,
	fonttitle=\sffamily\bfseries,
	title=Definizione~\thetcbcounter: #2,
	#1
}

% disponi codice
\usepackage{listings}
\usepackage[table]{xcolor}

\lstdefinestyle{codestyle}{
		backgroundcolor=\color{black!5}, 
		commentstyle=\color{codegreen},
		keywordstyle=\bfseries\color{magenta},
		numberstyle=\sffamily\tiny\color{black!60},
		stringstyle=\color{green!50!black},
		basicstyle=\ttfamily\footnotesize,
		breakatwhitespace=false,         
		breaklines=true,                 
		captionpos=b,                    
		keepspaces=true,                 
		numbers=left,                    
		numbersep=5pt,                  
		showspaces=false,                
		showstringspaces=false,
		showtabs=false,                  
		tabsize=2
}

\lstdefinestyle{shellstyle}{
		backgroundcolor=\color{black!5}, 
		basicstyle=\ttfamily\footnotesize\color{black}, 
		commentstyle=\color{black}, 
		keywordstyle=\color{black},
		numberstyle=\color{black!5},
		stringstyle=\color{black}, 
		showspaces=false,
		showstringspaces=false, 
		showtabs=false, 
		tabsize=2, 
		numbers=none, 
		breaklines=true
}

\lstdefinelanguage{javascript}{
	keywords={typeof, new, true, false, catch, function, return, null, catch, switch, var, if, in, while, do, else, case, break},
	keywordstyle=\color{blue}\bfseries,
	ndkeywords={class, export, boolean, throw, implements, import, this},
	ndkeywordstyle=\color{darkgray}\bfseries,
	identifierstyle=\color{black},
	sensitive=false,
	comment=[l]{//},
	morecomment=[s]{/*}{*/},
	commentstyle=\color{purple}\ttfamily,
	stringstyle=\color{red}\ttfamily,
	morestring=[b]',
	morestring=[b]"
}

% disponi sezioni
\usepackage{titlesec}

\titleformat{\section}
	{\sffamily\Large\bfseries} 
	{\thesection}{1em}{} 
\titleformat{\subsection}
	{\sffamily\large\bfseries}   
	{\thesubsection}{1em}{} 
\titleformat{\subsubsection}
	{\sffamily\normalsize\bfseries} 
	{\thesubsubsection}{1em}{}

% disponi alberi
\usepackage{forest}

\forestset{
	rectstyle/.style={
		for tree={rectangle,draw,font=\large\sffamily}
	},
	roundstyle/.style={
		for tree={circle,draw,font=\large}
	}
}

% disponi algoritmi
\usepackage{algorithm}
\usepackage{algorithmic}
\makeatletter
\renewcommand{\ALG@name}{Algoritmo}
\makeatother

% disponi numeri di pagina
\usepackage{fancyhdr}
\fancyhf{} 
\fancyfoot[L]{\sffamily{\thepage}}

\makeatletter
\fancyhead[L]{\raisebox{1ex}[0pt][0pt]{\sffamily{\@title \ \@date}}} 
\fancyhead[R]{\raisebox{1ex}[0pt][0pt]{\sffamily{\@author}}}
\makeatother

\begin{document}
% sezione (data)
\section{Lezione del 24-09-24}

% stili pagina
\thispagestyle{empty}
\pagestyle{fancy}

% testo

\subsection{Forma primale standard}
Ciò che abbiamo formulato finora è un problema di programmazione lineare.
Possiamo dire che la forma:
\[
	\begin{cases}
			\max(c^T \cdot x) \\
			A x \leq b
	\end{cases}
\]
rappresenta un problema LP in forma \textbf{primale standard}, ricordando che $c$ è il vettore dei coefficienti della funzione obiettivo, $A$ la matrice dei coefficienti per ogni vincolo, e $b$ il vettore dei termini noti per ogni vincolo.

\begin{definition}{Forma primale standard}
	Un problema di programmazione lineare si dice in forma primale standard quando è espresso in forma:
	
	\[
		\begin{cases}
			\max(c^T \cdot x) \\
			Ax \leq b \\
		\end{cases}
	\]

\end{definition}
\par\smallskip
Si adotta una forma primale standard in quanto si può trasformare ogni problema LP in una forma di questo tipo.

\subsubsection{Normalizzazione di un problema LP}
Un modo per portare un problema LP qualsiasi in forma primale standard è:

\begin{enumerate}
	\item Si trasformano le disuguaglianze: $ \geq \ \leftrightarrow \ \leq $
	\item Si riscrivono le uguaglianze come coppie di diseguaglianze:
		$$
			f(x) = c \ \rightarrow \
		\begin{cases}
			f(x) \leq c \\ 
			f(x) \geq c
		\end{cases}
		$$
		da cui la (1):
		$$
			f(x) = c \ \rightarrow \
S		\begin{cases}
			f(x) \leq c \\ 
			-f(x) \leq -c
		\end{cases}
		$$
	\item Se il problema richiede il minimo, si nota che $ \max(f) = -\min(-f) $, e sopratutto:
		$$
		\bar{x} \in \mathrm{argmax}(f) \Leftrightarrow \bar{x} \in \mathrm{argmin}(-f)
		$$
		con $ \mathrm{argmax}(f) $ e $ \mathrm{argmin}(-f) $ rispettivamente gli insiemi dei punti di massimo e minimo.
		Questo significa che posso semplicemente cambiare di segno la funzione obiettivo per trovare da massimi minimi, e viceversa.
\end{enumerate}

Notiamo inoltre che, nella forma primale standard, si ha:
$$
	x \in R^n, \quad
	A \in R^{n \times m}, \quad
	b \in R^m, \quad
	c \in R^n
$$

\subsection{Proprietà generali di un problema LP}
La regione ammissibile di un problema PL si chiama \textbf{poliedro}.
Si può dare agilmente una definizione algebrica di poliedro:
\begin{definition}{Definizione algebrica di poliedro}
	Algebricamente, un poliedro è l'insieme delle soluzioni di un sistema di disequazioni lineari in $\mathbb{R}^n$ variabili:
	$$
		P = \{ x \in \mathbb{R}^n : Ax \leq b \}
	$$
\end{definition}

Questa regione in un problema LP prende il nome di regione ammissibile.

\begin{definition}{Definizione geometrica di poliedro}
	Geometricamente, un poliedro è l'intersezione di un numero finito di semispazi chiusi.
\end{definition}

I semispazi chiaramente sono lineari, e in $\mathbb{R}^2$ rappresenterebbero semipiani.
Chiusi significa che nelle disequazioni che descrivono i vincoli compargono solo $\leq$ e non $<$, ergo la regione ammissibile contiene la sua frontiera.

Possiamo dimostrare 4 proprietà dei poliedri:

\begin{enumerate}
	\item 
		Un'osservazione fondamentale è la seguente:
		\begin{theorem}{Soluzione ottimale di un problema LP}
			La soluzione ottimale di un problema LP è contenuta nella frontiera della regione ammissibile.
		\end{theorem}
		Questo si può ricavare dai teoremi di Fermat e di Weierstrass, e dalla convessità della regione ammissibile.
		Inanzitutto, si è stabilito che la soluzione ottimale non è altro che il massimo o minimo assoluto all'interno della regione ammissibile del problema.
		Il gradiente della funzione obiettiva è $\neq 0$ in ogni suo punto (funzione lineare a gradiente costante). 
		Da Fermat, i massimi e minimi hanno sempre gradiente $0$, ergo massimi o minimi locali (che esistono per Weierstrass) possono trovarsi solo sulla frontiera.
		A questo punto, possiamo imporre la convessità per asserire che quei punti di massimo o minimo sono anche globali. 

	\item 
		Prendiamo in esempio il poliedro dato da:
		\[
			\begin{cases}
				x_A > 0 \\ 
				x_B > 0
			\end{cases}
		\]
		o se vogliamo, in forma primale standard, dato dalle matrici $A$ e $b$:
		$$
		A:
		\begin{pmatrix}
			-1 & 0 \\	
			0 & -1 \\	
		\end{pmatrix}
		, \quad b:
		\begin{pmatrix}
			0 \\ 
			0
		\end{pmatrix}
		$$
		questo poliedro non è limitato nella direzione positiva, ergo può arrivare a valori di $x_A$ e $x_B$ che tendono a $+\infty$.
		Da ciò si ha che può accadere che un problema LP ammetta soluzioni $x$ tali che $x \rightarrow \pm \infty $,
		ovvero che il poliedro sia illimitato.
		In particolare, un poliedro limitato si dice \textbf{politopo}.
	\item Notiamo poi che la soluzione di un problema LP può non essere unica.
		Questo accade ad esempio quando la soluzione sta su un segmento di frontiera: a quel punto tutti i punti del segmento sono soluzione.
		Da questo segue che:
		\begin{theorem}{Unicità della soluzione ottimale di un problema LP}
			Se un problema LP ha almeno 2 soluzioni, allora ne ha infinite.
		\end{theorem}
		Ciò si può dimostrare come segue.
		Si riporta innanzitutto la notazione parametrica del segmento $\bar{zw}$, dati i due vettori di estremo $z$ e $w$:
		$$
			\lambda z + (1 - \lambda)w, \quad \lambda \in [ 0, 1 ]
		$$
		A questo punto si pone che $z$ e $w$ sono entrambi soluzioni ottime, ergo: 
		$$ 
			\max(c^T \cdot x) = c^T z = c^T w = v 
		$$
		da cui si può dire che:
		$$ 
			c^T\left(\lambda z + (1 - \lambda)w\right) = \lambda c^T z + (1 - \lambda) c^T w = \lambda v + (1 - \lambda) v = v 
		$$
		Ovvero ogni punto sul segmento porta la funzione obiettiva a massimo assoluto, quindi è soluzione ottimale.
	\item Infine, notiamo che il poliedro della regione ammissible di un problema LP può essere vuoto, ergo $P = \emptyset$.
		In questo caso, si ha che $ \max(c^T \cdot x) = -\infty $ e $ \min(c^T \cdot x) = \infty $. 
		Un poliedro vuoto significa che i vincoli stessi vanno modificati. 
		Questo solitamente si fa risovendo una versione semplificata del problema LP.
\end{enumerate}

Si può fare un'altro esempio per sottolineare l'importanza del punto di massimo (o minimo), e non quel massimo (o minimo).
Finché nella funzione obiettivo i coefficienti compargono nello stesso rapporto (ergo finché si scelgono vettori $c$ linearmente dipendenti), il punto di massimo (o minimo) non cambia, per via della linearità (e si presume omogeneità) della funzione obiettiva stessa.
Sarà solo il massimo (o minimo) a variare di un rapporto pari a quello di cui variano i coefficienti.

\subsection{Gradiente e linee di isocosto}
Si può dimostrare il seguente teorema:
\begin{theorem}{Gradiente della funzione obiettivo}
	Il gradiente di una funzione obiettivo definita come $ f(x) = c^T \cdot x $ sulla base di un qualche vettore $c$ è in ogni punto il vettore $c$ stesso.
\end{theorem}
Da questo gradiente si possono ricavare le cosiddette linee di isocosto (in dimensioni $>2$ sarebbero superfici), cioè linee a valore costante della funzione obiettivo.
\begin{definition}{Linea di isocosto}
	Si definisce linea di isocosto di una funzione obiettivo con vettore $c$ una retta (o superficie):	
	$$ f(x) = c^T \cdot x = k $$
	per un qualsiasi $k$ costante.
\end{definition}

\subsection{Cono di competenza}
Dovrebbe essere chiaro adesso che i punti di soluzione ottima stanno tutti su un segmento o su un punto della frontiera.
Nel caso si abbia un vettore gradiente perpendicolare ad un segmento della frontiera, quel segmento sarà soluzione ottima. In caso contrario, spostandoci a destra avremo l'estremo destro del segmento, e spostandoci a sinistra viceversa, finché non si diventerà perpendicolari a qualche altro segmento di frontiera.

Il cono (in $R^2$, angolo) di valori possibili del gradiente che rendono un punto ottimale prende il nome di \textbf{cono di competenza}.
\begin{definition}{Cono di competenza}
	Il cono di competenza di un punto $x^*$ è il cono, ovvero l'insieme di vettori gradiente, tale per cui il punto $x^*$ conserva l'ottimalità sulla funzione obiettivo coi vincoli imposti.
\end{definition}

Vedremo in seguito l'importanza di una nozione di cono per i problemi LP.

\end{document}


\documentclass[a4paper,11pt]{article}
\usepackage[a4paper, margin=8em]{geometry}

% usa i pacchetti per la scrittura in italiano
\usepackage[french,italian]{babel}
\usepackage[T1]{fontenc}
\usepackage[utf8]{inputenc}
\frenchspacing 

% usa i pacchetti per la formattazione matematica
\usepackage{amsmath, amssymb, amsthm, amsfonts}

% usa altri pacchetti
\usepackage{gensymb}
\usepackage{hyperref}
\usepackage{standalone}

% imposta il titolo
\title{Appunti /home/luca/Desktop/Uni/appunti/Ricerca Operativa}
\author{Luca Seggiani}
\date{2024}

% disegni
\usepackage{pgfplots}
\pgfplotsset{width=10cm,compat=1.9}

% imposta lo stile
% usa helvetica
\usepackage[scaled]{helvet}
% usa palatino
\usepackage{palatino}
% usa un font monospazio guardabile
\usepackage{lmodern}

\renewcommand{\rmdefault}{ppl}
\renewcommand{\sfdefault}{phv}
\renewcommand{\ttdefault}{lmtt}

% disponi il titolo
\makeatletter
\renewcommand{\maketitle} {
	\begin{center} 
		\begin{minipage}[t]{.8\textwidth}
			\textsf{\huge\bfseries \@title} 
		\end{minipage}%
		\begin{minipage}[t]{.2\textwidth}
			\raggedleft \vspace{-1.65em}
			\textsf{\small \@author} \vfill
			\textsf{\small \@date}
		\end{minipage}
		\par
	\end{center}

	\thispagestyle{empty}
	\pagestyle{fancy}
}
\makeatother

% disponi teoremi
\usepackage{tcolorbox}
\newtcolorbox[auto counter, number within=section]{theorem}[2][]{%
	colback=blue!10, 
	colframe=blue!40!black, 
	sharp corners=northwest,
	fonttitle=\sffamily\bfseries, 
	title=Teorema~\thetcbcounter: #2, 
	#1
}

% disponi definizioni
\newtcolorbox[auto counter, number within=section]{definition}[2][]{%
	colback=red!10,
	colframe=red!40!black,
	sharp corners=northwest,
	fonttitle=\sffamily\bfseries,
	title=Definizione~\thetcbcounter: #2,
	#1
}

% disponi problemi
\newtcolorbox[auto counter, number within=section]{problem}[2][]{%
	colback=green!10,
	colframe=green!40!black,
	sharp corners=northwest,
	fonttitle=\sffamily\bfseries,
	title=Problema~\thetcbcounter: #2,
	#1
}

% disponi codice
\usepackage{listings}
\usepackage[table]{xcolor}

\lstdefinestyle{codestyle}{
		backgroundcolor=\color{black!5}, 
		commentstyle=\color{codegreen},
		keywordstyle=\bfseries\color{magenta},
		numberstyle=\sffamily\tiny\color{black!60},
		stringstyle=\color{green!50!black},
		basicstyle=\ttfamily\footnotesize,
		breakatwhitespace=false,         
		breaklines=true,                 
		captionpos=b,                    
		keepspaces=true,                 
		numbers=left,                    
		numbersep=5pt,                  
		showspaces=false,                
		showstringspaces=false,
		showtabs=false,                  
		tabsize=2
}

\lstdefinestyle{shellstyle}{
		backgroundcolor=\color{black!5}, 
		basicstyle=\ttfamily\footnotesize\color{black}, 
		commentstyle=\color{black}, 
		keywordstyle=\color{black},
		numberstyle=\color{black!5},
		stringstyle=\color{black}, 
		showspaces=false,
		showstringspaces=false, 
		showtabs=false, 
		tabsize=2, 
		numbers=none, 
		breaklines=true
}

\lstdefinelanguage{javascript}{
	keywords={typeof, new, true, false, catch, function, return, null, catch, switch, var, if, in, while, do, else, case, break},
	keywordstyle=\color{blue}\bfseries,
	ndkeywords={class, export, boolean, throw, implements, import, this},
	ndkeywordstyle=\color{darkgray}\bfseries,
	identifierstyle=\color{black},
	sensitive=false,
	comment=[l]{//},
	morecomment=[s]{/*}{*/},
	commentstyle=\color{purple}\ttfamily,
	stringstyle=\color{red}\ttfamily,
	morestring=[b]',
	morestring=[b]"
}

% disponi sezioni
\usepackage{titlesec}

\titleformat{\section}
	{\sffamily\Large\bfseries} 
	{\thesection}{1em}{} 
\titleformat{\subsection}
	{\sffamily\large\bfseries}   
	{\thesubsection}{1em}{} 
\titleformat{\subsubsection}
	{\sffamily\normalsize\bfseries} 
	{\thesubsubsection}{1em}{}

% disponi alberi
\usepackage{forest}

\forestset{
	rectstyle/.style={
		for tree={rectangle,draw,font=\large\sffamily}
	},
	roundstyle/.style={
		for tree={circle,draw,font=\large}
	}
}

% disponi algoritmi
\usepackage{algorithm}
\usepackage{algorithmic}
\makeatletter
\renewcommand{\ALG@name}{Algoritmo}
\makeatother

% disponi numeri di pagina
\usepackage{fancyhdr}
\fancyhf{} 
\fancyfoot[L]{\sffamily{\thepage}}

\makeatletter
\fancyhead[L]{\raisebox{1ex}[0pt][0pt]{\sffamily{\@title \ \@date}}} 
\fancyhead[R]{\raisebox{1ex}[0pt][0pt]{\sffamily{\@author}}}
\makeatother

\begin{document}

% sezione (data)
\section{Lezione del 25-09-24}

% stili pagina
\thispagestyle{empty}
\pagestyle{fancy}

% testo
\subsection{Assegnamento di costo minimo}
Vediamo un problema:
\begin{problem}{Assegnamento}
	Quattro agenzie possono occuparsi di 4 progetti.
	Ogni agenzia presenta il costo stimato per la realizzazione di ogni progetto, in migliaia di euro.
	In forma tabulare, si riportano i valori:

	\center \rowcolors{2}{green!10}{green!40!black!20}
	\begin{tabular} { | c | c | c | c | c | }
		\hline
		& \bfseries Agenzia 1 & \bfseries Agenzia 2 & \bfseries Agenzia 3 & \bfseries Agenzia 4 \\
		\hline 
		\bfseries Progetto 1 & 20 & 17 & 16 & 14 \\
		\bfseries Progetto 2 & 22 & 16 & 19 & 15 \\
		\bfseries Progetto 3 & 21 & 17 & 15 & 23 \\ 
		\bfseries Progetto 4 & 19 & 18 & 14 & 24 \\
		\hline
	\end{tabular}

	\par\bigskip
	
	Vogliamo capire quale agenzia deve occuparsi di quale progetto per minimizzare i costi.

\end{problem}

Con $n$ agenzie e progetti ci sono $n!$ possibili combinazioni, ergo dobbiamo trovare un algoritmo più efficiente. 
Applicando il modello studiato finora, abbiamo la matrice dei costi $c$:

$$
\begin{pmatrix}
 20 & 17 & 16 & 14 \\
 22 & 16 & 19 & 15 \\
 21 & 17 & 15 & 23 \\
 19 & 18 & 14 & 24 \\
\end{pmatrix}
$$

che possiamo portare a:
$$ c: ( -18, +18 + ... + 24 ) $$
come linearizzazione lessicografica della tabella sopra riportata (notare che sarebbe un vettore colonna).

Adesso dobbiamo solo trovare un metodo per esplicitare i vincoli del problema:
\begin{itemize}
	\item Ogni agenzia può occuparsi solo di un progetto;
	\item Ogni progetto richiede solo un'agenzia.
\end{itemize}

Possiamo rappresentare la corrispondenza fra elementi come un vettore, e quindi riportarne una matrice d'adiacenza.
Assumendo di appaiare elementi con lo stesso numero, avremo:
$$
\begin{pmatrix}
	1 & 0 & 0 & 0 \\ 
	0 & 1 & 0 & 0 \\ 
	0 & 0 & 1 & 0 \\ 
	0 & 0 & 0 & 1 \\ 
\end{pmatrix}
$$

La caratteristica di questa matrice, chiamiamola $x$, e che ogni elemento $x_{ij}$ è:
$$
x_{ij} = 
	\begin{cases}
		0 \\ 1	
	\end{cases}
$$

Decidiamo di trattare la $x$ come un vettore linearizzato lessicograficamente dalla matrice, proprio come avevamo fatto per il vettore costo.
Per una matrice di adiacenza $n \times n$, di $n$ elementi in ogni categoria, questo vettore ha dimensione $n^2$. 
Questo semplifica la notazione del problema, e sopratutto della matrice $A$, che sarebbe lasciando $x$ matrice effettivamente un tensore.

Si dimostra quindi facilmente che i vincoli riportati prima possono quindi esprimersi come:
\[
	\begin{cases}
		x_{11} + x_{12} + x_{13} + x_{14} = 1	\\
		x_{21} + x_{22} + x_{23} + x_{24} = 1 \\ 
		x_{31} + x_{32} + x_{33} + x_{34} = 1 \\ 
		x_{41} + x_{42} + x_{43} + x_{44} = 1 \\ 
	\end{cases}
\]

per il primo punto, e:
\[
	\begin{cases}
		x_{11} + x_{21} + x_{31} + x_{41} = 1	\\
		x_{12} + x_{22} + x_{32} + x_{42} = 1 \\ 
		x_{13} + x_{23} + x_{33} + x_{43} = 1 \\ 
		x_{14} + x_{24} + x_{34} + x_{44} = 1 \\ 
	\end{cases}
\]

per il secondo.
Imponendo la positività, si hanno quindi le matrici $A$ e $b$:

$$
A: 
\begin{pmatrix}
	1 & 1 & 1 & 1 & 0... & &  & & & ...0 \\
	0... & & ...0 & 1 & 1 & 1 & 1 & 0... & & ...0 \\
	... \\
	0... & &  & & & ...0 & 1 & 1 & 1 & 1 \\
	...\\
	-1 & 0... & & & & & & & & ...0 \\
	...\\
	0... & & & & & & & & & -1 \\
\end{pmatrix}, \quad 
b: ( 1, ..., 1, 0, ..., 0)
$$

Si nota che il numero di vincoli necessari per $n$ elementi è $2n + n^2$.

\subsubsection{Assegnamento cooperativo e non cooperativo}
A questo punto conviene fare una distinzione.
Abbiamo definito finora il modello:

\[
	\begin{cases}
		\min{c^T \cdot x} \\
		x_{11} + x_{12} + x_{13} + x_{14} = 1	\\
		...\\
		x_{41} + x_{42} + x_{43} + x_{44} = 1 \\
		x_{11} + x_{21} + x_{31} + x_{41} = 1	\\
		...\\
		x_{14} + x_{24} + x_{34} + x_{44} = 1 \\ 
 
	\end{cases}
\]

che così scritto non nega la possibilità di $x$ con componenti reali.
Nell'esempio ciò significa sono ammesse soluzioni dove più agenzie danno contributi reali ai progetti, che possiamo semanticamente interpretare come condividere il carico di lavoro, pur rispettando i vincoli imposti.
Decidiamo che questo è corretto se si parla di un problema di \textbf{assegnamento cooperativo}.
Visto che il problema posto non era di questo tipo, ma era di \textbf{assegnamento non cooperativo}, si introduce un'ulteriore vincolo:

$$
	x \in \mathbb{Z}^n
$$

Adesso ogni azienda darà un contributo intero al suo progetto, ergo coi vincoli imposti prima, ogni azienda sarà unica nel dirigere un solo progetto.

Più formalmente, possiamo dire che il passaggio ad assegnamento cooperativo comporta un \textbf{rilassamento} dei vincoli del problema.
Ovvero, in generale, se un problema non cooperativo ha minimo ottimale $nc$, il suo associato cooperativo avrà minimo ottimale $c$ con:

$$  c \leq nc $$

\subsubsection{Forma primale standard}
Portiamo quindi questo problema in una forma primale simile a quella vista per altri problemi LP, concesso il vincolo $x \in \mathbb{Z}^n$.

Finora avevamo usato le trasformazioni equivalenti per problemi LP:

\begin{enumerate}
	\item Trasformazione delle disuguaglianze: $ \geq \ \leftrightarrow \ \leq $
	\item Trasformazione delle uguaglianze:
		$$
			f(x) = c \ \rightarrow \
		\begin{cases}
			f(x) \leq c \\ 
			-f(x) \leq -c
		\end{cases}
		$$
	\item Trasformazione minimo / massimo: 
		$$
		\max{f} = -\min{f} \ \text{e sopratutto} \ 
		\bar{x} \in \mathrm{argmax}(f) \Leftrightarrow \bar{x} \in \mathrm{argmin}(-f)
		$$
\end{enumerate}

Possiamo applicare queste trasformazioni al modello, in particolare la (2), che porta il numero di vincoli a $4n + n^2$.

\subsection{Introduzione di surplus}
Vediamo un ulteriore tecnica per trasformare problemi LP: si può portare una disequazione del tipo:
$$ a_1x_1 + a_2x_2 + ... + a_nx_n \leq b $$
in un uguaglianza introducendo una variabile ausiliaria $s$:
$$ a_1x_1 + a_2x_2 + ... + a_nx_n + s = b $$
$s$ prende il nome di \textbf{slack}, in italiano scarto, o \textit{surplus}.
\end{document}


\documentclass[a4paper,11pt]{article}
\usepackage[a4paper, margin=8em]{geometry}

% usa i pacchetti per la scrittura in italiano
\usepackage[french,italian]{babel}
\usepackage[T1]{fontenc}
\usepackage[utf8]{inputenc}
\frenchspacing 

% usa i pacchetti per la formattazione matematica
\usepackage{amsmath, amssymb, amsthm, amsfonts}

% usa altri pacchetti
\usepackage{gensymb}
\usepackage{hyperref}
\usepackage{standalone}

% imposta il titolo
\title{Appunti Ricerca Operativa}
\author{Luca Seggiani}
\date{2024}

% disegni
\usepackage{pgfplots}
\pgfplotsset{width=10cm,compat=1.9}

% imposta lo stile
% usa helvetica
\usepackage[scaled]{helvet}
% usa palatino
\usepackage{palatino}
% usa un font monospazio guardabile
\usepackage{lmodern}

\renewcommand{\rmdefault}{ppl}
\renewcommand{\sfdefault}{phv}
\renewcommand{\ttdefault}{lmtt}

% disponi il titolo
\makeatletter
\renewcommand{\maketitle} {
	\begin{center} 
		\begin{minipage}[t]{.8\textwidth}
			\textsf{\huge\bfseries \@title} 
		\end{minipage}%
		\begin{minipage}[t]{.2\textwidth}
			\raggedleft \vspace{-1.65em}
			\textsf{\small \@author} \vfill
			\textsf{\small \@date}
		\end{minipage}
		\par
	\end{center}

	\thispagestyle{empty}
	\pagestyle{fancy}
}
\makeatother

% disponi teoremi
\usepackage{tcolorbox}
\newtcolorbox[auto counter, number within=section]{theorem}[2][]{%
	colback=blue!10, 
	colframe=blue!40!black, 
	sharp corners=northwest,
	fonttitle=\sffamily\bfseries, 
	title=Teorema~\thetcbcounter: #2, 
	#1
}

% disponi definizioni
\newtcolorbox[auto counter, number within=section]{definition}[2][]{%
	colback=red!10,
	colframe=red!40!black,
	sharp corners=northwest,
	fonttitle=\sffamily\bfseries,
	title=Definizione~\thetcbcounter: #2,
	#1
}

% disponi problemi
\newtcolorbox[auto counter, number within=section]{problem}[2][]{%
	colback=green!10,
	colframe=green!40!black,
	sharp corners=northwest,
	fonttitle=\sffamily\bfseries,
	title=Problema~\thetcbcounter: #2,
	#1
}

% disponi codice
\usepackage{listings}
\usepackage[table]{xcolor}

\lstdefinestyle{codestyle}{
		backgroundcolor=\color{black!5}, 
		commentstyle=\color{codegreen},
		keywordstyle=\bfseries\color{magenta},
		numberstyle=\sffamily\tiny\color{black!60},
		stringstyle=\color{green!50!black},
		basicstyle=\ttfamily\footnotesize,
		breakatwhitespace=false,         
		breaklines=true,                 
		captionpos=b,                    
		keepspaces=true,                 
		numbers=left,                    
		numbersep=5pt,                  
		showspaces=false,                
		showstringspaces=false,
		showtabs=false,                  
		tabsize=2
}

\lstdefinestyle{shellstyle}{
		backgroundcolor=\color{black!5}, 
		basicstyle=\ttfamily\footnotesize\color{black}, 
		commentstyle=\color{black}, 
		keywordstyle=\color{black},
		numberstyle=\color{black!5},
		stringstyle=\color{black}, 
		showspaces=false,
		showstringspaces=false, 
		showtabs=false, 
		tabsize=2, 
		numbers=none, 
		breaklines=true
}

\lstdefinelanguage{javascript}{
	keywords={typeof, new, true, false, catch, function, return, null, catch, switch, var, if, in, while, do, else, case, break},
	keywordstyle=\color{blue}\bfseries,
	ndkeywords={class, export, boolean, throw, implements, import, this},
	ndkeywordstyle=\color{darkgray}\bfseries,
	identifierstyle=\color{black},
	sensitive=false,
	comment=[l]{//},
	morecomment=[s]{/*}{*/},
	commentstyle=\color{purple}\ttfamily,
	stringstyle=\color{red}\ttfamily,
	morestring=[b]',
	morestring=[b]"
}

% disponi sezioni
\usepackage{titlesec}

\titleformat{\section}
	{\sffamily\Large\bfseries} 
	{\thesection}{1em}{} 
\titleformat{\subsection}
	{\sffamily\large\bfseries}   
	{\thesubsection}{1em}{} 
\titleformat{\subsubsection}
	{\sffamily\normalsize\bfseries} 
	{\thesubsubsection}{1em}{}

% disponi alberi
\usepackage{forest}

\forestset{
	rectstyle/.style={
		for tree={rectangle,draw,font=\large\sffamily}
	},
	roundstyle/.style={
		for tree={circle,draw,font=\large}
	}
}

% disponi algoritmi
\usepackage{algorithm}
\usepackage{algorithmic}
\makeatletter
\renewcommand{\ALG@name}{Algoritmo}
\makeatother

% disponi numeri di pagina
\usepackage{fancyhdr}
\fancyhf{} 
\fancyfoot[L]{\sffamily{\thepage}}

\makeatletter
\fancyhead[L]{\raisebox{1ex}[0pt][0pt]{\sffamily{\@title \ \@date}}} 
\fancyhead[R]{\raisebox{1ex}[0pt][0pt]{\sffamily{\@author}}}
\makeatother

\begin{document}

% sezione (data)
\section{Lezione del 26-09-24}

% stili pagina
\thispagestyle{empty}
\pagestyle{fancy}

% testo
\subsection{Geometria dei poliedri}
Introduciamo progressivamente i tipi di \textbf{combinazione} che ci sono utili nello studio dei problemi di programmazione lineare.

\subsubsection{Combinazioni lineari}
\begin{definition}{Combinazione lineare}
	Dati $ x_1, x_2, ..., x_k \in \mathbb{R}^n $ punti, $y$ si dice \textbf{combinazione lineare} di $ x_1, x_2, ..., x_k $ se:
	$$
	\exists \lambda_i \quad (i = 1, ..., k) \quad \text{t.c.} \quad y = \sum_{i=1}^k \lambda_i x_i 
	$$
\end{definition}

Le combinazioni lineari sono utili per esprimere la funzione obiettiva sulla base dei vettori costo, ma non bastano a trovarne una soluzione ottimale.

\subsubsection{Combinazioni convesse}
Si introduce quindi il concetto di:
\begin{definition}{Combinazione convessa}
	Dati $ x_1, x_2, ..., x_k \in \mathbb{R}^n $ punti, $y$ si dice \textbf{combinazione convessa} di $ x_1, x_2, ..., x_k $ se:
	$$
	\exists \lambda_i \in [0, 1] \quad (i = 1, ..., k), \quad \sum_{i=1}^k \lambda_i = 1 \quad \text{t.c.} \quad y = \sum_{i=1}^k \lambda_i x_i 
	$$
\end{definition}

Possiamo dare un esempio di cos'è la combinazione convessa di due punti in $\mathbb{R}^2$.
Posti $x_1$ e $x_2$, si ha:
$$
\lambda_1 + \lambda_2 = 1 \Rightarrow \lambda_2 = (1 - \lambda_1), \quad y = \lambda x_1 + (1 - \lambda ) x_2, \quad \lambda \in [0, 1]
$$
che riconosciamo essere l'equazione di un segmento $\bar{x_1x_2}$ (primo grafico).

Possiamo provare con tre punti: si avrà:
$$
y = \lambda_1 x_1 + \lambda_2 x_2 + \lambda_3 x_3, \quad \lambda_1 + \lambda_2 + \lambda_3 = 1, \quad \lambda_i \in [0, 1]
$$
che si riconduce all'equazione del triangolo di vertici $x_1$, $x_2$, $x_3$ (secondo grafico).
\par\medskip

\begin{minipage}{0.45\textwidth}
    % Segment on a graph
	\begin{tikzpicture}[scale=0.75]
    \begin{axis}[
        axis lines = center,
        xlabel = $x$, ylabel = $y$,
        xmin= -0.8, xmax=4.8, ymin=-0.8, ymax=4.8,
        grid = minor,
				title = {Combinazione di $x_1$ e $x_2$}
    ]
        \addplot[thick] coordinates {(1,2) (4,2)};
        \addplot[only marks, mark=*] coordinates {(1,2)} node[anchor=north] {$x_1$};
        \addplot[only marks, mark=*] coordinates {(4,2)} node[anchor=north] {$x_2$};
    \end{axis}
    \end{tikzpicture}
\end{minipage}
\hfill
\begin{minipage}{0.45\textwidth}
    % Triangle on a graph
    \begin{tikzpicture}[scale=0.75]
    \begin{axis}[
        axis lines = center,
        xlabel = $x$, ylabel = $y$,
        xmin= -0.8, xmax=4.8, ymin=-0.8, ymax=4.8,
        grid = minor,
				title = {Combinazione di $x_1$, $x_2$ e $x_3$}
    ]
        \addplot[thick] coordinates {(1,1) (4,1) (2,3) (1,1)};
        \addplot[only marks, mark=*] coordinates {(1,1)} node[anchor=north] {$x_1$};
        \addplot[only marks, mark=*] coordinates {(4,1)} node[anchor=north] {$x_2$};
        \addplot[only marks, mark=*] coordinates {(2,3)} node[anchor=south] {$x_3$};
    \end{axis}
    \end{tikzpicture}
\end{minipage}
\par\medskip

Dai grafici si nota come una combinazione convessa descrive una parte di spazio, che si può definire:
\begin{definition}{Involucro convesso}
	L'involucro convesso $\mathrm{conv}(K)$ di un'insieme di punti $K = \{x_1, x_2, ..., x_n\} \in \mathbb{R}^n$ è definito come il luogo di tutte le loro combinazioni convesse.
\end{definition}

Si nota che l'involucro convesso negli esempi precedenti è effettivamente un poliedro convesso che contiene tutti i punti che lo formano.
Si può infatti dire:
\begin{theorem}{Minimalità dell'involucro convesso}
	L'insieme $\mathrm{conv}(K)$ di tutte le combinazioni convesse di $n$ punti è il più piccolo poliedro convesso che li contiene tutti.
\end{theorem}

Si noti che non è detto che a $n$ punti corrisponda uno poligono di di $n$ vertici.
Può infatti accadere che uno dei punti è già parte dell'involucro convesso. 

Le combinazioni convesse ci permettono di descrivere parte delle regioni ammissibili (poliedri) dei problemi di programmazione lineare, ma restano ancora in sospeso problemi che ammettono regioni illimitate.
Per descrivere tali regioni, si introduce un altro tipo di combinazione.

\subsubsection{Combinazioni coniche}
\begin{definition}{Combinazione conica}
	Dati $ x_1, x_2, ..., x_k \in \mathbb{R}^n $ punti, $y$ si dice \textbf{combinazione conica} di $ x_1, x_2, ..., x_k $ se:
	$$
	\exists \lambda_i \geq 0 \quad (i = 1, ..., k) \quad \text{t.c.} \quad y = \sum_{i=1}^k \lambda_i x_i 
	$$
\end{definition}

La combinazione conica di più punti non è più il poliedro convesso che li contiene, ma il cono con vertice nell'origine, convesso o meno, che li contiene, definito come:
\begin{definition}{Involucro conico}
	L'involucro conico $\mathrm{cono}(K)$ di un'insieme di punti $K = \{x_1, x_2, ..., x_n\} \in \mathbb{R}^n$ è definito come il luogo di tutte le loro combinazioni coniche.
\end{definition}

Questo cono si estende fino all'infinito ($\lambda_i \geq 0$) nelle direzioni dei vettori che lo formano.
Il concetto è simile a quello di spazio somma, ma con la differenza che non si va ovunque nello span dei due vettori, ma si seguono le semirette che essi conducono.

\subsection{Poliedri}
Abbiamo definito un poliedro come la regione definita da un sistema di disequazioni lineari, o geometricamente come l'intersezione di un numero finito di semipiani chiusi in $\mathbb{R}^n$.
Un poliedro che è anche cono si chiama cono poliedrico. Si dimostra che:
\begin{theorem}{Cono poliedrico}
	Se $P$ è un cono poliedrico allora:
	$$
	\exists Q \quad \text{t.c.} \quad P = \{ x \in \mathbb{R}^n : Qx \leq 0 \}
	$$
	con $Q$ matrice.
\end{theorem}

Senza dimostrazioni, questo è chiaro dal fatto che le disequazioni che compongono il poliedrico sono omogenee (hanno frontiere che passano dall'origine).
Si definiscono poi i \textbf{vertici} del poliedro:
\begin{definition}{Vertice}
	Un vertice di un poliedro è un punto che non si può esprimere come combinazione convessa propria di altri punti del poliedro.
	Si indica l'insieme dei vettori di un poliedro $P$ come $\mathrm{vert}(P)$.
\end{definition}
Notiamo che i vertici di un poliedro limitato corrispondono ai punti che formano la combinazione convessa equivalente al poliedro.
Per poliedri illimitati, introduciamo invece:
\begin{definition}{Direzione di recessione}
	Un vettore $d$ è la direzione di recessione di un poliedro se:
	$$
	x + \lambda d \in P \quad \forall x \in P, \quad \forall \lambda \geq 0
	$$
	Si indica come $\mathrm{rec}(P)$ l'insieme delle direzioni di recessione di un poliedro.
\end{definition}

Chiaramente, per ogni poliedro $P$, $0 \in \mathrm{rec}(P)$ e per i poliedri limitati, $\mathrm{rec}(P) = \{0\}$.
Notiamo che le direzioni di recessione determinano i vettori del cono che coincide (almeno a distanze abbastanza grandi dall'origine) con i poliedri illimitati.
Più propriamente, si può dire che un cono poliedrico è l'involucro conico di un insieme finito dei suoi punti (basta prendere gli "estremi").

Definiamo poi lo spazio di linealità:
\begin{definition}{Spazio di linealità}
	Lo spazio di linealità di un poliedro illimtato $P$ è il più piccolo sottospazio contenuto interamente in $P$.
\end{definition}
La base di uno spazio di linealità è un vettore $d$ tale che:
$$
d \in \mathrm{rec}(P), \quad - d \in \mathrm{rec}(P)
$$
ovvero un vettore che è contenuto sia positivo che negativo nelle direzioni di recessione del poliedro.

Questa distinzione è importante in quanto non si può dimostrare completamente il prossimo problemi su poliedri con spazio di linealità $\neq {0}$.

\subsubsection{Teorema di rappresentazione dei poliedri}
Gli strumenti che abbiamo stabilito finora ci permettono di dimostrare un'importante risultato, noto come \textbf{teorema di rappresentazione dei poliedri}, o teorema di Minkowski-Weyl.

\begin{theorem}{Rappresentazione dei poliedri}
	Dato un poliedro $P$ definito come $P = \{ x \in \mathbb{R}^n : Ax \leq b \}$, si ha:
	$$
	\exists V = \{ v_1, ..., v_k \} \in \mathrm{vert}(P), \quad \exists E = {e_1, ..., e_p} \in \mathrm{rec}(P) \quad \text{t.c.} \quad P = \mathrm{conv}(V) + \mathrm{cono}(E)
	$$
\end{theorem}

Questo significa che è possibile rappresentare qualsiasi poliedro attraverso i suoi vertici, e le direzioni in cui si estende all'infinito (ergo le sue direzioni di recessione).

La somma in $P = \mathrm{conv}(V) + \mathrm{cono}(E)$ si riferisce alla somma vettoriale fra tutti i possibili punti di $\mathrm{conv}(V)$ e $\mathrm{conv}(E)$, come quella studiata sui sottospazi vettoriali (anche se nessuno dei due insiemi è un sottospazio vettoriale).
Per un dato insieme $\mathrm{conv}(V)$, quindi, l'aggiunta di $\mathrm{cono}(E)$ rappresenta la "proiezione" di tale insieme nelle direzioni di recessione indicate dal cono.

Più propriamente, posto $\mathrm{lineal}(P) = {0}$, si ha:

\begin{theorem}{Rappresentazione dei poliedri non lineali}
	Dato un poliedro $P$ definito come $P = \{ x \in \mathbb{R}^n : Ax \leq b \}$, tale che $\mathrm{lineal}(P)$, si ha:
	$$
	P = \mathrm{conv}(\mathrm{vert}(P)) + \mathrm{rec}(P)
	$$
\end{theorem}
la limitazione di linealità è necessaria in quanto un poliedro lineale potrebbe non essere rappresentato, nelle sue dimensioni infinite, dal semplice insieme dei suoi vettori.
In verità, è possibile dimostrare che:

\begin{theorem}{Linealità di poliedri}
	Per ogni poliedro $P$ non vuoto si ha:
	$$
		\mathrm{vert}(P) \neq \emptyset \Leftrightarrow \mathrm{lineal}(P) = {0}
	$$
\end{theorem}
ergo applicando lo scorso teorema potremmo provare a rappresentare un poliedro attraverso un'insieme di vettori vuoto.

Per i poliedri che otteniamo dai problemi di programmazione lineare, però, abbiamo i corollari:
\begin{itemize}
	\item Un poliedro limitato è l'involucro convesso dei suoi vertici;
	\item Se il poliedro ha vincoli di positività sulle sue variabili, allora non è lineale, ergo si applica il teorema di rappresentazione.
		Questo è il tipo di poliedri a cui siamo abituati.
\end{itemize}

\subsection{Teorema fondamentale della PL}
Quanto riportare finora sulla geometria dei poliedri può essere usato per dimostrare il seguente teorema:
\begin{theorem}{Teorema fondamentale della PL}
	Sia dato un poliedro $P$ rappresentato come:
	$$ 
	P = \mathrm{conv}(V) + \mathrm{cono}(E), \quad V = \{ v_1, ..., v_k \}, \quad E = \{ e_1, ..., e_p \}
	$$
	Se il problema $\mathcal{P}$ con regione ammissibile $P$ ha valore ottimo finito, allora esiste $s \in \{ 1, ..., k \}$ tale che $v_k$ è soluzione ottima di $\mathcal{P}$.
\end{theorem}

In sostanza, se un problema LP ha soluzione, essa si trova su uno dei vertici del poliedro della regione ammissibile.

\par\medskip
\noindent
\textbf{\textsf{Dimostrazione}} \\
Sia dato un problema LP $\mathcal{P}$ in forma primale standard, ergo posto come:
\[
	\begin{cases}
		\max{c^T \cdot x} \\ 
		Ax \leq b	
	\end{cases}
\]
ergo con regione ammissibile rappresentata da un poliedro $P$.

Dal teorema della rappresentazione, possiamo esprimere il poliedro come:
$$
	P = \mathrm{conv}(V) + \mathrm{cono}(E), \quad V = \{ v_1, ..., v_k \}, \quad E = \{ e_1, ..., e_p \}
$$

Combiniamo le due equazioni, ergo esprimiamo prima il punto $\bar{x}$ generico del poliedro applicando le definizioni di involucro convesso e conico:
$$
\bar{x} \in P : P = \mathrm{conv}(V) + \mathrm{cono}(E), \quad \bar{x} = \sum_{i=1}^k \lambda_i v_i + \sum_{j=i}^p \mu_j e_j
$$
ed esprimiamo quindi la funzione obiettivo come il prodotto scalare fra il vettore costo e il punto $\bar{x}$ del poliedro:
$$
c^T \cdot \bar{x} = c^T \cdot \left( \sum_{i=1}^k \lambda_i v_i + \sum_{j=i}^p \mu_j e_j \right) = \sum_{i=1}^k \lambda_i c^T v_i + \sum_{j=i}^p \mu_j c^T e_j
$$

A questo punto conviene chiarire su cosa significa che il problema ha valore ottimo finito.
Il secondo termine è la combinazione conica della sommatoria dei vettori di recessione scalati dal vettore costo.
Se almeno uno dei $c^T e_j > 0$, si avrà che portando $\mu_j \rightarrow +\infty$ la funzione avrà massimo $= \infty$.
Geometricamente, questo significa che esiste una direzione illimitata del poliedro dove i vettori costo permettono alla funzione di crescere all'infinito.

Dunque sarà vero che $c^T e_j \leq 0 \quad \forall j \in \{ 1, ..., p \}$ se vogliamo che la funzione abbia valore ottimo finito.

Possiamo quindi usare questa ipotesi per dire:
$$
c^T \cdot \bar{x} = \sum_{i=1}^k \lambda_i c^T v_i + \sum_{j=i}^p \mu_j c^T e_j \leq \sum_{i=1}^k \lambda_i c^T v_i \leq \sum_{i=1}^k \max_{1 \leq i \leq p} \left(\lambda_i c^T v_i\right) 
$$

$$
= \left( \max_{1 \leq i \leq p} c^T v_i\right) \sum_{i=1}^k \lambda_i = \max_{1\leq i \leq p} c^Tv_i = c^T v_k
$$

E quindi $\max_{x \in P} c^T \cdot x \leq c^T v_k$.
A questo punto, visto che $\bar{x}$ è effettivamente un punto della regione ammissibile, sarà vero che:
$$
c^Tv_k \leq \max_{x \in P} c^T \cdot x
$$

E dunque $\max_{x \in P} c^T \cdot x = c^T v_k$, C.V.D.

\end{document}


\documentclass[a4paper,11pt]{article}
\usepackage[a4paper, margin=8em]{geometry}

% usa i pacchetti per la scrittura in italiano
\usepackage[french,italian]{babel}
\usepackage[T1]{fontenc}
\usepackage[utf8]{inputenc}
\frenchspacing 

% usa i pacchetti per la formattazione matematica
\usepackage{amsmath, amssymb, amsthm, amsfonts}

% usa altri pacchetti
\usepackage{gensymb}
\usepackage{hyperref}
\usepackage{standalone}

% imposta il titolo
\title{Appunti Intelligenza Artificiale}
\author{Luca Seggiani}
\date{2024}

% disegni
\usepackage{pgfplots}
\pgfplotsset{width=10cm,compat=1.9}

% imposta lo stile
% usa helvetica
\usepackage[scaled]{helvet}
% usa palatino
\usepackage{palatino}
% usa un font monospazio guardabile
\usepackage{lmodern}

\renewcommand{\rmdefault}{ppl}
\renewcommand{\sfdefault}{phv}
\renewcommand{\ttdefault}{lmtt}

% disponi il titolo
\makeatletter
\renewcommand{\maketitle} {
	\begin{center} 
		\begin{minipage}[t]{.8\textwidth}
			\textsf{\huge\bfseries \@title} 
		\end{minipage}%
		\begin{minipage}[t]{.2\textwidth}
			\raggedleft \vspace{-1.65em}
			\textsf{\small \@author} \vfill
			\textsf{\small \@date}
		\end{minipage}
		\par
	\end{center}

	\thispagestyle{empty}
	\pagestyle{fancy}
}
\makeatother

% disponi teoremi
\usepackage{tcolorbox}
\newtcolorbox[auto counter, number within=section]{theorem}[2][]{%
	colback=blue!10, 
	colframe=blue!40!black, 
	sharp corners=northwest,
	fonttitle=\sffamily\bfseries, 
	title=Teorema~\thetcbcounter: #2, 
	#1
}

% disponi definizioni
\newtcolorbox[auto counter, number within=section]{definition}[2][]{%
	colback=red!10,
	colframe=red!40!black,
	sharp corners=northwest,
	fonttitle=\sffamily\bfseries,
	title=Definizione~\thetcbcounter: #2,
	#1
}

% disponi problemi
\newtcolorbox[auto counter, number within=section]{problem}[2][]{%
	colback=green!10,
	colframe=green!40!black,
	sharp corners=northwest,
	fonttitle=\sffamily\bfseries,
	title=Problema~\thetcbcounter: #2,
	#1
}

% disponi codice
\usepackage{listings}
\usepackage[table]{xcolor}

\lstdefinestyle{codestyle}{
		backgroundcolor=\color{black!5}, 
		commentstyle=\color{codegreen},
		keywordstyle=\bfseries\color{magenta},
		numberstyle=\sffamily\tiny\color{black!60},
		stringstyle=\color{green!50!black},
		basicstyle=\ttfamily\footnotesize,
		breakatwhitespace=false,         
		breaklines=true,                 
		captionpos=b,                    
		keepspaces=true,                 
		numbers=left,                    
		numbersep=5pt,                  
		showspaces=false,                
		showstringspaces=false,
		showtabs=false,                  
		tabsize=2
}

\lstdefinestyle{shellstyle}{
		backgroundcolor=\color{black!5}, 
		basicstyle=\ttfamily\footnotesize\color{black}, 
		commentstyle=\color{black}, 
		keywordstyle=\color{black},
		numberstyle=\color{black!5},
		stringstyle=\color{black}, 
		showspaces=false,
		showstringspaces=false, 
		showtabs=false, 
		tabsize=2, 
		numbers=none, 
		breaklines=true
}

\lstdefinelanguage{javascript}{
	keywords={typeof, new, true, false, catch, function, return, null, catch, switch, var, if, in, while, do, else, case, break},
	keywordstyle=\color{blue}\bfseries,
	ndkeywords={class, export, boolean, throw, implements, import, this},
	ndkeywordstyle=\color{darkgray}\bfseries,
	identifierstyle=\color{black},
	sensitive=false,
	comment=[l]{//},
	morecomment=[s]{/*}{*/},
	commentstyle=\color{purple}\ttfamily,
	stringstyle=\color{red}\ttfamily,
	morestring=[b]',
	morestring=[b]"
}

% disponi sezioni
\usepackage{titlesec}

\titleformat{\section}
	{\sffamily\Large\bfseries} 
	{\thesection}{1em}{} 
\titleformat{\subsection}
	{\sffamily\large\bfseries}   
	{\thesubsection}{1em}{} 
\titleformat{\subsubsection}
	{\sffamily\normalsize\bfseries} 
	{\thesubsubsection}{1em}{}

% disponi alberi
\usepackage{forest}

\forestset{
	rectstyle/.style={
		for tree={rectangle,draw,font=\large\sffamily}
	},
	roundstyle/.style={
		for tree={circle,draw,font=\large}
	}
}

% disponi algoritmi
\usepackage{algorithm}
\usepackage{algorithmic}
\makeatletter
\renewcommand{\ALG@name}{Algoritmo}
\makeatother

% disponi numeri di pagina
\usepackage{fancyhdr}
\fancyhf{} 
\fancyfoot[L]{\sffamily{\thepage}}

\makeatletter
\fancyhead[L]{\raisebox{1ex}[0pt][0pt]{\sffamily{\@title \ \@date}}} 
\fancyhead[R]{\raisebox{1ex}[0pt][0pt]{\sffamily{\@author}}}
\makeatother

\begin{document}

% sezione (data)
\section{Lezione del 30-09-24}

% stili pagina
\thispagestyle{empty}
\pagestyle{fancy}

% testo
\subsection{Agenti logici}
Vediamo adesso l'implementazione di \textbf{agenti logici}, cioè agenti che si basano su una certa \textbf{rappresentazione}, detta \textbf{base di conoscenza} (KB, \textit{Knowledge Base}) per immagazzinare \textbf{proposizioni} su ciò che hanno imparato riguardo all'ambiente esterno, e fare \textbf{inferenze}, sulla base di queste proposizioni, rispetto a informazioni non conosciute. Le proposizioni sono legate ad aspetti reali dell'ambiente esterno mediante una determinata \textbf{semantica}.

Possiamo schematizzare il funzionamento della KB, e la sua corrispondenza con l'ambiente esterno, come segue:

\begin{center}
	\begin{tikzpicture}
		\draw[dashed] (-1.5,0) -- (13, 0);

		\draw (2,0.5) rectangle ++(3,1);
		\node[anchor=center] at (3.5, 1) {Proposizioni};

		\draw (9,0.5) rectangle ++(3,1);
		\node[anchor=center] at (10.5, 1) {Proposizione};

		\draw (2,-0.5) rectangle ++(3,-1);
		\node[anchor=center] at (3.5, -1) {Aspetti reali};

		\draw (9,-0.5) rectangle ++(3,-1);
		\node[anchor=center] at (10.5, -1) {Aspetto reale};

		\draw[->] (5, 1) -- node[above, pos=0.5] {\textit{Inferenza}} (9, 1);
		\draw[->] (5, -1) -- node[above, pos=0.5] {\textit{Conseguenza}} (9, -1);

		\draw[->] (3.5,0.5) -- node[right, pos=0.3] {\textit{Semantica}} (3.5,-0.5);
		\draw[->] (10.5,0.5) -- node[right, pos=0.3] {\textit{Semantica}} (10.5,-0.5);

		\node[anchor=center] at (0, 0.5) {\textit{Rappresentazione}};
		\node[anchor=center] at (0, -0.5) {\textit{Ambiente esterno}};
	\end{tikzpicture}
\end{center}

La conoscenza che l'agente ottiene può essere fornita manualmente, cioè in forma di \textbf{assiomi} o conoscenza di \textit{background} (magari in un agente potrebbe essere "precaricata" della conoscenza riguardo allo stato iniziale del problema), estratta dai dati dei sensori, o ricavata dall'esperienza.
Un agente logico dovrebbe essere in grado di fare inferenze in quanto spesso le informazioni che ha sull'ambiente esterno  è \textbf{parziale} o \textbf{incompleta}.

Una KB deve poi rappresentare questa conoscenza attraverso un linguaggio, che dovrebbe essere sufficientemente \textbf{espressivo} da poter rappresentare la realtà dell'ambiente esterno, ma non troppo complesso da impedire di effettuare inferenze in modo efficiente. 

Esistono due approcci all'implementazione di una KB:
\begin{itemize}
	\item \textbf{Dichiarativo:} si concentra su una rappresentazione a \textbf{livello di conoscenza} dei fatti, cioè informazioni riguardo a \textit{cosa} è vero. In sistemi di questo tipo, si usano primitive di scrittura (\textsc{Tell}) e query (\textsc{Ask}) sulla KB, partendo da un insieme di informazioni nullo o comunque limitato, fino ad arrivare ad avere una serie di conoscenze comprensive dell'ambiente esterno. I dettagli delle operazioni che poi l'agente andrà a svolgere, il cosiddetto \textbf{livello di implementazione}, sono mantenuti separati dalla KB.
	\item \textbf{Procedurale:} si concentra su una rappresentazione di \textit{come} effettuare operazioni. Invece di implementare sistemi che possano immagazzinare proposizioni sull'ambiente esterno, si va a codificare l'informazione direttamente nel codice, attraverso procedure, algoritmi, o come avevamo visto nei modelli a riflesso, regole "if-then".
\end{itemize}

\subsubsection{Basi di conoscenza}
Come abbiamo detto, una base di conoscenza è formata da una serie di formule (formule \textit{atomiche}, cioè proposizioni) contenenti informazioni riguardo all'ambiente esterno e codificate in un certo \textbf{linguaggio formale}.
Si possono definire alcune primitive per l'interazione con la KB:
\begin{itemize}
	\item \textsc{\textbf{Tell}}: aggiungi una nuova proposizione alla KB;
	\item \textsc{\textbf{Ask}}: richiedi informazioni dalla KB;
	\item \textsc{\textbf{Retract}}: elimina informazioni dalla KB.
\end{itemize}

La KB si basa sui fatti che già conosce per ricavare inferenze o \textbf{deduzioni logiche} $\alpha$, della forma:
$$ \text{KB} \models \alpha $$

L'agente logico si interfaccia con la KB attraverso le primitive sopra definita, e implementa effettivamente un ciclo \textsc{Tell}-\textsc{Ask}-\textsc{Tell} del tipo:

\begin{algorithm}
\caption{Agente logico}
\begin{algorithmic}
	\STATE \textbf{Input:} le percezioni correnti % input
	\STATE \textbf{Output:} la prossima azione da eseguire % output
	% body
	\STATE \textsc{Tell}(percezioni correnti)
	\STATE prossima azione $\leftarrow$ \textsc{ASK}(KB)
	\STATE \textsc{TELL}(prossima azione)
	\RETURN prossima azione
\end{algorithmic}
\end{algorithm}

Ovvero, l'agente invia le sue percezioni correnti alla KB, e richiede la prossima azione da eseguire.
Invia poi l'azione scelta alla KB (così che possa diventare parte delle informazioni note), e la restituisce.

\subsubsection{Differenza fra KB e DB}
Una KB potrebbe sembrare simile ad un comune database: la differenza è che il database si occupa solo di ricavare fatti specifici, senza possibilità di deduzione di alcun tipo.
La KB è invece progettata per mantenere una rappresentazione strutturata dei fatti, specifici o generali, come riferiti a oggetti reali, e permettere quindi inferenze su quei fatti. 

\subsection{Logica proposizionale}


\end{document}


\documentclass[a4paper,11pt]{article}
\usepackage[a4paper, margin=8em]{geometry}

% usa i pacchetti per la scrittura in italiano
\usepackage[french,italian]{babel}
\usepackage[T1]{fontenc}
\usepackage[utf8]{inputenc}
\frenchspacing 

% usa i pacchetti per la formattazione matematica
\usepackage{amsmath, amssymb, amsthm, amsfonts}

% usa altri pacchetti
\usepackage{gensymb}
\usepackage{hyperref}
\usepackage{standalone}

% imposta il titolo
\title{Appunti Ricerca Operativa}
\author{Luca Seggiani}
\date{2024}

% disegni
\usepackage{pgfplots}
\pgfplotsset{width=10cm,compat=1.9}

% imposta lo stile
% usa helvetica
\usepackage[scaled]{helvet}
% usa palatino
\usepackage{palatino}
% usa un font monospazio guardabile
\usepackage{lmodern}

\renewcommand{\rmdefault}{ppl}
\renewcommand{\sfdefault}{phv}
\renewcommand{\ttdefault}{lmtt}

% disponi il titolo
\makeatletter
\renewcommand{\maketitle} {
	\begin{center} 
		\begin{minipage}[t]{.8\textwidth}
			\textsf{\huge\bfseries \@title} 
		\end{minipage}%
		\begin{minipage}[t]{.2\textwidth}
			\raggedleft \vspace{-1.65em}
			\textsf{\small \@author} \vfill
			\textsf{\small \@date}
		\end{minipage}
		\par
	\end{center}

	\thispagestyle{empty}
	\pagestyle{fancy}
}
\makeatother

% disponi teoremi
\usepackage{tcolorbox}
\newtcolorbox[auto counter, number within=section]{theorem}[2][]{%
	colback=blue!10, 
	colframe=blue!40!black, 
	sharp corners=northwest,
	fonttitle=\sffamily\bfseries, 
	title=Teorema~\thetcbcounter: #2, 
	#1
}

% disponi definizioni
\newtcolorbox[auto counter, number within=section]{definition}[2][]{%
	colback=red!10,
	colframe=red!40!black,
	sharp corners=northwest,
	fonttitle=\sffamily\bfseries,
	title=Definizione~\thetcbcounter: #2,
	#1
}

% disponi problemi
\newtcolorbox[auto counter, number within=section]{problem}[2][]{%
	colback=green!10,
	colframe=green!40!black,
	sharp corners=northwest,
	fonttitle=\sffamily\bfseries,
	title=Problema~\thetcbcounter: #2,
	#1
}

% disponi codice
\usepackage{listings}
\usepackage[table]{xcolor}

\lstdefinestyle{codestyle}{
		backgroundcolor=\color{black!5}, 
		commentstyle=\color{codegreen},
		keywordstyle=\bfseries\color{magenta},
		numberstyle=\sffamily\tiny\color{black!60},
		stringstyle=\color{green!50!black},
		basicstyle=\ttfamily\footnotesize,
		breakatwhitespace=false,         
		breaklines=true,                 
		captionpos=b,                    
		keepspaces=true,                 
		numbers=left,                    
		numbersep=5pt,                  
		showspaces=false,                
		showstringspaces=false,
		showtabs=false,                  
		tabsize=2
}

\lstdefinestyle{shellstyle}{
		backgroundcolor=\color{black!5}, 
		basicstyle=\ttfamily\footnotesize\color{black}, 
		commentstyle=\color{black}, 
		keywordstyle=\color{black},
		numberstyle=\color{black!5},
		stringstyle=\color{black}, 
		showspaces=false,
		showstringspaces=false, 
		showtabs=false, 
		tabsize=2, 
		numbers=none, 
		breaklines=true
}

\lstdefinelanguage{javascript}{
	keywords={typeof, new, true, false, catch, function, return, null, catch, switch, var, if, in, while, do, else, case, break},
	keywordstyle=\color{blue}\bfseries,
	ndkeywords={class, export, boolean, throw, implements, import, this},
	ndkeywordstyle=\color{darkgray}\bfseries,
	identifierstyle=\color{black},
	sensitive=false,
	comment=[l]{//},
	morecomment=[s]{/*}{*/},
	commentstyle=\color{purple}\ttfamily,
	stringstyle=\color{red}\ttfamily,
	morestring=[b]',
	morestring=[b]"
}

% disponi sezioni
\usepackage{titlesec}

\titleformat{\section}
	{\sffamily\Large\bfseries} 
	{\thesection}{1em}{} 
\titleformat{\subsection}
	{\sffamily\large\bfseries}   
	{\thesubsection}{1em}{} 
\titleformat{\subsubsection}
	{\sffamily\normalsize\bfseries} 
	{\thesubsubsection}{1em}{}

% disponi alberi
\usepackage{forest}

\forestset{
	rectstyle/.style={
		for tree={rectangle,draw,font=\large\sffamily}
	},
	roundstyle/.style={
		for tree={circle,draw,font=\large}
	}
}

% disponi algoritmi
\usepackage{algorithm}
\usepackage{algorithmic}
\makeatletter
\renewcommand{\ALG@name}{Algoritmo}
\makeatother

% disponi numeri di pagina
\usepackage{fancyhdr}
\fancyhf{} 
\fancyfoot[L]{\sffamily{\thepage}}

\makeatletter
\fancyhead[L]{\raisebox{1ex}[0pt][0pt]{\sffamily{\@title \ \@date}}} 
\fancyhead[R]{\raisebox{1ex}[0pt][0pt]{\sffamily{\@author}}}
\makeatother

\begin{document}

% sezione (data)
\section{Lezione del 01-10-24}

% stili pagina
\thispagestyle{empty}
\pagestyle{fancy}

% testo
\subsection{Soluzioni di base primali degeneri}
Abbiamo dato un teorema di caratterizzazione dei vertici primali.
Questo teorema si basava sulla nozione di \textbf{soluzione di base primale}.
Possiamo fare una distinzione fra soluzione di base degeneri e non degeneri:
\begin{definition}{Soluzione di base degenere}
Quando una soluzione di base è soluzione di più combinazioni delle disequazioni del problema, essa si dice degenere.
\end{definition}

Questa definizione è esatta ma non particolarmente utile.
Sostanzialmente, ci dice soltanto che una soluzione degenere è \textbf{ridondante} su più combinazioni di disequazioni (cioè risolve $A_Bx = b_B$ su più permutazioni degli $1,...,m$ elementi in classi $n$ in $B$).
Si noti che ridondante non significa \textbf{eliminabile}: questa affermazione purtroppo è vera soltanto in $R^2$, dove effettivamente si può rimuovere una delle disequazioni ridondanti ed avere sempre lo stesso risultato.

Diamo quindi una caratterizzazione delle soluzioni di base primali degeneri appoggiandoci al teorema di caratterizzazione dei vertici, ergo al concetto di soluzione di base:
\begin{theorem}{Caratterizzazione delle soluzioni di base primali degeneri}
	Se una soluzione è di base, ergo scelto $B = \{ 1, ..., m \}$ con $\mathrm{card}(B) = n$ è data da $A_Bx = b_B$, possiamo dire che è pure degenere quando $\exists i \in N$ t.c. $A_i x = b_i$, con $I$ = $\{1, ..., m\} - B$. 
\end{theorem}
Quindi, una soluzione di base è degenere quando almeno una variabile di base si annulla per almeno una delle disequazioni non di base indicate dagli indici $I$, che sono tutti gli indici fra $\{1,...,m\}$ non contenuti in $B$. 

Sulla stessa linea di pensiero, possiamo dimostrare un'altro teorema, stavolta sul concetto piuttosto intuitivo di ammissibilità.
Potremmo infatti dire che una soluzione di base ammissibile, cioè che rientra all'interno della regione ammissibile, è tale se:
\begin{theorem}{Caratterizzazione delle soluzioni di base primali ammissibili}
	Se una soluzione è di base, ergo scelto $B = \{ 1, ..., m \}$ con $\mathrm{card}(B) = n$ è data da $A_Bx = b_B$, possiamo dire che è ammissible quando $\forall i \in N$ si ha $A_i x \leq b_i$, con $I$ = $\{1, ..., m\} - B$. 
\end{theorem}
cioè banalmente rispetta tutte le disequazioni.

\subsubsection{Considerazioni numeriche sui numeri di soluzioni base}
Solitamente un problema con $n$ variabili decisionali a $m \geq n$ vincoli.
Posti questi vincoli, visto che per calcolare $\mathrm{vert}(P)$ prendiamo effettivamente tutte le combinazioni degli $m$ vincoli classe $n$ variabili decisionali, possiamo usare il coefficiente binomiale per calcolare il numero massimo di potenziali vertici: 
$$ \mathrm{card}(\mathrm{vert}(P)) \sim \binom{m}{n} = \frac{m!}{n!(m-n)!} $$

In verità, i vertici sono solitamente meno, in quanto possiamo rimuovere le soluzioni non ammissibili.
Inoltre, le soluzioni degeneri non contribuiscono al risultato, ergo anche quelle non sono rilevanti.

\subsection{Riassunto delle trasformazioni equivalenti}
Riassumiamo adesso le trasformazioni equivalenti che abbiamo individuato finora per le disequazioni di problemi LP:
\begin{itemize}
	\item $\mathrm{min}(C^T \cdot x) \leftrightarrow \mathrm{max}(C^T \cdot x)$: trasformiamo problemi di massimo in problemi di minimo invertendo i segni;
	\item $Ax \geq b \leftrightarrow -Ax \leq -b$: invertiamo il verso della diseguaglianza moltiplicando per $-1$;
	\item $Ax = b \rightarrow Ax \leq b \wedge Ax \geq b$: convertiamo un'uguaglianza in una coppia di diseguaglianze;
	\item $Ax \leq b \rightarrow Ax + s = b$: convertiamo una diseguaglianza in un'uguaglianza introducendo una variabile di surplus. Si nota che la variabile di surplus può essere riconosciuta per essere rimossa, da:
		\begin{itemize}
			\item $s > 0$;
			\item Compare in un solo vincolo, che è di uguaglianza;
			\item Ha coefficiente $0$ nella funzione obiettivo, e $1$ nell'equazione dove compare;
		\end{itemize}
	\item $ x \mathrel{\text{\tikz[baseline]{
    \draw (0,1.1ex)--(1.1em,0.1ex);
    \node[scale=1] at (0.6em,0.3em) {$\geq$};
  }}} 0 \rightarrow x = x^+ - x^-, \quad x^+ \geq 0, \quad x^- \geq 0$: aggiriamo il vincolo di positività introducendo parti positive e negative delle variabili decisionali, con rispettivi vincoli di positività.
\end{itemize}

Usiamo queste trasformazioni per portare i problemi LP in forme standard.
Esistono molteplici forme standard, ma in questo corso ci riguardano: il formato \textit{linprog}, usato dal pacchetto software \textit{MATLAB}, le forme standard primale (già vista) e duale (che vedremo fra poco). 

\end{document}


\documentclass[a4paper,11pt]{article}
\usepackage[a4paper, margin=8em]{geometry}

% usa i pacchetti per la scrittura in italiano
\usepackage[french,italian]{babel}
\usepackage[T1]{fontenc}
\usepackage[utf8]{inputenc}
\frenchspacing 

% usa i pacchetti per la formattazione matematica
\usepackage{amsmath, amssymb, amsthm, amsfonts}

% usa altri pacchetti
\usepackage{gensymb}
\usepackage{hyperref}
\usepackage{standalone}

% imposta il titolo
\title{Appunti Ricerca Operativa}
\author{Luca Seggiani}
\date{2024}

% disegni
\usepackage{pgfplots}
\pgfplotsset{width=10cm,compat=1.9}

% imposta lo stile
% usa helvetica
\usepackage[scaled]{helvet}
% usa palatino
\usepackage{palatino}
% usa un font monospazio guardabile
\usepackage{lmodern}

\renewcommand{\rmdefault}{ppl}
\renewcommand{\sfdefault}{phv}
\renewcommand{\ttdefault}{lmtt}

% disponi il titolo
\makeatletter
\renewcommand{\maketitle} {
	\begin{center} 
		\begin{minipage}[t]{.8\textwidth}
			\textsf{\huge\bfseries \@title} 
		\end{minipage}%
		\begin{minipage}[t]{.2\textwidth}
			\raggedleft \vspace{-1.65em}
			\textsf{\small \@author} \vfill
			\textsf{\small \@date}
		\end{minipage}
		\par
	\end{center}

	\thispagestyle{empty}
	\pagestyle{fancy}
}
\makeatother

% disponi teoremi
\usepackage{tcolorbox}
\newtcolorbox[auto counter, number within=section]{theorem}[2][]{%
	colback=blue!10, 
	colframe=blue!40!black, 
	sharp corners=northwest,
	fonttitle=\sffamily\bfseries, 
	title=Teorema~\thetcbcounter: #2, 
	#1
}

% disponi definizioni
\newtcolorbox[auto counter, number within=section]{definition}[2][]{%
	colback=red!10,
	colframe=red!40!black,
	sharp corners=northwest,
	fonttitle=\sffamily\bfseries,
	title=Definizione~\thetcbcounter: #2,
	#1
}

% disponi problemi
\newtcolorbox[auto counter, number within=section]{problem}[2][]{%
	colback=green!10,
	colframe=green!40!black,
	sharp corners=northwest,
	fonttitle=\sffamily\bfseries,
	title=Problema~\thetcbcounter: #2,
	#1
}

% disponi codice
\usepackage{listings}
\usepackage[table]{xcolor}

\lstdefinestyle{codestyle}{
		backgroundcolor=\color{black!5}, 
		commentstyle=\color{codegreen},
		keywordstyle=\bfseries\color{magenta},
		numberstyle=\sffamily\tiny\color{black!60},
		stringstyle=\color{green!50!black},
		basicstyle=\ttfamily\footnotesize,
		breakatwhitespace=false,         
		breaklines=true,                 
		captionpos=b,                    
		keepspaces=true,                 
		numbers=left,                    
		numbersep=5pt,                  
		showspaces=false,                
		showstringspaces=false,
		showtabs=false,                  
		tabsize=2
}

\lstdefinestyle{shellstyle}{
		backgroundcolor=\color{black!5}, 
		basicstyle=\ttfamily\footnotesize\color{black}, 
		commentstyle=\color{black}, 
		keywordstyle=\color{black},
		numberstyle=\color{black!5},
		stringstyle=\color{black}, 
		showspaces=false,
		showstringspaces=false, 
		showtabs=false, 
		tabsize=2, 
		numbers=none, 
		breaklines=true
}

\lstdefinelanguage{javascript}{
	keywords={typeof, new, true, false, catch, function, return, null, catch, switch, var, if, in, while, do, else, case, break},
	keywordstyle=\color{blue}\bfseries,
	ndkeywords={class, export, boolean, throw, implements, import, this},
	ndkeywordstyle=\color{darkgray}\bfseries,
	identifierstyle=\color{black},
	sensitive=false,
	comment=[l]{//},
	morecomment=[s]{/*}{*/},
	commentstyle=\color{purple}\ttfamily,
	stringstyle=\color{red}\ttfamily,
	morestring=[b]',
	morestring=[b]"
}

% disponi sezioni
\usepackage{titlesec}

\titleformat{\section}
	{\sffamily\Large\bfseries} 
	{\thesection}{1em}{} 
\titleformat{\subsection}
	{\sffamily\large\bfseries}   
	{\thesubsection}{1em}{} 
\titleformat{\subsubsection}
	{\sffamily\normalsize\bfseries} 
	{\thesubsubsection}{1em}{}

% disponi alberi
\usepackage{forest}

\forestset{
	rectstyle/.style={
		for tree={rectangle,draw,font=\large\sffamily}
	},
	roundstyle/.style={
		for tree={circle,draw,font=\large}
	}
}

% disponi algoritmi
\usepackage{algorithm}
\usepackage{algorithmic}
\makeatletter
\renewcommand{\ALG@name}{Algoritmo}
\makeatother

% disponi numeri di pagina
\usepackage{fancyhdr}
\fancyhf{} 
\fancyfoot[L]{\sffamily{\thepage}}

\makeatletter
\fancyhead[L]{\raisebox{1ex}[0pt][0pt]{\sffamily{\@title \ \@date}}} 
\fancyhead[R]{\raisebox{1ex}[0pt][0pt]{\sffamily{\@author}}}
\makeatother

\begin{document}

% sezione (data)
\section{Lezione del 02-10-24}

% stili pagina
\thispagestyle{empty}
\pagestyle{fancy}

% testo
\subsection{Trasporto}
Poniamo il seguente problema:

\begin{problem}{Trasporto}
	Due centrali del latte di Firenze producono	rispettivamente 50 e 60 mila litri di latte al giorno.
	Le centrali servono tre quartieri, che consumano rispettivamente 30, 30 e 20 mila litri di latte al giorno.
	Si conosce il costo necessario per portare un migliaio di litri di latte da ogni centrale a ogni quartiere, riportato nella seguente tabella:
	
	\center{} \rowcolors{2}{green!10}{green!40!black!20}
	\begin{tabular} { | c || c | c | c | }
		\hline
		& \bfseries Novoli & \bfseries Statuto & \bfseries Rifredi \\ 
		\hline
		\bfseries Centrale A & 6 & 8 & 4 \\
		\bfseries Centrale B & 7 & 3 & 9 \\
		\hline
	\end{tabular}

	\par\bigskip

	Vogliamo capire quanto latte deve spedire ogni centrale ad ogni quartiere.

	\raggedright
	\par\smallskip

	\tiny{Nota simpatica: secondo l'indagine INRAN-SCAI 2005-06, l'italiano medio consuma $0.115 \mathrm{g}$ di latte al giorno, che per un peso specifico di circa $1.040 \mathrm{kg}/\mathrm{L}$ fanno $0.11 \mathrm{L}$ di latte al giorno. Al 2024, il comune di Firenze ha $364\ 073$ abitanti, ergo dovrebbe avere bisogno di approssimativamente $40\ 258 \mathrm{L}$ di latte al giorno. I fiorentini nell'esempio devono avere le ossa veramente forti!}

\end{problem}

Possiamo esprimere il problema dell'esempio come un problema LP.
Abbiamo innanzitutto che i costi di trasporto formano una matrice:
$$
C_{matr} =
\begin{pmatrix}
6 & 8 & 4 \\
7 & 3 & 9 \\
\end{pmatrix}
$$
che possiamo linearizzare, come avevamo fatto nei problemi di assegnamento di costo minimo, in un vettore costo:
$$
C = (6, 8, 4, 7, 3, 9)
$$

Questo vettore moltiplica il vettore delle variabili decisionali, che è la linearizzazione della matrice:
$$
x_{matr} =
\begin{pmatrix}
	x_{13} & x_{14} & x_{15} \\ 
	x_{23} & x_{24} & x_{25}
\end{pmatrix}
$$

Questa matrice non rappresenta altro che quanto latte mandare ad ogni quartiere.

A questo punto, possiamo stabilire i vincoli.
Innanzitutto, non si può avere più latte di quanto viene prodotto, ergo:
\[
	\begin{cases}
		x_{13} + x_{14} + x_{15} \leq 50 \\ 
		x_{23} + x_{14} + x_{15} \leq 60
	\end{cases}
\]
inoltre, si vuole fornire ad ogni quartiere il fabbisogno richiesto, ergo:
\[
	\begin{cases}
		x_13 + x_{23} \geq 30 \\	
		x_14 + x_{24} \geq 30	\\
		x_15 + x_{25} \geq 20	
	\end{cases}
\]

Questo è un problema di programmazione lineare.

In generale, quindi, un problema di trasporto minimizza la funzione obiettivo data da una matrice di costo in $n \times m$ variabili, con $m$ vincoli di riga sul vettore $o_j$ dei limiti di produzione, e $n$ vincoli di colonna sul vettore $d_j$ della domanda, in forma:
\[
	\begin{cases}
		\min{\sum^m_{i=1} \sum^n_{j=1} c_{ij}x_{ij}} = C^T \cdot x \\
		\sum^m_{i=1} x_{ij} \geq d_{j} \quad \forall j = 1,...,n \\ 
		\sum^n_{j=1} x_{ij} \leq o_{i} \quad \forall i = 1,...,m \\ 
		x \geq 0
	\end{cases}
\]

Non ci sono soluzioni se la domanda supera l'offerta, cioè sé:
$$
\sum_{j=1}^m d_j \geq \sum_{i=1}^m o_i
$$

Mentre in caso di eccessi di produzione, potremmo trasformare le diseguaglianze in eguaglianze, e aggiugnere un carico "fittizio" con costo zero dove deviare il surplus di produzione.

Inoltre, come avevamo detto per i problemi di assegnamento di costo minimo, anche qui potremmo scegliere di distinguere fra trasporti divisibili (nel campo $x = \mathbb{R}^n$) e indivisibili (col vincolo aggiunto $x = \mathbb{Z}^n$).

\subsection{Forma duale standard}
Avevamo definito la forma primale standard:
\[
	\begin{cases}
		\max{C^T \cdot x} \\
		Ax \leq b
	\end{cases}
\]

Introduciamo adesso la forma duale standard:
\begin{definition}{Forma duale standard}
	Un problema di programmazione lineare si dice in forma duale standard quando è espresso in forma:
	
	\[
		\begin{cases}
			\min(c^T \cdot x) \\
			Ax = b \\
			x \geq 0
		\end{cases}
	\]

\end{definition}

\subsubsection{Vertici del duale}
Sulle forme duali è semplice il calcolo dei vertici. 
Possiamo infatti avere, come avevamo fatto sulla primale:
\begin{definition}{Soluzione di base duale}
	Sia dato un problema LP $\mathcal{P}$ in forma duale standard.
	Sia $B \subseteq \{ 1, ..., n \}$ un sottoinsieme di indici di variabili decisionali tale che $\mathrm{card}(B) = m$.
	Chiamiamo $x_B$ l'insieme delle variabili decisionali individuate da $B$, e $x_N$ l'insieme delle $n - m$ variabili decisionali rimanenti:
	$$ x = \{x_B, x_N\}$$
	Impostiamo quindi tutte le $x_N$ a 0: avremo un sistema di $m$ variabili in $m$ equazioni, quindi determinato.
	La soluzione di quel sistema è detta soluzione di base duale di $\mathcal{P}$.
\end{definition}

Indichiamo spesso questo vertice come $(bA_B^{-1}, 0)$.
Questa definizione porta ad una caratterizzazione dei vertici del tutto analoga a quella dichiarata sui problemi in forma primale standard:

\begin{theorem}{Caratterizzazione dei vertici duali}
	Su un problema in forma duale standard, un punto $x$ del poliedro $P$ è un vertice di $P$ se e solo se è una soluzione di base duale ammissibile, ovvero:
	$$ 
	x \in \mathrm{vert}(P) \Leftrightarrow \text{$x$ è soluzione di base duale}
	$$
\end{theorem}

\subsubsection{Soluzioni di base duali degeneri}
Possiamo ricavare il concetto di soluzione degenere (e anche di soluzione ammissibile) sui vertici del poliedro del duale. Si ha:

\begin{theorem}{Caratterizzazione delle soluzioni di base duali degeneri}
	Se una soluzione è di base, ergo scelto $B = \{ 1, ..., n \}$ con $\mathrm{card}(B) = m$ è data da $(bA_B^{-1}, 0)$, possiamo dire che è pure degenere quando $\exists i \in B$ tale che almeno una componente si annulla. 
\end{theorem}
e riguardo l'ammissibilità:
\begin{theorem}{Caratterizzazione delle soluzioni di base duali ammissibili}
	Se una soluzione è di base, ergo scelto $B = \{ 1, ..., n \}$ con $\mathrm{card}(B) = m$ è data da $(bA_B^{-1}, 0)$, possiamo dire che è ammissibile quando il vettore soluzione è $\geq 0$. 
\end{theorem}

\end{document}


\documentclass[a4paper,11pt]{article}
\usepackage[a4paper, margin=8em]{geometry}

% usa i pacchetti per la scrittura in italiano
\usepackage[french,italian]{babel}
\usepackage[T1]{fontenc}
\usepackage[utf8]{inputenc}
\frenchspacing 

% usa i pacchetti per la formattazione matematica
\usepackage{amsmath, amssymb, amsthm, amsfonts}

% usa altri pacchetti
\usepackage{gensymb}
\usepackage{hyperref}
\usepackage{standalone}

% imposta il titolo
\title{Appunti Ricerca Operativa}
\author{Luca Seggiani}
\date{2024}

% disegni
\usepackage{pgfplots}
\pgfplotsset{width=10cm,compat=1.9}

% imposta lo stile
% usa helvetica
\usepackage[scaled]{helvet}
% usa palatino
\usepackage{palatino}
% usa un font monospazio guardabile
\usepackage{lmodern}

\renewcommand{\rmdefault}{ppl}
\renewcommand{\sfdefault}{phv}
\renewcommand{\ttdefault}{lmtt}

% disponi il titolo
\makeatletter
\renewcommand{\maketitle} {
	\begin{center} 
		\begin{minipage}[t]{.8\textwidth}
			\textsf{\huge\bfseries \@title} 
		\end{minipage}%
		\begin{minipage}[t]{.2\textwidth}
			\raggedleft \vspace{-1.65em}
			\textsf{\small \@author} \vfill
			\textsf{\small \@date}
		\end{minipage}
		\par
	\end{center}

	\thispagestyle{empty}
	\pagestyle{fancy}
}
\makeatother

% disponi teoremi
\usepackage{tcolorbox}
\newtcolorbox[auto counter, number within=section]{theorem}[2][]{%
	colback=blue!10, 
	colframe=blue!40!black, 
	sharp corners=northwest,
	fonttitle=\sffamily\bfseries, 
	title=Teorema~\thetcbcounter: #2, 
	#1
}

% disponi definizioni
\newtcolorbox[auto counter, number within=section]{definition}[2][]{%
	colback=red!10,
	colframe=red!40!black,
	sharp corners=northwest,
	fonttitle=\sffamily\bfseries,
	title=Definizione~\thetcbcounter: #2,
	#1
}

% disponi problemi
\newtcolorbox[auto counter, number within=section]{problem}[2][]{%
	colback=green!10,
	colframe=green!40!black,
	sharp corners=northwest,
	fonttitle=\sffamily\bfseries,
	title=Problema~\thetcbcounter: #2,
	#1
}

% disponi codice
\usepackage{listings}
\usepackage[table]{xcolor}

\lstdefinestyle{codestyle}{
		backgroundcolor=\color{black!5}, 
		commentstyle=\color{codegreen},
		keywordstyle=\bfseries\color{magenta},
		numberstyle=\sffamily\tiny\color{black!60},
		stringstyle=\color{green!50!black},
		basicstyle=\ttfamily\footnotesize,
		breakatwhitespace=false,         
		breaklines=true,                 
		captionpos=b,                    
		keepspaces=true,                 
		numbers=left,                    
		numbersep=5pt,                  
		showspaces=false,                
		showstringspaces=false,
		showtabs=false,                  
		tabsize=2
}

\lstdefinestyle{shellstyle}{
		backgroundcolor=\color{black!5}, 
		basicstyle=\ttfamily\footnotesize\color{black}, 
		commentstyle=\color{black}, 
		keywordstyle=\color{black},
		numberstyle=\color{black!5},
		stringstyle=\color{black}, 
		showspaces=false,
		showstringspaces=false, 
		showtabs=false, 
		tabsize=2, 
		numbers=none, 
		breaklines=true
}

\lstdefinelanguage{javascript}{
	keywords={typeof, new, true, false, catch, function, return, null, catch, switch, var, if, in, while, do, else, case, break},
	keywordstyle=\color{blue}\bfseries,
	ndkeywords={class, export, boolean, throw, implements, import, this},
	ndkeywordstyle=\color{darkgray}\bfseries,
	identifierstyle=\color{black},
	sensitive=false,
	comment=[l]{//},
	morecomment=[s]{/*}{*/},
	commentstyle=\color{purple}\ttfamily,
	stringstyle=\color{red}\ttfamily,
	morestring=[b]',
	morestring=[b]"
}

% disponi sezioni
\usepackage{titlesec}

\titleformat{\section}
	{\sffamily\Large\bfseries} 
	{\thesection}{1em}{} 
\titleformat{\subsection}
	{\sffamily\large\bfseries}   
	{\thesubsection}{1em}{} 
\titleformat{\subsubsection}
	{\sffamily\normalsize\bfseries} 
	{\thesubsubsection}{1em}{}

% disponi alberi
\usepackage{forest}

\forestset{
	rectstyle/.style={
		for tree={rectangle,draw,font=\large\sffamily}
	},
	roundstyle/.style={
		for tree={circle,draw,font=\large}
	}
}

% disponi algoritmi
\usepackage{algorithm}
\usepackage{algorithmic}
\makeatletter
\renewcommand{\ALG@name}{Algoritmo}
\makeatother

% disponi numeri di pagina
\usepackage{fancyhdr}
\fancyhf{} 
\fancyfoot[L]{\sffamily{\thepage}}

\makeatletter
\fancyhead[L]{\raisebox{1ex}[0pt][0pt]{\sffamily{\@title \ \@date}}} 
\fancyhead[R]{\raisebox{1ex}[0pt][0pt]{\sffamily{\@author}}}
\makeatother

\begin{document}

% sezione (data)
\section{Lezione del 03-10-24}

% stili pagina
\thispagestyle{empty}
\pagestyle{fancy}

% testo
\subsection{Teoria della dualità}
Introduciamo adesso uno dei concetti più importanti della programmazione lineare.
Avevamo posto problemi LP in forma primale standard come:

\[
	\begin{cases}
		\min(c^T \cdot x) \\
		Ax \leq b
	\end{cases}
\]

Ottimizzare questo problema significa partire dal basso e avvicinarsi verso un punto di massimo.
Potremmo scegliere di seguire il percorso opposto: cercare di estrapolare un limite superiore per la soluzione dai vincoli, e minimizzarlo.

Per fare ciò introduciamo $m$ variabili, una per ogni disequazione, che denoteremo come $y_1, ..., y_m$.
Moltiplichiamo ogni disequazione per la $y_i$ corrispondente a destra e a sinistra.
Su un semplice problema $n, m = 2$, questo darà una forma del tipo:
\[
	\begin{cases}
		a_{11} x_1 + a_{12} x_2 \leq b_1 \\
		a_{21} x_1 + a_{22} x_2 \leq b_2
	\end{cases}
	\rightarrow
	\begin{cases}
	y_1 \cdot \left(	a_{11} x_1 + a_{12} x_2 \right) \leq b_1 y_1 \\
	y_2 \cdot \left( a_{21} x_1 + a_{22} x_2 \right) \leq b_2 y_2
	\end{cases}
\]

Se vincoliamo gli $y_i$ in modo che ogni variabile decisionale $x_i$ del sistema abbia un coefficiente del costo $\geq c_i$ corrispondente, otterremo una disequazione che ha a sinistra una situazione di valore uguale o addirittura migliore di quella data dalla funzione costo, e a destra un massimo (che era ciò che stavamo cercando, un limite superiore).
Abbiamo quindi una serie di variabili vincolate:
\[
	\begin{cases}			
		y_1 a_11 + y_2 a_21 \geq c_1 \\ 
		y_1 a_12 + y_2 a_22 \geq c_2 
	\end{cases}
\]
e una funzione da minimizzare:
$$
\min(b_1 y_1 + b_2 y)
$$

Cioè, ci siamo ricondotti ad un altro problema LP.
Possiamo formalizzare questo risultato:
\begin{definition}{Duale di un problema LP}
	Per un qualsiasi problema LP $\mathcal{P}$, detto primale, con $m \geq n$, possiamo definire il duale $\mathcal{D}$:
	\[
		P:
		\begin{cases}
			\max(c^T \cdot x) \\ 
			Ax \leq b
		\end{cases}
	\rightarrow
		D:
		\begin{cases}
			\min(b^T \cdot y) \\ 
			A^T y = c \\
			y \geq 0
		\end{cases}
	\]
	dove si nota $x \in \mathbb{R}^n$ e $y \in \mathbb{R}^m$.
\end{definition}

Il duale viene posto in forma duale standard in quanto ciò che ci interessa è \textit{stringere} il limite superiore fino al suo minimo, in un modo che fa combaciare perfettamente le variabili con il loro vettore costo, da cui le uguaglianze.

Si può dimostrare che l'operazione del calcolo del duale è involutoria: il duale del duale è nuovamente il primale, e così via.

\subsubsection{Dualità debole}
Visto che abbiamo costruito il duale per avere un limite superiore dei valori ottenuti dalla funzione obiettivo del primale, potremo dimostrare facilmente:
\begin{theorem}{Dualità debole}
	Se i poliedri $P$ e il suo duale $D$ non sono vuoti, allora:
	$$ c^T x \leq y^T b \quad \forall x \in P, \forall y \in D$$
\end{theorem}
Cioè il duale è sempre maggiore del primale.

\subsubsection{Dualità forte}
Idealmente, ciò che vorremmo è che primale e duale convergessero verso un punto comune, ergo l'ottimo di entrambi.
Effettivamente, questo risultato è verificato:
\begin{theorem}{Dualità forte}
	Se i poliedri $P$ e il suo duale $D$ non sono vuoti, allora:
	$$ -\infty \leq \min_{y \in D} b^T y = \max_{x \in P} c^T x \leq +\infty $$
\end{theorem}

Il teorema della dualità forte afferma che, se entrambi i poliedri (primale e duale) sono non vuoti, allora condividono l'ottimo, e anzi, che due soluzioni nel primale e nel duale sono ottime solo se hanno lo stesso valore.
Se invece solo il primale (solo il duale) è vuoto, si ha che entrambi condividono ottimo $-\infty$ ($\infty$).
Quando entrambi sono vuoti non si ha soluzione condivisa.

\subsubsection{Scarti complementari}
Si può dimostrare il seguente teorema:
\begin{theorem}{Scarti complementari}
	Se le soluzioni $x$ e $y$ dei problemi primale e duale $\mathcal{P}$ e $\mathcal{D}$ sono entrambe ottime, allora vale:
	$$ y^T (b - Ax) = 0 $$
\end{theorem}
Questo si ricava dal fatto che, per la dualità forte, si ha che:
$$
c^T x = y^T Ax = y^T b \Rightarrow y^T(b - Ax) = 0
$$

Il significato del teorema è che, se una disequazione nel primale è \textit{stretta}, allora la corrispondente variabile nel duale è $\neq 0$, e viceversa.

\subsubsection{Soluzioni di base}
Avevamo dato una definizione di soluzione di base per problemi LP in forma sia primale che duale.
Possiamo dimostrare che non solo questa nozione esiste su entrambe le formule, ma è analoga su coppie primale / duale.

Avevamo posto che la formazione di una certa base $B \in \{ 1, ..., m \}$ per ricavare soluzioni di base.
Per il primale, questo significa partizionare la matrice e i termini noti:
$$
A = \left( A_B \over A_N \right), \quad b = \left( b_A \over b_N \right)
$$
mentre per il duale, significherà partizionare le variabili introdotte:
$$
y = \left( y_B \over y_N \right)
$$
noto il numero di $y_1, ..., y_m$ uguale a $m$.

Questo significa che possiamo trovare due soluzioni di base corrispondenti per un'unica base su primale e duale.
Queste sono:
\begin{itemize}
	\item Soluzione di base primale: $ x = A_B^{-1} b_B $;
	\item Soluzione di base duale: $ y_B^T = c^T A_B^-1, \quad y_N = 0 $;
\end{itemize}
Si dice che le soluzioni di base sono \textbf{complementari}.

\par\smallskip
\noindent 
\textbf{\textsf{Dimostrazione}}
Vogliamo che $y^T(b - Ax)$ sia $=0$ soddisfatte le condizioni di base.
Applichiamo quindi la base:
$$
y^T(b - Ax) = \left( y_B^T, y_N^T \right) \binom{b_B - A_B x}{b_N - A_N x} = \left( c^T A_B^{-1}, 0 \right) \binom{b_B - A_B A_B^{-1} b_B}{b_N - A_N A_B^{-1} b_B}
$$
$$
= \left( c^T A_B^{-1}, 0 \right) \binom{0}{b_N - A_N A_B^{-1} b_B} = 0
$$

Questo nome non è a caso, in quanto si può dimostrare le due soluzioni sono in scarti complementari.
Da questo risultato, si ha che se entrambe le soluzioni sono ammissibili, cioé:
\begin{itemize}
	\item La primale è ammissibile: 
		$$ \forall i \in N \ \text{si ha} \ A_i x \leq b_i $$
		ergo i vincoli sono soddisfatti;
	\item La duale è ammissibile:
		$$ y \geq 0$$
\end{itemize}
questo è condizione sufficiente perche la soluzione sia ottima, e dagli scorsi corollari, sia l'ottima sia del primale che del duale.

Formalizziamo quanto detto in un teorema:
\begin{theorem}{Condizioni di ottimalità di soluzione di base}
	Dato un vertice del primale, ottenuto da una certa base, si può costruire il complemento duale sulla stessa base.
	Se entrambi i vertici ottenuti sono ammissibili, allora sono uguali e ottimi dei rispettivi problemi.
\end{theorem}

\end{document}


\documentclass[a4paper,11pt]{article}
\usepackage[a4paper, margin=8em]{geometry}

% usa i pacchetti per la scrittura in italiano
\usepackage[french,italian]{babel}
\usepackage[T1]{fontenc}
\usepackage[utf8]{inputenc}
\frenchspacing 

% usa i pacchetti per la formattazione matematica
\usepackage{amsmath, amssymb, amsthm, amsfonts}

% usa altri pacchetti
\usepackage{gensymb}
\usepackage{hyperref}
\usepackage{standalone}

% cose fluttuanti
\usepackage{float}

% imposta il titolo
\title{Appunti Ricerca Operativa}
\author{Luca Seggiani}
\date{2024}

% disegni
\usepackage{pgfplots}
\pgfplotsset{width=10cm,compat=1.9}

% imposta lo stile
% usa helvetica
\usepackage[scaled]{helvet}
% usa palatino
\usepackage{palatino}
% usa un font monospazio guardabile
\usepackage{lmodern}

\renewcommand{\rmdefault}{ppl}
\renewcommand{\sfdefault}{phv}
\renewcommand{\ttdefault}{lmtt}

% disponi il titolo
\makeatletter
\renewcommand{\maketitle} {
	\begin{center} 
		\begin{minipage}[t]{.8\textwidth}
			\textsf{\huge\bfseries \@title} 
		\end{minipage}%
		\begin{minipage}[t]{.2\textwidth}
			\raggedleft \vspace{-1.65em}
			\textsf{\small \@author} \vfill
			\textsf{\small \@date}
		\end{minipage}
		\par
	\end{center}

	\thispagestyle{empty}
	\pagestyle{fancy}
}
\makeatother

% disponi teoremi
\usepackage{tcolorbox}
\newtcolorbox[auto counter, number within=section]{theorem}[2][]{%
	colback=blue!10, 
	colframe=blue!40!black, 
	sharp corners=northwest,
	fonttitle=\sffamily\bfseries, 
	title=Teorema~\thetcbcounter: #2, 
	#1
}

% disponi definizioni
\newtcolorbox[auto counter, number within=section]{definition}[2][]{%
	colback=red!10,
	colframe=red!40!black,
	sharp corners=northwest,
	fonttitle=\sffamily\bfseries,
	title=Definizione~\thetcbcounter: #2,
	#1
}

% disponi problemi
\newtcolorbox[auto counter, number within=section]{problem}[2][]{%
	colback=green!10,
	colframe=green!40!black,
	sharp corners=northwest,
	fonttitle=\sffamily\bfseries,
	title=Problema~\thetcbcounter: #2,
	#1
}

% disponi codice
\usepackage{listings}
\usepackage[table]{xcolor}

\lstdefinestyle{codestyle}{
		backgroundcolor=\color{black!5}, 
		commentstyle=\color{codegreen},
		keywordstyle=\bfseries\color{magenta},
		numberstyle=\sffamily\tiny\color{black!60},
		stringstyle=\color{green!50!black},
		basicstyle=\ttfamily\footnotesize,
		breakatwhitespace=false,         
		breaklines=true,                 
		captionpos=b,                    
		keepspaces=true,                 
		numbers=left,                    
		numbersep=5pt,                  
		showspaces=false,                
		showstringspaces=false,
		showtabs=false,                  
		tabsize=2
}

\lstdefinestyle{shellstyle}{
		backgroundcolor=\color{black!5}, 
		basicstyle=\ttfamily\footnotesize\color{black}, 
		commentstyle=\color{black}, 
		keywordstyle=\color{black},
		numberstyle=\color{black!5},
		stringstyle=\color{black}, 
		showspaces=false,
		showstringspaces=false, 
		showtabs=false, 
		tabsize=2, 
		numbers=none, 
		breaklines=true
}

\lstdefinelanguage{javascript}{
	keywords={typeof, new, true, false, catch, function, return, null, catch, switch, var, if, in, while, do, else, case, break},
	keywordstyle=\color{blue}\bfseries,
	ndkeywords={class, export, boolean, throw, implements, import, this},
	ndkeywordstyle=\color{darkgray}\bfseries,
	identifierstyle=\color{black},
	sensitive=false,
	comment=[l]{//},
	morecomment=[s]{/*}{*/},
	commentstyle=\color{purple}\ttfamily,
	stringstyle=\color{red}\ttfamily,
	morestring=[b]',
	morestring=[b]"
}

% disponi sezioni
\usepackage{titlesec}

\titleformat{\section}
	{\sffamily\Large\bfseries} 
	{\thesection}{1em}{} 
\titleformat{\subsection}
	{\sffamily\large\bfseries}   
	{\thesubsection}{1em}{} 
\titleformat{\subsubsection}
	{\sffamily\normalsize\bfseries} 
	{\thesubsubsection}{1em}{}

% disponi alberi
\usepackage{forest}

\forestset{
	rectstyle/.style={
		for tree={rectangle,draw,font=\large\sffamily}
	},
	roundstyle/.style={
		for tree={circle,draw,font=\large}
	}
}

% disponi algoritmi
\usepackage{algorithm}
\usepackage{algorithmic}
\makeatletter
\renewcommand{\ALG@name}{Algoritmo}
\makeatother

% disponi numeri di pagina
\usepackage{fancyhdr}
\fancyhf{} 
\fancyfoot[L]{\sffamily{\thepage}}

\makeatletter
\fancyhead[L]{\raisebox{1ex}[0pt][0pt]{\sffamily{\@title \ \@date}}} 
\fancyhead[R]{\raisebox{1ex}[0pt][0pt]{\sffamily{\@author}}}
\makeatother

\begin{document}

% sezione (data)
\section{Lezione del 07-10-24}

% stili pagina
\thispagestyle{empty}
\pagestyle{fancy}

% testo
\subsection{Algoritmo del simplesso primale}
Supponiamo di avere un problema LP in formato primale standard con $n = 8$ vincoli, espresso come:

\[
	\begin{cases}
		\max(c^T \cdot x) \\
		Ax \leq b
	\end{cases}
\]

e con poliedro:

\begin{center}
	\begin{tikzpicture}
	\begin{axis}[
			axis lines = middle,
			xlabel = {$x_A$},
			ylabel = {$x_B$},
			xmin=0, xmax=7.9,
			ymin=0, ymax=3.9,
			samples=100,
			width=13cm, height=7cm,
			legend pos=north east
		]

	\addplot[blue, thick] coordinates {(0,1) (1,0)};  
	\addplot[blue, thick] coordinates {(1,0) (2,0)};  
	\addplot[blue, thick] coordinates {(0,1) (0,2)};  
	\addplot[blue, thick] coordinates {(0,2) (1,3)};  
	\addplot[blue, thick] coordinates {(1,3) (2,3)};  
	\addplot[blue, thick] coordinates {(1,3) (2,3)};  
	\addplot[blue, thick] coordinates {(2,3) (3,2)};  
	\addplot[blue, thick] coordinates {(3,2) (3,1)};  
	\addplot[blue, thick] coordinates {(3,1) (2,0)}; 

	\node at (axis cs:0.7,0.7) [anchor=center] {1};
	\node at (axis cs:1.5,0.3) [anchor=center] {2};
	\node at (axis cs:2.3,0.7) [anchor=center] {3};
	\node at (axis cs:2.7,1.5) [anchor=center] {4};
	\node at (axis cs:2.3,2.3) [anchor=center] {5};
	\node at (axis cs:1.5,2.7) [anchor=center] {6};
	\node at (axis cs:0.7,2.3) [anchor=center] {7};
	\node at (axis cs:0.3,1.5) [anchor=center] {8};
		
	\end{axis}
	\end{tikzpicture}
\end{center}

Scegliamo un vertice di partenza, per adesso ad arbitrio: diciamo $\bar{x} = (0, 1)$ (vedremo in seguito un'algoritmo particolare per ricavare un vertice, che ci permetterà anche di determinare se il poliedro è vuoto o meno).
Ci chiediamo se questo vertice $\bar{x}$ è ottimo.
Visto che è vertice, abbiamo che per una matrice $A_B$ e un vettore $b_B$ di base:
$$
\bar{x} = A_B^{-1} b_B
$$
e che possiamo costruire il complementare duale $\bar{y}$, impostando a zero le variabili fuori base e risolvendo il sistema:
$$
\bar{y} = (cA_B^{-1}, 0)
$$
e applicare il test di ottimalità, cioè vedere se:
$$
cA_B^{-1} \geq 0
$$
ergo $\bar{y} \in D$, quindi il complementare duale esiste e il vertice è ottimo.
Se questa condizione risulta verificata, possiamo fermarci, in quanto abbiamo trovato la soluzione ottimale.

In caso contrario, avremo $\exists k \in B$ tale che $\bar{y}_k < 0$.
Dovremo quindi spostarci verso un'altro vertice, magari \textit{adiacente}, che dal punto di vista delle basi, significa cambiare un solo indice di base, conservando gli altri.
Possiamo formalizzare questa affermazione definendo un \textbf{indice uscente} $h$ ed un \textbf{indice entrante} $k$.
Sostituire un indice di base significa effettuare il cambio di base:
$$
B := B \setminus \{h\} \cup \{k\}
$$

Resta la domanda di \textit{quale} spigolo scegliere: in uno spazio vettoriale $\mathbb{R}^n$, ho a disposizione $n$ spigoli che si staccano dallo stesso vertice.
Ovviamente, vorrei scegliere uno spigolo che accresce la funzione obiettivo, e si può dimostrare che ne esiste almeno uno: altrimenti sarei già all'ottimo.
Inoltre, avendo un metodo per scegliere sempre lo spigolo di crescita maggiore potrei dire 2 cose: l'algoritmo tende all'ottimo (il vertice da cui non si staccano spigoli che accrescono la funzione obiettivo), e termina in un numero finito di passi (prima o poi raggiungerà inevitabilmente un vertice che massimizza la funzione).

Prima però dobbiamo chiarire una questione: scegliere un nuovo spigolo significa trovare 2 indici base, uno da eliminare e uno da inserire.
Si può dire che il primo indice, quello uscente, indica anche la direzione di spostamento: allentando un vincolo ci spostiamo sulla semiretta del prossimo.
Allo stesso tempo, scegliere un indice da rimuovere non basta: dobbiamo scegliere quale introdurre, che geometricamente significa capire \textit{quanto} ci possiamo spostare lungo la semiretta prima di uscire dalla regione di ammissibilità.
Vediamo quindi questi due passaggi in ordine.

\begin{itemize}
	\item \textbf{\textsf{Indice uscente}} \\
Diciamo:
$$
W = \left( -A_B^{-1} \right)
$$
e prendiamo le colonne $W^i$ corrispondenti agli indici di base scelti.

Possiamo allora dire che l'equazione degli spigoli dati dalle disequazioni all'indice $i$ sono:
$$
\bar{x} + \lambda W^i
$$

Mettiamo questa equazione nella funzione costo:
$$
c\left( \bar{x} + \lambda W^i \right) = c \bar{x} + \lambda c W^{i}
$$

Qui abbiamo $c\bar{x}$, che è il valore nel vertice, e un'altro termine scalato da $\lambda$.
Ricordiamo poi che $cA_B^{-1} = \bar{y}_B$, e che $W = \left( -A_B^{-1} \right)$, ergo $c W^{i} = -\bar{y}_B$:
$$
c \bar{x} + \lambda c W^{i} = c \bar{x} - \lambda \bar{y}_B
$$
Vogliamo quindi "allentare" l'indice (e il corrispettivo vertice) che ci dà $\bar{y}_i < 0$, in quanto è quello che restituisce un $c W^{i} > 0$, e quindi un accrescimento della funzione. 
Definiamo allora questo indice:
\begin{definition}{Indice uscente primale}
Chiamiamo indice uscente $h$, da una certa soluzione della base $B$:
$$h := \min\{ i \in B \ \text{t.c.} \ \bar{y_i} < 0  \}$$
\end{definition}
Il $\min$ significa che in caso di più $i$ negativi, si adotta la regola anticiclo (di Bland) di scegliere il primo.
In caso di nessun $i$ negativo, la complementare duale esiste e siamo sull'ottimo.

	\item \textbf{\textsf{Indice entrante}} \\
Adesso cerchiamo per quali $\lambda$ lo spigolo $\bar{x} + \lambda W^h$ resta ammissibile, ergo soddisfa:
$$
A_i \left( \bar{x} + \lambda W^h \right) \leq b_i, \quad i \in N 
$$
Questo significa effettivamente vedere qual'è il primo vincolo che "stringiamo", o che incontriamo, spostandoci lungo la semiretta ottenuta allentando il vincolo dato dall'indice uscente.

Possiamo dire:
$$
A_i \left( \bar{x} + \lambda W^h \right) = A_i \bar{x} + \lambda A_i W^h \leq b_i
$$
da cui si ricava (e si risolve) la disequazione di primo grado:
$$
\lambda A_i W^h \leq b_i - A_i \bar{x} \Rightarrow \lambda \leq \frac{b_i - A_i \bar{x}}{A_i W^h}
$$
Notiamo che se fosse $A_i W^h \leq 0, \ \forall i \in N$, avremmo che l'indice rappresenta una direzione di regressione, in quanto $\lambda \rightarrow +\infty$.
Si ha quindi che il duale non ha soluzione, e il primale $\rightarrow +\infty$.
In caso contrario, noi vogliamo trovare il primo vincolo che si va a stringere, quindi dovremo calcolare tutti gli $r_i$:
$$
r_i = \frac{b_i - A_i \bar{x}}{A_i W^h}, \quad i \in N, \quad A_i W^h > 0
$$
e scegliere l'indice che dà $\vartheta = \min(r_i)$.
Definiamo allora anche questo indice:
\begin{definition}{Indice entrante primale}
	Chiamiamo indice entrante $k$, da una certa soluzione della base $B$ e un certo indice uscente $h$:
	$$
	h := \min\{ i \in N \ \text{t.c.} \ A_i W^h > 0, \quad \frac{b_i - A_i \bar{x}}{A_i W^h} = \vartheta \}	
	$$
\end{definition}
Anche qui, il $\min$ serve a selezionare il primo indice valido, ed è una regola anticiclo (di Bland).
Notiamo due possibili situazioni:
\begin{itemize}
	\item Si potrebbero avere più $r_i$ uguali: questi rappresentano soluzioni di base degenere \textit{in arrivo}, in quanto sono più modi di arrivare allo stesso vertice stringendo vincoli diversi;
	\item Si potrebbe avere un $r_i$ nullo: questo significa che il vertice è sullo stesso vertice da dove siamo partiti, ergo rappresenta una soluzione di base degenere \textit{in partenza}.
\end{itemize}
Come prima, le regole anticiclo di Bland assicurano anche che l'algoritmo non si blocchi a ciclare su queste soluzioni degeneri.
\end{itemize}

Abbiamo quindi tutti gli strumenti necessari alla formulazione dell'algoritmo del simplesso:
\begin{algorithm}[H]
\caption{del simplesso primale}
\begin{algorithmic}
	\STATE \textbf{Input:} un problema LP in forma primale standard
	\STATE \textbf{Output:} la soluzione ottima 
	\STATE Trova una base B che genera una soluzione di base primale ammissibile.
	\STATE \textsf{ciclo:}
	\STATE Calcola la soluzione di base primale $\bar{x} = A_b^{-1} b_B$ e la soluzione di base duale $\bar{y} = (cA_b^{-1}, 0)$
	\IF{$\bar{y_B} \geq 0$}
		\STATE Fermati, $\bar{x}$ è ottima per $P$ e $\bar{y}$ è ottima per $D$
	\ELSE
		\STATE Calcola l'indice uscente: 
		$$
		h := \min\{ i \in B \ \text{t.c.} \ \bar{y_i} < 0 \}
		$$
		poni $W := -A_B^{-1}$ e indica con $W^h$ la $h$-esima colonna di $W$
	\ENDIF
	\IF{$A_i W^h \leq 0 \quad \forall i \in N$}
		\STATE Fermati, $P \rightarrow +\infty$ e $D$ non ha soluzione ottima.
	\ELSE
		\STATE Calcola:
		$$
		\vartheta = \min\{ \frac{b_i - A_i \bar{x}}{A_i W^h} \text{t.c.} \quad i \in N, \quad A_i W^h > 0 \}
		$$
		e trova l'indice entrante: 
		$$ 
		h := \min\{ i \in N \ \text{t.c.} \ A_i W^h > 0, \quad \frac{b_i - A_i \bar{x}}{A_i W^h} = \vartheta \} 
		$$
	\ENDIF
	\STATE Aggiorna la base come:
	$$
	B := B \setminus \{h\} \cup \{k\}
	$$
	\STATE Torna a \textsf{ciclo}
\end{algorithmic}
\end{algorithm}

\end{document}

\end{document}