
\documentclass[a4paper,11pt]{article}
\usepackage[a4paper, margin=8em]{geometry}

% usa i pacchetti per la scrittura in italiano
\usepackage[french,italian]{babel}
\usepackage[T1]{fontenc}
\usepackage[utf8]{inputenc}
\frenchspacing 

% usa i pacchetti per la formattazione matematica
\usepackage{amsmath, amssymb, amsthm, amsfonts}

% usa altri pacchetti
\usepackage{gensymb}
\usepackage{hyperref}
\usepackage{standalone}

% imposta il titolo
\title{Appunti Ricerca Operativa}
\author{Luca Seggiani}
\date{2024}

% disegni
\usepackage{pgfplots}
\pgfplotsset{width=10cm,compat=1.9}

% imposta lo stile
% usa helvetica
\usepackage[scaled]{helvet}
% usa palatino
\usepackage{palatino}
% usa un font monospazio guardabile
\usepackage{lmodern}

\renewcommand{\rmdefault}{ppl}
\renewcommand{\sfdefault}{phv}
\renewcommand{\ttdefault}{lmtt}

% disponi il titolo
\makeatletter
\renewcommand{\maketitle} {
	\begin{center} 
		\begin{minipage}[t]{.8\textwidth}
			\textsf{\huge\bfseries \@title} 
		\end{minipage}%
		\begin{minipage}[t]{.2\textwidth}
			\raggedleft \vspace{-1.65em}
			\textsf{\small \@author} \vfill
			\textsf{\small \@date}
		\end{minipage}
		\par
	\end{center}

	\thispagestyle{empty}
	\pagestyle{fancy}
}
\makeatother

% disponi teoremi
\usepackage{tcolorbox}
\newtcolorbox[auto counter, number within=section]{theorem}[2][]{%
	colback=blue!10, 
	colframe=blue!40!black, 
	sharp corners=northwest,
	fonttitle=\sffamily\bfseries, 
	title=Teorema~\thetcbcounter: #2, 
	#1
}

% disponi definizioni
\newtcolorbox[auto counter, number within=section]{definition}[2][]{%
	colback=red!10,
	colframe=red!40!black,
	sharp corners=northwest,
	fonttitle=\sffamily\bfseries,
	title=Definizione~\thetcbcounter: #2,
	#1
}

% disponi problemi
\newtcolorbox[auto counter, number within=section]{problem}[2][]{%
	colback=green!10,
	colframe=green!40!black,
	sharp corners=northwest,
	fonttitle=\sffamily\bfseries,
	title=Problema~\thetcbcounter: #2,
	#1
}

% disponi codice
\usepackage{listings}
\usepackage[table]{xcolor}

\lstdefinestyle{codestyle}{
		backgroundcolor=\color{black!5}, 
		commentstyle=\color{codegreen},
		keywordstyle=\bfseries\color{magenta},
		numberstyle=\sffamily\tiny\color{black!60},
		stringstyle=\color{green!50!black},
		basicstyle=\ttfamily\footnotesize,
		breakatwhitespace=false,         
		breaklines=true,                 
		captionpos=b,                    
		keepspaces=true,                 
		numbers=left,                    
		numbersep=5pt,                  
		showspaces=false,                
		showstringspaces=false,
		showtabs=false,                  
		tabsize=2
}

\lstdefinestyle{shellstyle}{
		backgroundcolor=\color{black!5}, 
		basicstyle=\ttfamily\footnotesize\color{black}, 
		commentstyle=\color{black}, 
		keywordstyle=\color{black},
		numberstyle=\color{black!5},
		stringstyle=\color{black}, 
		showspaces=false,
		showstringspaces=false, 
		showtabs=false, 
		tabsize=2, 
		numbers=none, 
		breaklines=true
}

\lstdefinelanguage{javascript}{
	keywords={typeof, new, true, false, catch, function, return, null, catch, switch, var, if, in, while, do, else, case, break},
	keywordstyle=\color{blue}\bfseries,
	ndkeywords={class, export, boolean, throw, implements, import, this},
	ndkeywordstyle=\color{darkgray}\bfseries,
	identifierstyle=\color{black},
	sensitive=false,
	comment=[l]{//},
	morecomment=[s]{/*}{*/},
	commentstyle=\color{purple}\ttfamily,
	stringstyle=\color{red}\ttfamily,
	morestring=[b]',
	morestring=[b]"
}

% disponi sezioni
\usepackage{titlesec}

\titleformat{\section}
	{\sffamily\Large\bfseries} 
	{\thesection}{1em}{} 
\titleformat{\subsection}
	{\sffamily\large\bfseries}   
	{\thesubsection}{1em}{} 
\titleformat{\subsubsection}
	{\sffamily\normalsize\bfseries} 
	{\thesubsubsection}{1em}{}

% disponi alberi
\usepackage{forest}

\forestset{
	rectstyle/.style={
		for tree={rectangle,draw,font=\large\sffamily}
	},
	roundstyle/.style={
		for tree={circle,draw,font=\large}
	}
}

% disponi algoritmi
\usepackage{algorithm}
\usepackage{algorithmic}
\makeatletter
\renewcommand{\ALG@name}{Algoritmo}
\makeatother

% disponi numeri di pagina
\usepackage{fancyhdr}
\fancyhf{} 
\fancyfoot[L]{\sffamily{\thepage}}

\makeatletter
\fancyhead[L]{\raisebox{1ex}[0pt][0pt]{\sffamily{\@title \ \@date}}} 
\fancyhead[R]{\raisebox{1ex}[0pt][0pt]{\sffamily{\@author}}}
\makeatother

\begin{document}
\maketitle
\documentclass[a4paper,11pt]{article}
\usepackage[a4paper, margin=8em]{geometry}

% usa i pacchetti per la scrittura in italiano
\usepackage[french,italian]{babel}
\usepackage[T1]{fontenc}
\usepackage[utf8]{inputenc}
\frenchspacing 

% usa i pacchetti per la formattazione matematica
\usepackage{amsmath, amssymb, amsthm, amsfonts}

% usa altri pacchetti
\usepackage{gensymb}
\usepackage{hyperref}
\usepackage{standalone}

% imposta il titolo
\title{Appunti Ricerca Operativa}
\author{Luca Seggiani}
\date{23-09-24}

% imposta lo stile
% usa helvetica
\usepackage[scaled]{helvet}
% usa palatino
\usepackage{palatino}
% usa un font monospazio guardabile
\usepackage{lmodern}

\renewcommand{\rmdefault}{ppl}
\renewcommand{\sfdefault}{phv}
\renewcommand{\ttdefault}{lmtt}

% disponi teoremi
\usepackage{tcolorbox}
\newtcolorbox[auto counter, number within=section]{theorem}[2][]{%
	colback=blue!10, 
	colframe=blue!40!black, 
	sharp corners=northwest,
	fonttitle=\sffamily\bfseries, 
	title=Teorema~\thetcbcounter: #2, 
	#1
}

% disponi definizioni
\newtcolorbox[auto counter, number within=section]{definition}[2][]{%
	colback=red!10,
	colframe=red!40!black,
	sharp corners=northwest,
	fonttitle=\sffamily\bfseries,
	title=Definizione~\thetcbcounter: #2,
	#1
}

% disponi problemi
\newtcolorbox[auto counter, number within=section]{problem}[2][]{%
	colback=green!10,
	colframe=green!40!black,
	sharp corners=northwest,
	fonttitle=\sffamily\bfseries,
	title=Problema~\thetcbcounter: #2,
	#1
}

% disponi codice
\usepackage{listings}
\usepackage[table]{xcolor}

\lstdefinestyle{codestyle}{
		backgroundcolor=\color{black!5}, 
		commentstyle=\color{codegreen},
		keywordstyle=\bfseries\color{magenta},
		numberstyle=\sffamily\tiny\color{black!60},
		stringstyle=\color{green!50!black},
		basicstyle=\ttfamily\footnotesize,
		breakatwhitespace=false,         
		breaklines=true,                 
		captionpos=b,                    
		keepspaces=true,                 
		numbers=left,                    
		numbersep=5pt,                  
		showspaces=false,                
		showstringspaces=false,
		showtabs=false,                  
		tabsize=2
}

\lstdefinestyle{shellstyle}{
		backgroundcolor=\color{black!5}, 
		basicstyle=\ttfamily\footnotesize\color{black}, 
		commentstyle=\color{black}, 
		keywordstyle=\color{black},
		numberstyle=\color{black!5},
		stringstyle=\color{black}, 
		showspaces=false,
		showstringspaces=false, 
		showtabs=false, 
		tabsize=2, 
		numbers=none, 
		breaklines=true
}

\lstdefinelanguage{javascript}{
	keywords={typeof, new, true, false, catch, function, return, null, catch, switch, var, if, in, while, do, else, case, break},
	keywordstyle=\color{blue}\bfseries,
	ndkeywords={class, export, boolean, throw, implements, import, this},
	ndkeywordstyle=\color{darkgray}\bfseries,
	identifierstyle=\color{black},
	sensitive=false,
	comment=[l]{//},
	morecomment=[s]{/*}{*/},
	commentstyle=\color{purple}\ttfamily,
	stringstyle=\color{red}\ttfamily,
	morestring=[b]',
	morestring=[b]"
}

% disponi sezioni
\usepackage{titlesec}

\titleformat{\section}
	{\sffamily\Large\bfseries} 
	{\thesection}{1em}{} 
\titleformat{\subsection}
	{\sffamily\large\bfseries}   
	{\thesubsection}{1em}{} 
\titleformat{\subsubsection}
	{\sffamily\normalsize\bfseries} 
	{\thesubsubsection}{1em}{}

% disponi alberi
\usepackage{forest}

\forestset{
	rectstyle/.style={
		for tree={rectangle,draw,font=\large\sffamily}
	},
	roundstyle/.style={
		for tree={circle,draw,font=\large}
	}
}

% disponi algoritmi
\usepackage{algorithm}
\usepackage{algorithmic}
\makeatletter
\renewcommand{\ALG@name}{Algoritmo}
\makeatother

% disponi numeri di pagina
\usepackage{fancyhdr}
\fancyhf{} 
\fancyfoot[L]{\sffamily{\thepage}}

\makeatletter
\fancyhead[L]{\raisebox{1ex}[0pt][0pt]{\sffamily{\@title \ \@date}}} 
\fancyhead[R]{\raisebox{1ex}[0pt][0pt]{\sffamily{\@author}}}
\makeatother

% disegni
\usepackage{pgfplots}
\pgfplotsset{width=10cm,compat=1.9}

\begin{document}
% sezione (data)
\section{Lezione del 23-09-24}

% stili pagina
\thispagestyle{empty}
\pagestyle{fancy}

% testo
\subsection{Introduzione}

\subsubsection{Programma del corso}
Il corso di ricerca operativa si divide in 4 parti:

\begin{enumerate}
	\item Modello di Programmazione Lineare;
	\item Programmazione Lineare su reti, ergo programmazione lineare su grafi;
	\item Programmazione Lineare intera, ergo programmazione lineare col vincolo $x \in \mathbb{Z}^n$;
	\item Programmazione Non Lineare.
\end{enumerate}

Le prime 3 parti hanno come prerequisiti l'algebra lineare: in particolare operazioni matriciali, prodotti scalari, sistemi lineari, teorema di Rouché-Capelli.
La quarta parte richiede invece conoscenze di Analisi II.

\subsubsection{Un problema di programmazione lineare}

La ricerca operativa si occupa di risolvere problemi di ottimizzazione con variabili decisionali e risorse limitate.
Poniamo un problema di esempio:

\begin{problem}{Produzione}
Una ditta produce due prodotti: \textbf{laminato A} e \textbf{laminato B}.
Ogni prodotto deve passare attraverso diversi reparti: il reparto \textbf{materie prime}, il reparto \textbf{taglio}, il reparto \textbf{finiture A} e il reparto \textbf{finiture B}.
Il guadagno è rispettivamente di 8.4 e 11.2 (unità di misura irrilevante) per ogni tipo di laminato.

Ora, nel reparto materie prime, il laminato A occupa 30, ore, e lo B 20 ore.
Nel reparto taglio il laminato A occupa 10 ore e lo B 20 ore.
Il laminato A occupa poi 20 ore nel reparto finiture A, mentre il laminato B occupa 30 ore nel reparto finiture B.
I reparti hanno a disposizione, rispettivamente, 120, 80, 62 e 105 ore.
Possiamo porre queste informazioni in forma tabulare:

	\center \rowcolors{2}{green!10}{green!40!black!20}
	\begin{tabular} { | c || c | c | c | }
		\hline
		\bfseries Reparto & \bfseries Capienza & \bfseries Laminato A & \bfseries Laminato B \\
		\hline 
		Materie prime & 120 & 30 & 20 \\
		Taglio & 80 & 10 & 20 \\
		Finiture A & 62 & 20 & / \\
		Finiture B & 105 & / & 30 \\
		\hline
		\textbf{Guadagno} & & 8.4 & 11.2 \\
		\hline
	\end{tabular}

	\par\bigskip

Quello che ci interessa è chiaramente massimizzare il guadagno.
\end{problem}

Decidiamo di modellizzare questa situazione con un modello matematico.

Il guadagno che abbiamo dai laminati rappresenta una \textbf{funzione obiettivo}, ovvero la funzione che vogliamo ottimizzare.
Ottimizzare significa trovare il modo migliore di massimizzare o minimizzare i valori della funzione agendo sulle variabili decisionali.
La funzione obiettivo va ottimizzata rispettando determinati \textbf{vincoli}, che modellizzano il fatto che le risorse sono limitate.
Una \textbf{soluzione ammissibile} è una qualsiasi soluzione che rispetta i vincoli del problema.
Chiamiamo quindi \textbf{regione ammissibile} l'insieme di tutte le soluzioni ammissibili.
All'interno della regione ammissibile c'è la soluzione che cerchiamo, ovvero la \textbf{soluzione ottima}.

Decidiamo quindi le \textbf{variabili decisionali}, ed esplicitiamo la funzione obiettivo e i vincoli.

In questo caso le variabili decisionali saranno le quantità di laminato A e B da produrre, che individuano un punto in $ \mathbb{R}^2 $ denominato $ ( x_A, x_B ) $. 
Decidere di usare la soluzione $ (1,1) $ significa decidere di produrre 1 unità di laminato A e 1 unità di laminato B, per un guadagno complessivo di $ 8.4 + 11.2 = 19.6 $.

La funzione obiettivo sarà quindi:

$$ f(x_A, x_B) = 8.4 x_A + 11.2 x_B, \quad f: \mathbb{R}^2 \rightarrow \mathbb{R} $$

lineare, e noi saremo interessati a:

$$ \max(f(x_A, x_B)) $$

rispettando i vincoli, ergo nella regione ammissibile.
Per esprimere questi vincoli, cioè il tempo limitato all'interno di ogni reparto, introduciamo il sistema di disequazioni:

\[
	\begin{cases}
		30 x_A + 20 x_B \leq 120 \\
		10 x_A + 20 x_B \leq 80	\\
		20 x_A + 0 x_B \leq 62 \\	
		0 x_A + 30 x_B \leq 105 \\
		- x_A \leq 0 \\
		- x_B \leq 0 \\
	\end{cases}
\]

dove notiamo le ultime due disequazioni indicano la positività di $x_A$ e $x_B$, in forma $ f(x_A, x_B) \leq b $.
Questo sistema non indica altro che la regione ammissibile.

Possiamo riscrivere questo modello usando la notazione dell'algebra lineare.
La funzione obiettiva e i vincoli diventano semplicemente:

\[
	\begin{cases}
		\max(c^T \cdot x) \\
		A \cdot x \leq b	
	\end{cases}
\]

dove $c$ rappresenta il vettore dei costi, $A$ rappresenta la matrice dei costi a $b$ il vettore dei vincoli.
$c$ è trasposto per indicare prodotto fra vettori.

Possiamo scrivere $A$, $b$ e $c$ per esteso:

$$
A:
\begin{pmatrix}
	30 & 20 \\
	10 & 20 \\
	20 & 0 \\
	0 & 30 \\
	-1 & 0 \\
	0 & -1
\end{pmatrix}, \quad
b:
\begin{pmatrix}
	120 \\
	80 \\
	62 \\ 
	105 \\ 
	0 \\ 
	0 
\end{pmatrix}, \quad 
c:
\begin{pmatrix}
	8.4 \\
	11.2 \\
\end{pmatrix}
$$

Notiamo come $A$ e $b$ hanno dimensione verticale $ 4 + 2 = 6 $, dai 4 vincoli superiori e i 2 vincoli inferiori.

A questo punto, possiamo disegnare la regione ammissibile come l'intersezione dei semipiani individuati da ogni singola disuguaglianza.
Si riporta un grafico:

\begin{tikzpicture}
\begin{axis}[
    axis lines = middle,
    xlabel = {$x_A$},
    ylabel = {$x_B$},
    xmin=0, xmax=6,
    ymin=0, ymax=6,
    domain=0:10,
    samples=100,
    width=10cm, height=10cm,
    legend pos=north east
  ]

% regione ammissibile

	\addplot[fill=gray, opacity=0.4] 
    coordinates {
			(0, 0)
			(3.1, 0)
			(3.1, 1.35)
			(2,3)
			(1, 3.5)
			(0, 3.5)
		};

% rette

\addplot[domain=2:3.1, thick, blue] {6 - 1.5*x}; 
\addlegendentry{$ 30 x_A + 20 x_B \leq 120 $}

\addplot[domain=1:2, thick, green] {4 - 0.5*x}; 
\addlegendentry{$ 10 x_A + 20 x_B \leq 80 $}

\addplot[thick, purple] coordinates {(3.1, 0) (3.1, 1.35)};
\addlegendentry{$ 20 x_A + 0 x_B \leq 62 $}

\addplot[domain=0:1, thick, red] {3.5}; 
\addlegendentry{$ 0 x_A + 30 x_B \leq 105 $}
	
\end{axis}
\end{tikzpicture}

In diversi colori sono riportate i margini delle disequazioni, mentre in grigio è evidenziata la regione ammissibile.

\par\smallskip

Il modello finora descritto prende il nome di modello di programmazione lineare, e permette di formulare problemi di programmazione lineare (LP).

\begin{definition}{Problema di programmazione lineare (1)}
Un problema di programmazione lineare (LP) riguarda l'ottimizzazione di una funzione lineare in più variabili
soggetta a vincoli di tipo $ =, \ \leq $ e $ \geq $, ovvero in forma:
\[
	\begin{cases}
			\min / \max(c^T \cdot x) \\
			A x \leq b \\
			... \\
			B x \geq d \\
			... \\
			C x = e \\
			...
	\end{cases}
\]
\end{definition}

"Programmazione" qui non ha alcun legame col concetto di programmazione informatica, ma si riferisce al fatto che il modello è effettivamente \textit{programmabile}.

"Lineare" si riferisce alla linearità del problema (e quindi del modello).




\end{document}

\documentclass[a4paper,11pt]{article}
\usepackage[a4paper, margin=8em]{geometry}

% usa i pacchetti per la scrittura in italiano
\usepackage[french,italian]{babel}
\usepackage[T1]{fontenc}
\usepackage[utf8]{inputenc}
\frenchspacing 

% usa i pacchetti per la formattazione matematica
\usepackage{amsmath, amssymb, amsthm, amsfonts}

% usa altri pacchetti
\usepackage{gensymb}
\usepackage{hyperref}
\usepackage{standalone}

% imposta il titolo
\title{Appunti Ricerca Operativa}
\author{Luca Seggiani}
\date{24-09-24}

% imposta lo stile
% usa helvetica
\usepackage[scaled]{helvet}
% usa palatino
\usepackage{palatino}
% usa un font monospazio guardabile
\usepackage{lmodern}

\renewcommand{\rmdefault}{ppl}
\renewcommand{\sfdefault}{phv}
\renewcommand{\ttdefault}{lmtt}

% disponi teoremi
\usepackage{tcolorbox}
\newtcolorbox[auto counter, number within=section]{theorem}[2][]{%
	colback=blue!10, 
	colframe=blue!40!black, 
	sharp corners=northwest,
	fonttitle=\sffamily\bfseries, 
	title=Teorema~\thetcbcounter: #2, 
	#1
}

% disponi definizioni
\newtcolorbox[auto counter, number within=section]{definition}[2][]{%
	colback=red!10,
	colframe=red!40!black,
	sharp corners=northwest,
	fonttitle=\sffamily\bfseries,
	title=Definizione~\thetcbcounter: #2,
	#1
}

% disponi codice
\usepackage{listings}
\usepackage[table]{xcolor}

\lstdefinestyle{codestyle}{
		backgroundcolor=\color{black!5}, 
		commentstyle=\color{codegreen},
		keywordstyle=\bfseries\color{magenta},
		numberstyle=\sffamily\tiny\color{black!60},
		stringstyle=\color{green!50!black},
		basicstyle=\ttfamily\footnotesize,
		breakatwhitespace=false,         
		breaklines=true,                 
		captionpos=b,                    
		keepspaces=true,                 
		numbers=left,                    
		numbersep=5pt,                  
		showspaces=false,                
		showstringspaces=false,
		showtabs=false,                  
		tabsize=2
}

\lstdefinestyle{shellstyle}{
		backgroundcolor=\color{black!5}, 
		basicstyle=\ttfamily\footnotesize\color{black}, 
		commentstyle=\color{black}, 
		keywordstyle=\color{black},
		numberstyle=\color{black!5},
		stringstyle=\color{black}, 
		showspaces=false,
		showstringspaces=false, 
		showtabs=false, 
		tabsize=2, 
		numbers=none, 
		breaklines=true
}

\lstdefinelanguage{javascript}{
	keywords={typeof, new, true, false, catch, function, return, null, catch, switch, var, if, in, while, do, else, case, break},
	keywordstyle=\color{blue}\bfseries,
	ndkeywords={class, export, boolean, throw, implements, import, this},
	ndkeywordstyle=\color{darkgray}\bfseries,
	identifierstyle=\color{black},
	sensitive=false,
	comment=[l]{//},
	morecomment=[s]{/*}{*/},
	commentstyle=\color{purple}\ttfamily,
	stringstyle=\color{red}\ttfamily,
	morestring=[b]',
	morestring=[b]"
}

% disponi sezioni
\usepackage{titlesec}

\titleformat{\section}
	{\sffamily\Large\bfseries} 
	{\thesection}{1em}{} 
\titleformat{\subsection}
	{\sffamily\large\bfseries}   
	{\thesubsection}{1em}{} 
\titleformat{\subsubsection}
	{\sffamily\normalsize\bfseries} 
	{\thesubsubsection}{1em}{}

% disponi alberi
\usepackage{forest}

\forestset{
	rectstyle/.style={
		for tree={rectangle,draw,font=\large\sffamily}
	},
	roundstyle/.style={
		for tree={circle,draw,font=\large}
	}
}

% disponi algoritmi
\usepackage{algorithm}
\usepackage{algorithmic}
\makeatletter
\renewcommand{\ALG@name}{Algoritmo}
\makeatother

% disponi numeri di pagina
\usepackage{fancyhdr}
\fancyhf{} 
\fancyfoot[L]{\sffamily{\thepage}}

\makeatletter
\fancyhead[L]{\raisebox{1ex}[0pt][0pt]{\sffamily{\@title \ \@date}}} 
\fancyhead[R]{\raisebox{1ex}[0pt][0pt]{\sffamily{\@author}}}
\makeatother

\begin{document}
% sezione (data)
\section{Lezione del 24-09-24}

% stili pagina
\thispagestyle{empty}
\pagestyle{fancy}

% testo

\subsection{Forma primale standard}
Ciò che abbiamo formulato finora è un problema di programmazione lineare.
Possiamo dire che la forma:
\[
	\begin{cases}
			\max(c^T \cdot x) \\
			A x \leq b
	\end{cases}
\]
rappresenta un problema LP in forma \textbf{primale standard}, ricordando che $c$ è il vettore dei coefficienti della funzione obiettivo, $A$ la matrice dei coefficienti per ogni vincolo, e $b$ il vettore dei termini noti per ogni vincolo.

\begin{definition}{Forma primale standard}
	Un problema di programmazione lineare si dice in forma primale standard quando è espresso in forma:
	
	\[
		\begin{cases}
			\max(c^T \cdot x) \\
			Ax \leq b \\
		\end{cases}
	\]

\end{definition}
\par\smallskip
Si adotta una forma primale standard in quanto si può trasformare ogni problema LP in una forma di questo tipo.

\subsubsection{Normalizzazione di un problema LP}
Un modo per portare un problema LP qualsiasi in forma primale standard è:

\begin{enumerate}
	\item Si trasformano le disuguaglianze: $ \geq \ \leftrightarrow \ \leq $
	\item Si riscrivono le uguaglianze come coppie di diseguaglianze:
		$$
			f(x) = c \ \rightarrow \
		\begin{cases}
			f(x) \leq c \\ 
			f(x) \geq c
		\end{cases}
		$$
		da cui la (1):
		$$
			f(x) = c \ \rightarrow \
S		\begin{cases}
			f(x) \leq c \\ 
			-f(x) \leq -c
		\end{cases}
		$$
	\item Se il problema richiede il minimo, si nota che $ \max(f) = -\min(-f) $, e sopratutto:
		$$
		\bar{x} \in \mathrm{argmax}(f) \Leftrightarrow \bar{x} \in \mathrm{argmin}(-f)
		$$
		con $ \mathrm{argmax}(f) $ e $ \mathrm{argmin}(-f) $ rispettivamente gli insiemi dei punti di massimo e minimo.
		Questo significa che posso semplicemente cambiare di segno la funzione obiettivo per trovare da massimi minimi, e viceversa.
\end{enumerate}

Notiamo inoltre che, nella forma primale standard, si ha:
$$
	x \in R^n, \quad
	A \in R^{n \times m}, \quad
	b \in R^m, \quad
	c \in R^n
$$

\subsection{Proprietà generali di un problema LP}
La regione ammissibile di un problema PL si chiama \textbf{poliedro}.
Si può dare agilmente una definizione algebrica di poliedro:
\begin{definition}{Definizione algebrica di poliedro}
	Algebricamente, un poliedro è l'insieme delle soluzioni di un sistema di disequazioni lineari in $\mathbb{R}^n$ variabili:
	$$
		P = \{ x \in \mathbb{R}^n : Ax \leq b \}
	$$
\end{definition}

Questa regione in un problema LP prende il nome di regione ammissibile.

\begin{definition}{Definizione geometrica di poliedro}
	Geometricamente, un poliedro è l'intersezione di un numero finito di semispazi chiusi.
\end{definition}

I semispazi chiaramente sono lineari, e in $\mathbb{R}^2$ rappresenterebbero semipiani.
Chiusi significa che nelle disequazioni che descrivono i vincoli compargono solo $\leq$ e non $<$, ergo la regione ammissibile contiene la sua frontiera.

Possiamo dimostrare 4 proprietà dei poliedri:

\begin{enumerate}
	\item 
		Un'osservazione fondamentale è la seguente:
		\begin{theorem}{Soluzione ottimale di un problema LP}
			La soluzione ottimale di un problema LP è contenuta nella frontiera della regione ammissibile.
		\end{theorem}
		Questo si può ricavare dai teoremi di Fermat e di Weierstrass, e dalla convessità della regione ammissibile.
		Inanzitutto, si è stabilito che la soluzione ottimale non è altro che il massimo o minimo assoluto all'interno della regione ammissibile del problema.
		Il gradiente della funzione obiettiva è $\neq 0$ in ogni suo punto (funzione lineare a gradiente costante). 
		Da Fermat, i massimi e minimi hanno sempre gradiente $0$, ergo massimi o minimi locali (che esistono per Weierstrass) possono trovarsi solo sulla frontiera.
		A questo punto, possiamo imporre la convessità per asserire che quei punti di massimo o minimo sono anche globali. 

	\item 
		Prendiamo in esempio il poliedro dato da:
		\[
			\begin{cases}
				x_A > 0 \\ 
				x_B > 0
			\end{cases}
		\]
		o se vogliamo, in forma primale standard, dato dalle matrici $A$ e $b$:
		$$
		A:
		\begin{pmatrix}
			-1 & 0 \\	
			0 & -1 \\	
		\end{pmatrix}
		, \quad b:
		\begin{pmatrix}
			0 \\ 
			0
		\end{pmatrix}
		$$
		questo poliedro non è limitato nella direzione positiva, ergo può arrivare a valori di $x_A$ e $x_B$ che tendono a $+\infty$.
		Da ciò si ha che può accadere che un problema LP ammetta soluzioni $x$ tali che $x \rightarrow \pm \infty $,
		ovvero che il poliedro sia illimitato.
		In particolare, un poliedro limitato si dice \textbf{politopo}.
	\item Notiamo poi che la soluzione di un problema LP può non essere unica.
		Questo accade ad esempio quando la soluzione sta su un segmento di frontiera: a quel punto tutti i punti del segmento sono soluzione.
		Da questo segue che:
		\begin{theorem}{Unicità della soluzione ottimale di un problema LP}
			Se un problema LP ha almeno 2 soluzioni, allora ne ha infinite.
		\end{theorem}
		Ciò si può dimostrare come segue.
		Si riporta innanzitutto la notazione parametrica del segmento $\bar{zw}$, dati i due vettori di estremo $z$ e $w$:
		$$
			\lambda z + (1 - \lambda)w, \quad \lambda \in [ 0, 1 ]
		$$
		A questo punto si pone che $z$ e $w$ sono entrambi soluzioni ottime, ergo: 
		$$ 
			\max(c^T \cdot x) = c^T z = c^T w = v 
		$$
		da cui si può dire che:
		$$ 
			c^T\left(\lambda z + (1 - \lambda)w\right) = \lambda c^T z + (1 - \lambda) c^T w = \lambda v + (1 - \lambda) v = v 
		$$
		Ovvero ogni punto sul segmento porta la funzione obiettiva a massimo assoluto, quindi è soluzione ottimale.
	\item Infine, notiamo che il poliedro della regione ammissible di un problema LP può essere vuoto, ergo $P = \emptyset$.
		In questo caso, si ha che $ \max(c^T \cdot x) = -\infty $ e $ \min(c^T \cdot x) = \infty $. 
		Un poliedro vuoto significa che i vincoli stessi vanno modificati. 
		Questo solitamente si fa risovendo una versione semplificata del problema LP.
\end{enumerate}

Si può fare un'altro esempio per sottolineare l'importanza del punto di massimo (o minimo), e non quel massimo (o minimo).
Finché nella funzione obiettivo i coefficienti compargono nello stesso rapporto (ergo finché si scelgono vettori $c$ linearmente dipendenti), il punto di massimo (o minimo) non cambia, per via della linearità (e si presume omogeneità) della funzione obiettiva stessa.
Sarà solo il massimo (o minimo) a variare di un rapporto pari a quello di cui variano i coefficienti.

\subsection{Gradiente e linee di isocosto}
Si può dimostrare il seguente teorema:
\begin{theorem}{Gradiente della funzione obiettivo}
	Il gradiente di una funzione obiettivo definita come $ f(x) = c^T \cdot x $ sulla base di un qualche vettore $c$ è in ogni punto il vettore $c$ stesso.
\end{theorem}
Da questo gradiente si possono ricavare le cosiddette linee di isocosto (in dimensioni $>2$ sarebbero superfici), cioè linee a valore costante della funzione obiettivo.
\begin{definition}{Linea di isocosto}
	Si definisce linea di isocosto di una funzione obiettivo con vettore $c$ una retta (o superficie):	
	$$ f(x) = c^T \cdot x = k $$
	per un qualsiasi $k$ costante.
\end{definition}

\subsection{Cono di competenza}
Dovrebbe essere chiaro adesso che i punti di soluzione ottima stanno tutti su un segmento o su un punto della frontiera.
Nel caso si abbia un vettore gradiente perpendicolare ad un segmento della frontiera, quel segmento sarà soluzione ottima. In caso contrario, spostandoci a destra avremo l'estremo destro del segmento, e spostandoci a sinistra viceversa, finché non si diventerà perpendicolari a qualche altro segmento di frontiera.

Il cono (in $R^2$, angolo) di valori possibili del gradiente che rendono un punto ottimale prende il nome di \textbf{cono di competenza}.
\begin{definition}{Cono di competenza}
	Il cono di competenza di un punto $x^*$ è il cono, ovvero l'insieme di vettori gradiente, tale per cui il punto $x^*$ conserva l'ottimalità sulla funzione obiettivo coi vincoli imposti.
\end{definition}

Vedremo in seguito l'importanza di una nozione di cono per i problemi LP.

\end{document}


\documentclass[a4paper,11pt]{article}
\usepackage[a4paper, margin=8em]{geometry}

% usa i pacchetti per la scrittura in italiano
\usepackage[french,italian]{babel}
\usepackage[T1]{fontenc}
\usepackage[utf8]{inputenc}
\frenchspacing 

% usa i pacchetti per la formattazione matematica
\usepackage{amsmath, amssymb, amsthm, amsfonts}

% usa altri pacchetti
\usepackage{gensymb}
\usepackage{hyperref}
\usepackage{standalone}

% imposta il titolo
\title{Appunti /home/luca/Desktop/Uni/appunti/Ricerca Operativa}
\author{Luca Seggiani}
\date{2024}

% disegni
\usepackage{pgfplots}
\pgfplotsset{width=10cm,compat=1.9}

% imposta lo stile
% usa helvetica
\usepackage[scaled]{helvet}
% usa palatino
\usepackage{palatino}
% usa un font monospazio guardabile
\usepackage{lmodern}

\renewcommand{\rmdefault}{ppl}
\renewcommand{\sfdefault}{phv}
\renewcommand{\ttdefault}{lmtt}

% disponi il titolo
\makeatletter
\renewcommand{\maketitle} {
	\begin{center} 
		\begin{minipage}[t]{.8\textwidth}
			\textsf{\huge\bfseries \@title} 
		\end{minipage}%
		\begin{minipage}[t]{.2\textwidth}
			\raggedleft \vspace{-1.65em}
			\textsf{\small \@author} \vfill
			\textsf{\small \@date}
		\end{minipage}
		\par
	\end{center}

	\thispagestyle{empty}
	\pagestyle{fancy}
}
\makeatother

% disponi teoremi
\usepackage{tcolorbox}
\newtcolorbox[auto counter, number within=section]{theorem}[2][]{%
	colback=blue!10, 
	colframe=blue!40!black, 
	sharp corners=northwest,
	fonttitle=\sffamily\bfseries, 
	title=Teorema~\thetcbcounter: #2, 
	#1
}

% disponi definizioni
\newtcolorbox[auto counter, number within=section]{definition}[2][]{%
	colback=red!10,
	colframe=red!40!black,
	sharp corners=northwest,
	fonttitle=\sffamily\bfseries,
	title=Definizione~\thetcbcounter: #2,
	#1
}

% disponi problemi
\newtcolorbox[auto counter, number within=section]{problem}[2][]{%
	colback=green!10,
	colframe=green!40!black,
	sharp corners=northwest,
	fonttitle=\sffamily\bfseries,
	title=Problema~\thetcbcounter: #2,
	#1
}

% disponi codice
\usepackage{listings}
\usepackage[table]{xcolor}

\lstdefinestyle{codestyle}{
		backgroundcolor=\color{black!5}, 
		commentstyle=\color{codegreen},
		keywordstyle=\bfseries\color{magenta},
		numberstyle=\sffamily\tiny\color{black!60},
		stringstyle=\color{green!50!black},
		basicstyle=\ttfamily\footnotesize,
		breakatwhitespace=false,         
		breaklines=true,                 
		captionpos=b,                    
		keepspaces=true,                 
		numbers=left,                    
		numbersep=5pt,                  
		showspaces=false,                
		showstringspaces=false,
		showtabs=false,                  
		tabsize=2
}

\lstdefinestyle{shellstyle}{
		backgroundcolor=\color{black!5}, 
		basicstyle=\ttfamily\footnotesize\color{black}, 
		commentstyle=\color{black}, 
		keywordstyle=\color{black},
		numberstyle=\color{black!5},
		stringstyle=\color{black}, 
		showspaces=false,
		showstringspaces=false, 
		showtabs=false, 
		tabsize=2, 
		numbers=none, 
		breaklines=true
}

\lstdefinelanguage{javascript}{
	keywords={typeof, new, true, false, catch, function, return, null, catch, switch, var, if, in, while, do, else, case, break},
	keywordstyle=\color{blue}\bfseries,
	ndkeywords={class, export, boolean, throw, implements, import, this},
	ndkeywordstyle=\color{darkgray}\bfseries,
	identifierstyle=\color{black},
	sensitive=false,
	comment=[l]{//},
	morecomment=[s]{/*}{*/},
	commentstyle=\color{purple}\ttfamily,
	stringstyle=\color{red}\ttfamily,
	morestring=[b]',
	morestring=[b]"
}

% disponi sezioni
\usepackage{titlesec}

\titleformat{\section}
	{\sffamily\Large\bfseries} 
	{\thesection}{1em}{} 
\titleformat{\subsection}
	{\sffamily\large\bfseries}   
	{\thesubsection}{1em}{} 
\titleformat{\subsubsection}
	{\sffamily\normalsize\bfseries} 
	{\thesubsubsection}{1em}{}

% disponi alberi
\usepackage{forest}

\forestset{
	rectstyle/.style={
		for tree={rectangle,draw,font=\large\sffamily}
	},
	roundstyle/.style={
		for tree={circle,draw,font=\large}
	}
}

% disponi algoritmi
\usepackage{algorithm}
\usepackage{algorithmic}
\makeatletter
\renewcommand{\ALG@name}{Algoritmo}
\makeatother

% disponi numeri di pagina
\usepackage{fancyhdr}
\fancyhf{} 
\fancyfoot[L]{\sffamily{\thepage}}

\makeatletter
\fancyhead[L]{\raisebox{1ex}[0pt][0pt]{\sffamily{\@title \ \@date}}} 
\fancyhead[R]{\raisebox{1ex}[0pt][0pt]{\sffamily{\@author}}}
\makeatother

\begin{document}

% sezione (data)
\section{Lezione del 25-09-24}

% stili pagina
\thispagestyle{empty}
\pagestyle{fancy}

% testo
\subsection{Assegnamento di costo minimo}
Vediamo un problema:
\begin{problem}{Assegnamento}
	Quattro agenzie possono occuparsi di 4 progetti.
	Ogni agenzia presenta il costo stimato per la realizzazione di ogni progetto, in migliaia di euro.
	In forma tabulare, si riportano i valori:

	\center \rowcolors{2}{green!10}{green!40!black!20}
	\begin{tabular} { | c | c | c | c | c | }
		\hline
		& \bfseries Agenzia 1 & \bfseries Agenzia 2 & \bfseries Agenzia 3 & \bfseries Agenzia 4 \\
		\hline 
		\bfseries Progetto 1 & 20 & 17 & 16 & 14 \\
		\bfseries Progetto 2 & 22 & 16 & 19 & 15 \\
		\bfseries Progetto 3 & 21 & 17 & 15 & 23 \\ 
		\bfseries Progetto 4 & 19 & 18 & 14 & 24 \\
		\hline
	\end{tabular}

	\par\bigskip
	
	Vogliamo capire quale agenzia deve occuparsi di quale progetto per minimizzare i costi.

\end{problem}

Con $n$ agenzie e progetti ci sono $n!$ possibili combinazioni, ergo dobbiamo trovare un algoritmo più efficiente. 
Applicando il modello studiato finora, abbiamo la matrice dei costi $c$:

$$
\begin{pmatrix}
 20 & 17 & 16 & 14 \\
 22 & 16 & 19 & 15 \\
 21 & 17 & 15 & 23 \\
 19 & 18 & 14 & 24 \\
\end{pmatrix}
$$

che possiamo portare a:
$$ c: ( -18, +18 + ... + 24 ) $$
come linearizzazione lessicografica della tabella sopra riportata (notare che sarebbe un vettore colonna).

Adesso dobbiamo solo trovare un metodo per esplicitare i vincoli del problema:
\begin{itemize}
	\item Ogni agenzia può occuparsi solo di un progetto;
	\item Ogni progetto richiede solo un'agenzia.
\end{itemize}

Possiamo rappresentare la corrispondenza fra elementi come un vettore, e quindi riportarne una matrice d'adiacenza.
Assumendo di appaiare elementi con lo stesso numero, avremo:
$$
\begin{pmatrix}
	1 & 0 & 0 & 0 \\ 
	0 & 1 & 0 & 0 \\ 
	0 & 0 & 1 & 0 \\ 
	0 & 0 & 0 & 1 \\ 
\end{pmatrix}
$$

La caratteristica di questa matrice, chiamiamola $x$, e che ogni elemento $x_{ij}$ è:
$$
x_{ij} = 
	\begin{cases}
		0 \\ 1	
	\end{cases}
$$

Decidiamo di trattare la $x$ come un vettore linearizzato lessicograficamente dalla matrice, proprio come avevamo fatto per il vettore costo.
Per una matrice di adiacenza $n \times n$, di $n$ elementi in ogni categoria, questo vettore ha dimensione $n^2$. 
Questo semplifica la notazione del problema, e sopratutto della matrice $A$, che sarebbe lasciando $x$ matrice effettivamente un tensore.

Si dimostra quindi facilmente che i vincoli riportati prima possono quindi esprimersi come:
\[
	\begin{cases}
		x_{11} + x_{12} + x_{13} + x_{14} = 1	\\
		x_{21} + x_{22} + x_{23} + x_{24} = 1 \\ 
		x_{31} + x_{32} + x_{33} + x_{34} = 1 \\ 
		x_{41} + x_{42} + x_{43} + x_{44} = 1 \\ 
	\end{cases}
\]

per il primo punto, e:
\[
	\begin{cases}
		x_{11} + x_{21} + x_{31} + x_{41} = 1	\\
		x_{12} + x_{22} + x_{32} + x_{42} = 1 \\ 
		x_{13} + x_{23} + x_{33} + x_{43} = 1 \\ 
		x_{14} + x_{24} + x_{34} + x_{44} = 1 \\ 
	\end{cases}
\]

per il secondo.
Imponendo la positività, si hanno quindi le matrici $A$ e $b$:

$$
A: 
\begin{pmatrix}
	1 & 1 & 1 & 1 & 0... & &  & & & ...0 \\
	0... & & ...0 & 1 & 1 & 1 & 1 & 0... & & ...0 \\
	... \\
	0... & &  & & & ...0 & 1 & 1 & 1 & 1 \\
	...\\
	-1 & 0... & & & & & & & & ...0 \\
	...\\
	0... & & & & & & & & & -1 \\
\end{pmatrix}, \quad 
b: ( 1, ..., 1, 0, ..., 0)
$$

Si nota che il numero di vincoli necessari per $n$ elementi è $2n + n^2$.

\subsubsection{Assegnamento cooperativo e non cooperativo}
A questo punto conviene fare una distinzione.
Abbiamo definito finora il modello:

\[
	\begin{cases}
		\min{c^T \cdot x} \\
		x_{11} + x_{12} + x_{13} + x_{14} = 1	\\
		...\\
		x_{41} + x_{42} + x_{43} + x_{44} = 1 \\
		x_{11} + x_{21} + x_{31} + x_{41} = 1	\\
		...\\
		x_{14} + x_{24} + x_{34} + x_{44} = 1 \\ 
 
	\end{cases}
\]

che così scritto non nega la possibilità di $x$ con componenti reali.
Nell'esempio ciò significa sono ammesse soluzioni dove più agenzie danno contributi reali ai progetti, che possiamo semanticamente interpretare come condividere il carico di lavoro, pur rispettando i vincoli imposti.
Decidiamo che questo è corretto se si parla di un problema di \textbf{assegnamento cooperativo}.
Visto che il problema posto non era di questo tipo, ma era di \textbf{assegnamento non cooperativo}, si introduce un'ulteriore vincolo:

$$
	x \in \mathbb{Z}^n
$$

Adesso ogni azienda darà un contributo intero al suo progetto, ergo coi vincoli imposti prima, ogni azienda sarà unica nel dirigere un solo progetto.

Più formalmente, possiamo dire che il passaggio ad assegnamento cooperativo comporta un \textbf{rilassamento} dei vincoli del problema.
Ovvero, in generale, se un problema non cooperativo ha minimo ottimale $nc$, il suo associato cooperativo avrà minimo ottimale $c$ con:

$$  c \leq nc $$

\subsubsection{Forma primale standard}
Portiamo quindi questo problema in una forma primale simile a quella vista per altri problemi LP, concesso il vincolo $x \in \mathbb{Z}^n$.

Finora avevamo usato le trasformazioni equivalenti per problemi LP:

\begin{enumerate}
	\item Trasformazione delle disuguaglianze: $ \geq \ \leftrightarrow \ \leq $
	\item Trasformazione delle uguaglianze:
		$$
			f(x) = c \ \rightarrow \
		\begin{cases}
			f(x) \leq c \\ 
			-f(x) \leq -c
		\end{cases}
		$$
	\item Trasformazione minimo / massimo: 
		$$
		\max{f} = -\min{f} \ \text{e sopratutto} \ 
		\bar{x} \in \mathrm{argmax}(f) \Leftrightarrow \bar{x} \in \mathrm{argmin}(-f)
		$$
\end{enumerate}

Possiamo applicare queste trasformazioni al modello, in particolare la (2), che porta il numero di vincoli a $4n + n^2$.

\subsection{Introduzione di surplus}
Vediamo un ulteriore tecnica per trasformare problemi LP: si può portare una disequazione del tipo:
$$ a_1x_1 + a_2x_2 + ... + a_nx_n \leq b $$
in un uguaglianza introducendo una variabile ausiliaria $s$:
$$ a_1x_1 + a_2x_2 + ... + a_nx_n + s = b $$
$s$ prende il nome di \textbf{slack}, in italiano scarto, o \textit{surplus}.
\end{document}


\documentclass[a4paper,11pt]{article}
\usepackage[a4paper, margin=8em]{geometry}

% usa i pacchetti per la scrittura in italiano
\usepackage[french,italian]{babel}
\usepackage[T1]{fontenc}
\usepackage[utf8]{inputenc}
\frenchspacing 

% usa i pacchetti per la formattazione matematica
\usepackage{amsmath, amssymb, amsthm, amsfonts}

% usa altri pacchetti
\usepackage{gensymb}
\usepackage{hyperref}
\usepackage{standalone}

% imposta il titolo
\title{Appunti Ricerca Operativa}
\author{Luca Seggiani}
\date{2024}

% disegni
\usepackage{pgfplots}
\pgfplotsset{width=10cm,compat=1.9}

% imposta lo stile
% usa helvetica
\usepackage[scaled]{helvet}
% usa palatino
\usepackage{palatino}
% usa un font monospazio guardabile
\usepackage{lmodern}

\renewcommand{\rmdefault}{ppl}
\renewcommand{\sfdefault}{phv}
\renewcommand{\ttdefault}{lmtt}

% disponi il titolo
\makeatletter
\renewcommand{\maketitle} {
	\begin{center} 
		\begin{minipage}[t]{.8\textwidth}
			\textsf{\huge\bfseries \@title} 
		\end{minipage}%
		\begin{minipage}[t]{.2\textwidth}
			\raggedleft \vspace{-1.65em}
			\textsf{\small \@author} \vfill
			\textsf{\small \@date}
		\end{minipage}
		\par
	\end{center}

	\thispagestyle{empty}
	\pagestyle{fancy}
}
\makeatother

% disponi teoremi
\usepackage{tcolorbox}
\newtcolorbox[auto counter, number within=section]{theorem}[2][]{%
	colback=blue!10, 
	colframe=blue!40!black, 
	sharp corners=northwest,
	fonttitle=\sffamily\bfseries, 
	title=Teorema~\thetcbcounter: #2, 
	#1
}

% disponi definizioni
\newtcolorbox[auto counter, number within=section]{definition}[2][]{%
	colback=red!10,
	colframe=red!40!black,
	sharp corners=northwest,
	fonttitle=\sffamily\bfseries,
	title=Definizione~\thetcbcounter: #2,
	#1
}

% disponi problemi
\newtcolorbox[auto counter, number within=section]{problem}[2][]{%
	colback=green!10,
	colframe=green!40!black,
	sharp corners=northwest,
	fonttitle=\sffamily\bfseries,
	title=Problema~\thetcbcounter: #2,
	#1
}

% disponi codice
\usepackage{listings}
\usepackage[table]{xcolor}

\lstdefinestyle{codestyle}{
		backgroundcolor=\color{black!5}, 
		commentstyle=\color{codegreen},
		keywordstyle=\bfseries\color{magenta},
		numberstyle=\sffamily\tiny\color{black!60},
		stringstyle=\color{green!50!black},
		basicstyle=\ttfamily\footnotesize,
		breakatwhitespace=false,         
		breaklines=true,                 
		captionpos=b,                    
		keepspaces=true,                 
		numbers=left,                    
		numbersep=5pt,                  
		showspaces=false,                
		showstringspaces=false,
		showtabs=false,                  
		tabsize=2
}

\lstdefinestyle{shellstyle}{
		backgroundcolor=\color{black!5}, 
		basicstyle=\ttfamily\footnotesize\color{black}, 
		commentstyle=\color{black}, 
		keywordstyle=\color{black},
		numberstyle=\color{black!5},
		stringstyle=\color{black}, 
		showspaces=false,
		showstringspaces=false, 
		showtabs=false, 
		tabsize=2, 
		numbers=none, 
		breaklines=true
}

\lstdefinelanguage{javascript}{
	keywords={typeof, new, true, false, catch, function, return, null, catch, switch, var, if, in, while, do, else, case, break},
	keywordstyle=\color{blue}\bfseries,
	ndkeywords={class, export, boolean, throw, implements, import, this},
	ndkeywordstyle=\color{darkgray}\bfseries,
	identifierstyle=\color{black},
	sensitive=false,
	comment=[l]{//},
	morecomment=[s]{/*}{*/},
	commentstyle=\color{purple}\ttfamily,
	stringstyle=\color{red}\ttfamily,
	morestring=[b]',
	morestring=[b]"
}

% disponi sezioni
\usepackage{titlesec}

\titleformat{\section}
	{\sffamily\Large\bfseries} 
	{\thesection}{1em}{} 
\titleformat{\subsection}
	{\sffamily\large\bfseries}   
	{\thesubsection}{1em}{} 
\titleformat{\subsubsection}
	{\sffamily\normalsize\bfseries} 
	{\thesubsubsection}{1em}{}

% disponi alberi
\usepackage{forest}

\forestset{
	rectstyle/.style={
		for tree={rectangle,draw,font=\large\sffamily}
	},
	roundstyle/.style={
		for tree={circle,draw,font=\large}
	}
}

% disponi algoritmi
\usepackage{algorithm}
\usepackage{algorithmic}
\makeatletter
\renewcommand{\ALG@name}{Algoritmo}
\makeatother

% disponi numeri di pagina
\usepackage{fancyhdr}
\fancyhf{} 
\fancyfoot[L]{\sffamily{\thepage}}

\makeatletter
\fancyhead[L]{\raisebox{1ex}[0pt][0pt]{\sffamily{\@title \ \@date}}} 
\fancyhead[R]{\raisebox{1ex}[0pt][0pt]{\sffamily{\@author}}}
\makeatother

\begin{document}

% sezione (data)
\section{Lezione del 26-09-24}

% stili pagina
\thispagestyle{empty}
\pagestyle{fancy}

% testo
\subsection{Geometria dei poliedri}
Introduciamo progressivamente i tipi di \textbf{combinazione} che ci sono utili nello studio dei problemi di programmazione lineare.

\subsubsection{Combinazioni lineari}
\begin{definition}{Combinazione lineare}
	Dati $ x_1, x_2, ..., x_k \in \mathbb{R}^n $ punti, $y$ si dice \textbf{combinazione lineare} di $ x_1, x_2, ..., x_k $ se:
	$$
	\exists \lambda_i \quad (i = 1, ..., k) \quad \text{t.c.} \quad y = \sum_{i=1}^k \lambda_i x_i 
	$$
\end{definition}

Le combinazioni lineari sono utili per esprimere la funzione obiettiva sulla base dei vettori costo, ma non bastano a trovarne una soluzione ottimale.

\subsubsection{Combinazioni convesse}
Si introduce quindi il concetto di:
\begin{definition}{Combinazione convessa}
	Dati $ x_1, x_2, ..., x_k \in \mathbb{R}^n $ punti, $y$ si dice \textbf{combinazione convessa} di $ x_1, x_2, ..., x_k $ se:
	$$
	\exists \lambda_i \in [0, 1] \quad (i = 1, ..., k), \quad \sum_{i=1}^k \lambda_i = 1 \quad \text{t.c.} \quad y = \sum_{i=1}^k \lambda_i x_i 
	$$
\end{definition}

Possiamo dare un esempio di cos'è la combinazione convessa di due punti in $\mathbb{R}^2$.
Posti $x_1$ e $x_2$, si ha:
$$
\lambda_1 + \lambda_2 = 1 \Rightarrow \lambda_2 = (1 - \lambda_1), \quad y = \lambda x_1 + (1 - \lambda ) x_2, \quad \lambda \in [0, 1]
$$
che riconosciamo essere l'equazione di un segmento $\bar{x_1x_2}$ (primo grafico).

Possiamo provare con tre punti: si avrà:
$$
y = \lambda_1 x_1 + \lambda_2 x_2 + \lambda_3 x_3, \quad \lambda_1 + \lambda_2 + \lambda_3 = 1, \quad \lambda_i \in [0, 1]
$$
che si riconduce all'equazione del triangolo di vertici $x_1$, $x_2$, $x_3$ (secondo grafico).
\par\medskip

\begin{minipage}{0.45\textwidth}
    % Segment on a graph
	\begin{tikzpicture}[scale=0.75]
    \begin{axis}[
        axis lines = center,
        xlabel = $x$, ylabel = $y$,
        xmin= -0.8, xmax=4.8, ymin=-0.8, ymax=4.8,
        grid = minor,
				title = {Combinazione di $x_1$ e $x_2$}
    ]
        \addplot[thick] coordinates {(1,2) (4,2)};
        \addplot[only marks, mark=*] coordinates {(1,2)} node[anchor=north] {$x_1$};
        \addplot[only marks, mark=*] coordinates {(4,2)} node[anchor=north] {$x_2$};
    \end{axis}
    \end{tikzpicture}
\end{minipage}
\hfill
\begin{minipage}{0.45\textwidth}
    % Triangle on a graph
    \begin{tikzpicture}[scale=0.75]
    \begin{axis}[
        axis lines = center,
        xlabel = $x$, ylabel = $y$,
        xmin= -0.8, xmax=4.8, ymin=-0.8, ymax=4.8,
        grid = minor,
				title = {Combinazione di $x_1$, $x_2$ e $x_3$}
    ]
        \addplot[thick] coordinates {(1,1) (4,1) (2,3) (1,1)};
        \addplot[only marks, mark=*] coordinates {(1,1)} node[anchor=north] {$x_1$};
        \addplot[only marks, mark=*] coordinates {(4,1)} node[anchor=north] {$x_2$};
        \addplot[only marks, mark=*] coordinates {(2,3)} node[anchor=south] {$x_3$};
    \end{axis}
    \end{tikzpicture}
\end{minipage}
\par\medskip

Dai grafici si nota come una combinazione convessa descrive una parte di spazio, che si può definire:
\begin{definition}{Involucro convesso}
	L'involucro convesso $\mathrm{conv}(K)$ di un'insieme di punti $K = \{x_1, x_2, ..., x_n\} \in \mathbb{R}^n$ è definito come il luogo di tutte le loro combinazioni convesse.
\end{definition}

Si nota che l'involucro convesso negli esempi precedenti è effettivamente un poliedro convesso che contiene tutti i punti che lo formano.
Si può infatti dire:
\begin{theorem}{Minimalità dell'involucro convesso}
	L'insieme $\mathrm{conv}(K)$ di tutte le combinazioni convesse di $n$ punti è il più piccolo poliedro convesso che li contiene tutti.
\end{theorem}

Si noti che non è detto che a $n$ punti corrisponda uno poligono di di $n$ vertici.
Può infatti accadere che uno dei punti è già parte dell'involucro convesso. 

Le combinazioni convesse ci permettono di descrivere parte delle regioni ammissibili (poliedri) dei problemi di programmazione lineare, ma restano ancora in sospeso problemi che ammettono regioni illimitate.
Per descrivere tali regioni, si introduce un altro tipo di combinazione.

\subsubsection{Combinazioni coniche}
\begin{definition}{Combinazione conica}
	Dati $ x_1, x_2, ..., x_k \in \mathbb{R}^n $ punti, $y$ si dice \textbf{combinazione conica} di $ x_1, x_2, ..., x_k $ se:
	$$
	\exists \lambda_i \geq 0 \quad (i = 1, ..., k) \quad \text{t.c.} \quad y = \sum_{i=1}^k \lambda_i x_i 
	$$
\end{definition}

La combinazione conica di più punti non è più il poliedro convesso che li contiene, ma il cono con vertice nell'origine, convesso o meno, che li contiene, definito come:
\begin{definition}{Involucro conico}
	L'involucro conico $\mathrm{cono}(K)$ di un'insieme di punti $K = \{x_1, x_2, ..., x_n\} \in \mathbb{R}^n$ è definito come il luogo di tutte le loro combinazioni coniche.
\end{definition}

Questo cono si estende fino all'infinito ($\lambda_i \geq 0$) nelle direzioni dei vettori che lo formano.
Il concetto è simile a quello di spazio somma, ma con la differenza che non si va ovunque nello span dei due vettori, ma si seguono le semirette che essi conducono.

\subsection{Poliedri}
Abbiamo definito un poliedro come la regione definita da un sistema di disequazioni lineari, o geometricamente come l'intersezione di un numero finito di semipiani chiusi in $\mathbb{R}^n$.
Un poliedro che è anche cono si chiama cono poliedrico. Si dimostra che:
\begin{theorem}{Cono poliedrico}
	Se $P$ è un cono poliedrico allora:
	$$
	\exists Q \quad \text{t.c.} \quad P = \{ x \in \mathbb{R}^n : Qx \leq 0 \}
	$$
	con $Q$ matrice.
\end{theorem}

Senza dimostrazioni, questo è chiaro dal fatto che le disequazioni che compongono il poliedrico sono omogenee (hanno frontiere che passano dall'origine).
Si definiscono poi i \textbf{vertici} del poliedro:
\begin{definition}{Vertice}
	Un vertice di un poliedro è un punto che non si può esprimere come combinazione convessa propria di altri punti del poliedro.
	Si indica l'insieme dei vettori di un poliedro $P$ come $\mathrm{vert}(P)$.
\end{definition}
Notiamo che i vertici di un poliedro limitato corrispondono ai punti che formano la combinazione convessa equivalente al poliedro.
Per poliedri illimitati, introduciamo invece:
\begin{definition}{Direzione di recessione}
	Un vettore $d$ è la direzione di recessione di un poliedro se:
	$$
	x + \lambda d \in P \quad \forall x \in P, \quad \forall \lambda \geq 0
	$$
	Si indica come $\mathrm{rec}(P)$ l'insieme delle direzioni di recessione di un poliedro.
\end{definition}

Chiaramente, per ogni poliedro $P$, $0 \in \mathrm{rec}(P)$ e per i poliedri limitati, $\mathrm{rec}(P) = \{0\}$.
Notiamo che le direzioni di recessione determinano i vettori del cono che coincide (almeno a distanze abbastanza grandi dall'origine) con i poliedri illimitati.
Più propriamente, si può dire che un cono poliedrico è l'involucro conico di un insieme finito dei suoi punti (basta prendere gli "estremi").

Definiamo poi lo spazio di linealità:
\begin{definition}{Spazio di linealità}
	Lo spazio di linealità di un poliedro illimtato $P$ è il più piccolo sottospazio contenuto interamente in $P$.
\end{definition}
La base di uno spazio di linealità è un vettore $d$ tale che:
$$
d \in \mathrm{rec}(P), \quad - d \in \mathrm{rec}(P)
$$
ovvero un vettore che è contenuto sia positivo che negativo nelle direzioni di recessione del poliedro.

Questa distinzione è importante in quanto non si può dimostrare completamente il prossimo problemi su poliedri con spazio di linealità $\neq {0}$.

\subsubsection{Teorema di rappresentazione dei poliedri}
Gli strumenti che abbiamo stabilito finora ci permettono di dimostrare un'importante risultato, noto come \textbf{teorema di rappresentazione dei poliedri}, o teorema di Minkowski-Weyl.

\begin{theorem}{Rappresentazione dei poliedri}
	Dato un poliedro $P$ definito come $P = \{ x \in \mathbb{R}^n : Ax \leq b \}$, si ha:
	$$
	\exists V = \{ v_1, ..., v_k \} \in \mathrm{vert}(P), \quad \exists E = {e_1, ..., e_p} \in \mathrm{rec}(P) \quad \text{t.c.} \quad P = \mathrm{conv}(V) + \mathrm{cono}(E)
	$$
\end{theorem}

Questo significa che è possibile rappresentare qualsiasi poliedro attraverso i suoi vertici, e le direzioni in cui si estende all'infinito (ergo le sue direzioni di recessione).

La somma in $P = \mathrm{conv}(V) + \mathrm{cono}(E)$ si riferisce alla somma vettoriale fra tutti i possibili punti di $\mathrm{conv}(V)$ e $\mathrm{conv}(E)$, come quella studiata sui sottospazi vettoriali (anche se nessuno dei due insiemi è un sottospazio vettoriale).
Per un dato insieme $\mathrm{conv}(V)$, quindi, l'aggiunta di $\mathrm{cono}(E)$ rappresenta la "proiezione" di tale insieme nelle direzioni di recessione indicate dal cono.

Più propriamente, posto $\mathrm{lineal}(P) = {0}$, si ha:

\begin{theorem}{Rappresentazione dei poliedri non lineali}
	Dato un poliedro $P$ definito come $P = \{ x \in \mathbb{R}^n : Ax \leq b \}$, tale che $\mathrm{lineal}(P)$, si ha:
	$$
	P = \mathrm{conv}(\mathrm{vert}(P)) + \mathrm{rec}(P)
	$$
\end{theorem}
la limitazione di linealità è necessaria in quanto un poliedro lineale potrebbe non essere rappresentato, nelle sue dimensioni infinite, dal semplice insieme dei suoi vettori.
In verità, è possibile dimostrare che:

\begin{theorem}{Linealità di poliedri}
	Per ogni poliedro $P$ non vuoto si ha:
	$$
		\mathrm{vert}(P) \neq \emptyset \Leftrightarrow \mathrm{lineal}(P) = {0}
	$$
\end{theorem}
ergo applicando lo scorso teorema potremmo provare a rappresentare un poliedro attraverso un'insieme di vettori vuoto.

Per i poliedri che otteniamo dai problemi di programmazione lineare, però, abbiamo i corollari:
\begin{itemize}
	\item Un poliedro limitato è l'involucro convesso dei suoi vertici;
	\item Se il poliedro ha vincoli di positività sulle sue variabili, allora non è lineale, ergo si applica il teorema di rappresentazione.
		Questo è il tipo di poliedri a cui siamo abituati.
\end{itemize}

\subsection{Teorema fondamentale della PL}
Quanto riportare finora sulla geometria dei poliedri può essere usato per dimostrare il seguente teorema:
\begin{theorem}{Teorema fondamentale della PL}
	Sia dato un poliedro $P$ rappresentato come:
	$$ 
	P = \mathrm{conv}(V) + \mathrm{cono}(E), \quad V = \{ v_1, ..., v_k \}, \quad E = \{ e_1, ..., e_p \}
	$$
	Se il problema $\mathcal{P}$ con regione ammissibile $P$ ha valore ottimo finito, allora esiste $s \in \{ 1, ..., k \}$ tale che $v_k$ è soluzione ottima di $\mathcal{P}$.
\end{theorem}

In sostanza, se un problema LP ha soluzione, essa si trova su uno dei vertici del poliedro della regione ammissibile.

\par\medskip
\noindent
\textbf{\textsf{Dimostrazione}} \\
Sia dato un problema LP $\mathcal{P}$ in forma primale standard, ergo posto come:
\[
	\begin{cases}
		\max{c^T \cdot x} \\ 
		Ax \leq b	
	\end{cases}
\]
ergo con regione ammissibile rappresentata da un poliedro $P$.

Dal teorema della rappresentazione, possiamo esprimere il poliedro come:
$$
	P = \mathrm{conv}(V) + \mathrm{cono}(E), \quad V = \{ v_1, ..., v_k \}, \quad E = \{ e_1, ..., e_p \}
$$

Combiniamo le due equazioni, ergo esprimiamo prima il punto $\bar{x}$ generico del poliedro applicando le definizioni di involucro convesso e conico:
$$
\bar{x} \in P : P = \mathrm{conv}(V) + \mathrm{cono}(E), \quad \bar{x} = \sum_{i=1}^k \lambda_i v_i + \sum_{j=i}^p \mu_j e_j
$$
ed esprimiamo quindi la funzione obiettivo come il prodotto scalare fra il vettore costo e il punto $\bar{x}$ del poliedro:
$$
c^T \cdot \bar{x} = c^T \cdot \left( \sum_{i=1}^k \lambda_i v_i + \sum_{j=i}^p \mu_j e_j \right) = \sum_{i=1}^k \lambda_i c^T v_i + \sum_{j=i}^p \mu_j c^T e_j
$$

A questo punto conviene chiarire su cosa significa che il problema ha valore ottimo finito.
Il secondo termine è la combinazione conica della sommatoria dei vettori di recessione scalati dal vettore costo.
Se almeno uno dei $c^T e_j > 0$, si avrà che portando $\mu_j \rightarrow +\infty$ la funzione avrà massimo $= \infty$.
Geometricamente, questo significa che esiste una direzione illimitata del poliedro dove i vettori costo permettono alla funzione di crescere all'infinito.

Dunque sarà vero che $c^T e_j \leq 0 \quad \forall j \in \{ 1, ..., p \}$ se vogliamo che la funzione abbia valore ottimo finito.

Possiamo quindi usare questa ipotesi per dire:
$$
c^T \cdot \bar{x} = \sum_{i=1}^k \lambda_i c^T v_i + \sum_{j=i}^p \mu_j c^T e_j \leq \sum_{i=1}^k \lambda_i c^T v_i \leq \sum_{i=1}^k \max_{1 \leq i \leq p} \left(\lambda_i c^T v_i\right) 
$$

$$
= \left( \max_{1 \leq i \leq p} c^T v_i\right) \sum_{i=1}^k \lambda_i = \max_{1\leq i \leq p} c^Tv_i = c^T v_k
$$

E quindi $\max_{x \in P} c^T \cdot x \leq c^T v_k$.
A questo punto, visto che $\bar{x}$ è effettivamente un punto della regione ammissibile, sarà vero che:
$$
c^Tv_k \leq \max_{x \in P} c^T \cdot x
$$

E dunque $\max_{x \in P} c^T \cdot x = c^T v_k$, C.V.D.

\end{document}

\end{document}