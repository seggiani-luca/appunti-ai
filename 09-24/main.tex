\documentclass[a4paper,11pt]{article}
\usepackage[a4paper, margin=8em]{geometry}

% usa i pacchetti per la scrittura in italiano
\usepackage[french,italian]{babel}
\usepackage[T1]{fontenc}
\usepackage[utf8]{inputenc}
\frenchspacing 

% usa i pacchetti per la formattazione matematica
\usepackage{amsmath, amssymb, amsthm, amsfonts}

% usa altri pacchetti
\usepackage{gensymb}
\usepackage{hyperref}
\usepackage{standalone}

% imposta il titolo
\title{Appunti Ricerca Operativa}
\author{Luca Seggiani}
\date{24-09-24}

% imposta lo stile
% usa helvetica
\usepackage[scaled]{helvet}
% usa palatino
\usepackage{palatino}
% usa un font monospazio guardabile
\usepackage{lmodern}

\renewcommand{\rmdefault}{ppl}
\renewcommand{\sfdefault}{phv}
\renewcommand{\ttdefault}{lmtt}

% disponi teoremi
\usepackage{tcolorbox}
\newtcolorbox[auto counter, number within=section]{theorem}[2][]{%
	colback=blue!10, 
	colframe=blue!40!black, 
	sharp corners=northwest,
	fonttitle=\sffamily\bfseries, 
	title=Teorema~\thetcbcounter: #2, 
	#1
}

% disponi definizioni
\newtcolorbox[auto counter, number within=section]{definition}[2][]{%
	colback=red!10,
	colframe=red!40!black,
	sharp corners=northwest,
	fonttitle=\sffamily\bfseries,
	title=Definizione~\thetcbcounter: #2,
	#1
}

% disponi codice
\usepackage{listings}
\usepackage[table]{xcolor}

\lstdefinestyle{codestyle}{
		backgroundcolor=\color{black!5}, 
		commentstyle=\color{codegreen},
		keywordstyle=\bfseries\color{magenta},
		numberstyle=\sffamily\tiny\color{black!60},
		stringstyle=\color{green!50!black},
		basicstyle=\ttfamily\footnotesize,
		breakatwhitespace=false,         
		breaklines=true,                 
		captionpos=b,                    
		keepspaces=true,                 
		numbers=left,                    
		numbersep=5pt,                  
		showspaces=false,                
		showstringspaces=false,
		showtabs=false,                  
		tabsize=2
}

\lstdefinestyle{shellstyle}{
		backgroundcolor=\color{black!5}, 
		basicstyle=\ttfamily\footnotesize\color{black}, 
		commentstyle=\color{black}, 
		keywordstyle=\color{black},
		numberstyle=\color{black!5},
		stringstyle=\color{black}, 
		showspaces=false,
		showstringspaces=false, 
		showtabs=false, 
		tabsize=2, 
		numbers=none, 
		breaklines=true
}

\lstdefinelanguage{javascript}{
	keywords={typeof, new, true, false, catch, function, return, null, catch, switch, var, if, in, while, do, else, case, break},
	keywordstyle=\color{blue}\bfseries,
	ndkeywords={class, export, boolean, throw, implements, import, this},
	ndkeywordstyle=\color{darkgray}\bfseries,
	identifierstyle=\color{black},
	sensitive=false,
	comment=[l]{//},
	morecomment=[s]{/*}{*/},
	commentstyle=\color{purple}\ttfamily,
	stringstyle=\color{red}\ttfamily,
	morestring=[b]',
	morestring=[b]"
}

% disponi sezioni
\usepackage{titlesec}

\titleformat{\section}
	{\sffamily\Large\bfseries} 
	{\thesection}{1em}{} 
\titleformat{\subsection}
	{\sffamily\large\bfseries}   
	{\thesubsection}{1em}{} 
\titleformat{\subsubsection}
	{\sffamily\normalsize\bfseries} 
	{\thesubsubsection}{1em}{}

% disponi alberi
\usepackage{forest}

\forestset{
	rectstyle/.style={
		for tree={rectangle,draw,font=\large\sffamily}
	},
	roundstyle/.style={
		for tree={circle,draw,font=\large}
	}
}

% disponi algoritmi
\usepackage{algorithm}
\usepackage{algorithmic}
\makeatletter
\renewcommand{\ALG@name}{Algoritmo}
\makeatother

% disponi numeri di pagina
\usepackage{fancyhdr}
\fancyhf{} 
\fancyfoot[L]{\sffamily{\thepage}}

\makeatletter
\fancyhead[L]{\raisebox{1ex}[0pt][0pt]{\sffamily{\@title \ \@date}}} 
\fancyhead[R]{\raisebox{1ex}[0pt][0pt]{\sffamily{\@author}}}
\makeatother

\begin{document}
% sezione (data)
\section{Lezione del 24-09-24}

% stili pagina
\thispagestyle{empty}
\pagestyle{fancy}

% testo

\subsection{Forma primale standard}
Ciò che abbiamo formulato finora è un problema di programmazione lineare.
Possiamo dire che la forma:
\[
	\begin{cases}
			\max(c^T \cdot x) \\
			A x \leq b
	\end{cases}
\]
rappresenta un problema LP in forma \textbf{primale standard}, ricordando che $c$ è il vettore dei coefficienti della funzione obiettivo, $A$ la matrice dei coefficienti per ogni vincolo, e $b$ il vettore dei termini noti per ogni vincolo.

\begin{definition}{Forma primale standard}
	Un problema di programmazione lineare si dice in forma primale standard quando è espresso in forma:
	
	\[
		\begin{cases}
			\max(c^T \cdot x) \\
			Ax \leq b \\
		\end{cases}
	\]

\end{definition}
\par\smallskip
Si adotta una forma primale standard in quanto si può trasformare ogni problema LP in una forma di questo tipo.

\subsubsection{Normalizzazione di un problema LP}
Un modo per portare un problema LP qualsiasi in forma primale standard è:

\begin{enumerate}
	\item Si trasformano le disuguaglianze: $ \geq \ \leftrightarrow \ \leq $
	\item Si riscrivono le uguaglianze come coppie di diseguaglianze:
		$$
			f(x) = c \ \rightarrow \
		\begin{cases}
			f(x) \leq c \\ 
			f(x) \geq c
		\end{cases}
		$$
		da cui la (1):
		$$
			f(x) = c \ \rightarrow \
S		\begin{cases}
			f(x) \leq c \\ 
			-f(x) \leq -c
		\end{cases}
		$$
	\item Se il problema richiede il minimo, si nota che $ \max(f) = -\min(-f) $, e sopratutto:
		$$
		\bar{x} \in \mathrm{argmax}(f) \Leftrightarrow \bar{x} \in \mathrm{argmin}(-f)
		$$
		con $ \mathrm{argmax}(f) $ e $ \mathrm{argmin}(-f) $ rispettivamente gli insiemi dei punti di massimo e minimo.
		Questo significa che posso semplicemente cambiare di segno la funzione obiettivo per trovare da massimi minimi, e viceversa.
\end{enumerate}

Notiamo inoltre che, nella forma primale standard, si ha:
$$
	x \in R^n, \quad
	A \in R^{n \times m}, \quad
	b \in R^m, \quad
	c \in R^n
$$

\subsection{Proprietà generali di un problema LP}
La regione ammissibile di un problema PL si chiama \textbf{poliedro}.
Si può dare agilmente una definizione algebrica di poliedro:
\begin{definition}{Definizione algebrica di poliedro}
	Algebricamente, un poliedro è l'insieme delle soluzioni di un sistema di disequazioni lineari in $\mathbb{R}^n$ variabili:
	$$
		P = \{ x \in \mathbb{R}^n : Ax \leq b \}
	$$
\end{definition}

Questa regione in un problema LP prende il nome di regione ammissibile.

\begin{definition}{Definizione geometrica di poliedro}
	Geometricamente, un poliedro è l'intersezione di un numero finito di semispazi chiusi.
\end{definition}

I semispazi chiaramente sono lineari, e in $\mathbb{R}^2$ rappresenterebbero semipiani.
Chiusi significa che nelle disequazioni che descrivono i vincoli compargono solo $\leq$ e non $<$, ergo la regione ammissibile contiene la sua frontiera.

Possiamo dimostrare 4 proprietà dei poliedri:

\begin{enumerate}
	\item 
		Un'osservazione fondamentale è la seguente:
		\begin{theorem}{Soluzione ottimale di un problema LP}
			La soluzione ottimale di un problema LP è contenuta nella frontiera della regione ammissibile.
		\end{theorem}
		Questo si può ricavare dai teoremi di Fermat e di Weierstrass, e dalla convessità della regione ammissibile.
		Inanzitutto, si è stabilito che la soluzione ottimale non è altro che il massimo o minimo assoluto all'interno della regione ammissibile del problema.
		Il gradiente della funzione obiettiva è $\neq 0$ in ogni suo punto (funzione lineare a gradiente costante). 
		Da Fermat, i massimi e minimi hanno sempre gradiente $0$, ergo massimi o minimi locali (che esistono per Weierstrass) possono trovarsi solo sulla frontiera.
		A questo punto, possiamo imporre la convessità per asserire che quei punti di massimo o minimo sono anche globali. 

	\item 
		Prendiamo in esempio il poliedro dato da:
		\[
			\begin{cases}
				x_A > 0 \\ 
				x_B > 0
			\end{cases}
		\]
		o se vogliamo, in forma primale standard, dato dalle matrici $A$ e $b$:
		$$
		A:
		\begin{pmatrix}
			-1 & 0 \\	
			0 & -1 \\	
		\end{pmatrix}
		, \quad b:
		\begin{pmatrix}
			0 \\ 
			0
		\end{pmatrix}
		$$
		questo poliedro non è limitato nella direzione positiva, ergo può arrivare a valori di $x_A$ e $x_B$ che tendono a $+\infty$.
		Da ciò si ha che può accadere che un problema LP ammetta soluzioni $x$ tali che $x \rightarrow \pm \infty $,
		ovvero che il poliedro sia illimitato.
		In particolare, un poliedro limitato si dice \textbf{politopo}.
	\item Notiamo poi che la soluzione di un problema LP può non essere unica.
		Questo accade ad esempio quando la soluzione sta su un segmento di frontiera: a quel punto tutti i punti del segmento sono soluzione.
		Da questo segue che:
		\begin{theorem}{Unicità della soluzione ottimale di un problema LP}
			Se un problema LP ha almeno 2 soluzioni, allora ne ha infinite.
		\end{theorem}
		Ciò si può dimostrare come segue.
		Si riporta innanzitutto la notazione parametrica del segmento $\bar{zw}$, dati i due vettori di estremo $z$ e $w$:
		$$
			\lambda z + (1 - \lambda)w, \quad \lambda \in [ 0, 1 ]
		$$
		A questo punto si pone che $z$ e $w$ sono entrambi soluzioni ottime, ergo: 
		$$ 
			\max(c^T \cdot x) = c^T z = c^T w = v 
		$$
		da cui si può dire che:
		$$ 
			c^T\left(\lambda z + (1 - \lambda)w\right) = \lambda c^T z + (1 - \lambda) c^T w = \lambda v + (1 - \lambda) v = v 
		$$
		Ovvero ogni punto sul segmento porta la funzione obiettiva a massimo assoluto, quindi è soluzione ottimale.
	\item Infine, notiamo che il poliedro della regione ammissible di un problema LP può essere vuoto, ergo $P = \emptyset$.
		In questo caso, si ha che $ \max(c^T \cdot x) = -\infty $ e $ \min(c^T \cdot x) = \infty $. 
		Un poliedro vuoto significa che i vincoli stessi vanno modificati. 
		Questo solitamente si fa risovendo una versione semplificata del problema LP.
\end{enumerate}

Si può fare un'altro esempio per sottolineare l'importanza del punto di massimo (o minimo), e non quel massimo (o minimo).
Finché nella funzione obiettivo i coefficienti compargono nello stesso rapporto (ergo finché si scelgono vettori $c$ linearmente dipendenti), il punto di massimo (o minimo) non cambia, per via della linearità (e si presume omogeneità) della funzione obiettiva stessa.
Sarà solo il massimo (o minimo) a variare di un rapporto pari a quello di cui variano i coefficienti.

\subsection{Gradiente e linee di isocosto}
Si può dimostrare il seguente teorema:
\begin{theorem}{Gradiente della funzione obiettivo}
	Il gradiente di una funzione obiettivo definita come $ f(x) = c^T \cdot x $ sulla base di un qualche vettore $c$ è in ogni punto il vettore $c$ stesso.
\end{theorem}
Da questo gradiente si possono ricavare le cosiddette linee di isocosto (in dimensioni $>2$ sarebbero superfici), cioè linee a valore costante della funzione obiettivo.
\begin{definition}{Linea di isocosto}
	Si definisce linea di isocosto di una funzione obiettivo con vettore $c$ una retta (o superficie):	
	$$ f(x) = c^T \cdot x = k $$
	per un qualsiasi $k$ costante.
\end{definition}

\subsection{Cono di competenza}
Dovrebbe essere chiaro adesso che i punti di soluzione ottima stanno tutti su un segmento o su un punto della frontiera.
Nel caso si abbia un vettore gradiente perpendicolare ad un segmento della frontiera, quel segmento sarà soluzione ottima. In caso contrario, spostandoci a destra avremo l'estremo destro del segmento, e spostandoci a sinistra viceversa, finché non si diventerà perpendicolari a qualche altro segmento di frontiera.

Il cono (in $R^2$, angolo) di valori possibili del gradiente che rendono un punto ottimale prende il nome di \textbf{cono di competenza}.
\begin{definition}{Cono di competenza}
	Il cono di competenza di un punto $x^*$ è il cono, ovvero l'insieme di vettori gradiente, tale per cui il punto $x^*$ conserva l'ottimalità sulla funzione obiettivo coi vincoli imposti.
\end{definition}

Vedremo in seguito l'importanza di una nozione di cono per i problemi LP.

\end{document}
