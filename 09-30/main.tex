
\documentclass[a4paper,11pt]{article}
\usepackage[a4paper, margin=8em]{geometry}

% usa i pacchetti per la scrittura in italiano
\usepackage[french,italian]{babel}
\usepackage[T1]{fontenc}
\usepackage[utf8]{inputenc}
\frenchspacing 

% usa i pacchetti per la formattazione matematica
\usepackage{amsmath, amssymb, amsthm, amsfonts}

% usa altri pacchetti
\usepackage{gensymb}
\usepackage{hyperref}
\usepackage{standalone}

% imposta il titolo
\title{Appunti Intelligenza Artificiale}
\author{Luca Seggiani}
\date{2024}

% disegni
\usepackage{pgfplots}
\pgfplotsset{width=10cm,compat=1.9}

% imposta lo stile
% usa helvetica
\usepackage[scaled]{helvet}
% usa palatino
\usepackage{palatino}
% usa un font monospazio guardabile
\usepackage{lmodern}

\renewcommand{\rmdefault}{ppl}
\renewcommand{\sfdefault}{phv}
\renewcommand{\ttdefault}{lmtt}

% disponi il titolo
\makeatletter
\renewcommand{\maketitle} {
	\begin{center} 
		\begin{minipage}[t]{.8\textwidth}
			\textsf{\huge\bfseries \@title} 
		\end{minipage}%
		\begin{minipage}[t]{.2\textwidth}
			\raggedleft \vspace{-1.65em}
			\textsf{\small \@author} \vfill
			\textsf{\small \@date}
		\end{minipage}
		\par
	\end{center}

	\thispagestyle{empty}
	\pagestyle{fancy}
}
\makeatother

% disponi teoremi
\usepackage{tcolorbox}
\newtcolorbox[auto counter, number within=section]{theorem}[2][]{%
	colback=blue!10, 
	colframe=blue!40!black, 
	sharp corners=northwest,
	fonttitle=\sffamily\bfseries, 
	title=Teorema~\thetcbcounter: #2, 
	#1
}

% disponi definizioni
\newtcolorbox[auto counter, number within=section]{definition}[2][]{%
	colback=red!10,
	colframe=red!40!black,
	sharp corners=northwest,
	fonttitle=\sffamily\bfseries,
	title=Definizione~\thetcbcounter: #2,
	#1
}

% disponi problemi
\newtcolorbox[auto counter, number within=section]{problem}[2][]{%
	colback=green!10,
	colframe=green!40!black,
	sharp corners=northwest,
	fonttitle=\sffamily\bfseries,
	title=Problema~\thetcbcounter: #2,
	#1
}

% disponi codice
\usepackage{listings}
\usepackage[table]{xcolor}

\lstdefinestyle{codestyle}{
		backgroundcolor=\color{black!5}, 
		commentstyle=\color{codegreen},
		keywordstyle=\bfseries\color{magenta},
		numberstyle=\sffamily\tiny\color{black!60},
		stringstyle=\color{green!50!black},
		basicstyle=\ttfamily\footnotesize,
		breakatwhitespace=false,         
		breaklines=true,                 
		captionpos=b,                    
		keepspaces=true,                 
		numbers=left,                    
		numbersep=5pt,                  
		showspaces=false,                
		showstringspaces=false,
		showtabs=false,                  
		tabsize=2
}

\lstdefinestyle{shellstyle}{
		backgroundcolor=\color{black!5}, 
		basicstyle=\ttfamily\footnotesize\color{black}, 
		commentstyle=\color{black}, 
		keywordstyle=\color{black},
		numberstyle=\color{black!5},
		stringstyle=\color{black}, 
		showspaces=false,
		showstringspaces=false, 
		showtabs=false, 
		tabsize=2, 
		numbers=none, 
		breaklines=true
}

\lstdefinelanguage{javascript}{
	keywords={typeof, new, true, false, catch, function, return, null, catch, switch, var, if, in, while, do, else, case, break},
	keywordstyle=\color{blue}\bfseries,
	ndkeywords={class, export, boolean, throw, implements, import, this},
	ndkeywordstyle=\color{darkgray}\bfseries,
	identifierstyle=\color{black},
	sensitive=false,
	comment=[l]{//},
	morecomment=[s]{/*}{*/},
	commentstyle=\color{purple}\ttfamily,
	stringstyle=\color{red}\ttfamily,
	morestring=[b]',
	morestring=[b]"
}

% disponi sezioni
\usepackage{titlesec}

\titleformat{\section}
	{\sffamily\Large\bfseries} 
	{\thesection}{1em}{} 
\titleformat{\subsection}
	{\sffamily\large\bfseries}   
	{\thesubsection}{1em}{} 
\titleformat{\subsubsection}
	{\sffamily\normalsize\bfseries} 
	{\thesubsubsection}{1em}{}

% disponi alberi
\usepackage{forest}

\forestset{
	rectstyle/.style={
		for tree={rectangle,draw,font=\large\sffamily}
	},
	roundstyle/.style={
		for tree={circle,draw,font=\large}
	}
}

% disponi algoritmi
\usepackage{algorithm}
\usepackage{algorithmic}
\makeatletter
\renewcommand{\ALG@name}{Algoritmo}
\makeatother

% disponi numeri di pagina
\usepackage{fancyhdr}
\fancyhf{} 
\fancyfoot[L]{\sffamily{\thepage}}

\makeatletter
\fancyhead[L]{\raisebox{1ex}[0pt][0pt]{\sffamily{\@title \ \@date}}} 
\fancyhead[R]{\raisebox{1ex}[0pt][0pt]{\sffamily{\@author}}}
\makeatother

\begin{document}

% sezione (data)
\section{Lezione del 30-09-24}

% stili pagina
\thispagestyle{empty}
\pagestyle{fancy}

% testo
\subsection{Agenti logici}
Vediamo adesso l'implementazione di \textbf{agenti logici}, cioè agenti che si basano su una certa \textbf{rappresentazione}, detta \textbf{base di conoscenza} (KB, \textit{Knowledge Base}) per immagazzinare \textbf{proposizioni} su ciò che hanno imparato riguardo all'ambiente esterno, e fare \textbf{inferenze}, sulla base di queste proposizioni, rispetto a informazioni non conosciute. Le proposizioni sono legate ad aspetti reali dell'ambiente esterno mediante una determinata \textbf{semantica}.

Possiamo schematizzare il funzionamento della KB, e la sua corrispondenza con l'ambiente esterno, come segue:

\begin{center}
	\begin{tikzpicture}
		\draw[dashed] (-1.5,0) -- (13, 0);

		\draw (2,0.5) rectangle ++(3,1);
		\node[anchor=center] at (3.5, 1) {Proposizioni};

		\draw (9,0.5) rectangle ++(3,1);
		\node[anchor=center] at (10.5, 1) {Proposizione};

		\draw (2,-0.5) rectangle ++(3,-1);
		\node[anchor=center] at (3.5, -1) {Aspetti reali};

		\draw (9,-0.5) rectangle ++(3,-1);
		\node[anchor=center] at (10.5, -1) {Aspetto reale};

		\draw[->] (5, 1) -- node[above, pos=0.5] {\textit{Inferenza}} (9, 1);
		\draw[->] (5, -1) -- node[above, pos=0.5] {\textit{Conseguenza}} (9, -1);

		\draw[->] (3.5,0.5) -- node[right, pos=0.3] {\textit{Semantica}} (3.5,-0.5);
		\draw[->] (10.5,0.5) -- node[right, pos=0.3] {\textit{Semantica}} (10.5,-0.5);

		\node[anchor=center] at (0, 0.5) {\textit{Rappresentazione}};
		\node[anchor=center] at (0, -0.5) {\textit{Ambiente esterno}};
	\end{tikzpicture}
\end{center}

La conoscenza che l'agente ottiene può essere fornita manualmente, cioè in forma di \textbf{assiomi} o conoscenza di \textit{background} (magari in un agente potrebbe essere "precaricata" della conoscenza riguardo allo stato iniziale del problema), estratta dai dati dei sensori, o ricavata dall'esperienza.
Un agente logico dovrebbe essere in grado di fare inferenze in quanto spesso le informazioni che ha sull'ambiente esterno  è \textbf{parziale} o \textbf{incompleta}.

Una KB deve poi rappresentare questa conoscenza attraverso un linguaggio, che dovrebbe essere sufficientemente \textbf{espressivo} da poter rappresentare la realtà dell'ambiente esterno, ma non troppo complesso da impedire di effettuare inferenze in modo efficiente. 

Esistono due approcci all'implementazione di una KB:
\begin{itemize}
	\item \textbf{Dichiarativo:} si concentra su una rappresentazione a \textbf{livello di conoscenza} dei fatti, cioè informazioni riguardo a \textit{cosa} è vero. In sistemi di questo tipo, si usano primitive di scrittura (\textsc{Tell}) e query (\textsc{Ask}) sulla KB, partendo da un insieme di informazioni nullo o comunque limitato, fino ad arrivare ad avere una serie di conoscenze comprensive dell'ambiente esterno. I dettagli delle operazioni che poi l'agente andrà a svolgere, il cosiddetto \textbf{livello di implementazione}, sono mantenuti separati dalla KB.
	\item \textbf{Procedurale:} si concentra su una rappresentazione di \textit{come} effettuare operazioni. Invece di implementare sistemi che possano immagazzinare proposizioni sull'ambiente esterno, si va a codificare l'informazione direttamente nel codice, attraverso procedure, algoritmi, o come avevamo visto nei modelli a riflesso, regole "if-then".
\end{itemize}

\subsubsection{Basi di conoscenza}
Come abbiamo detto, una base di conoscenza è formata da una serie di formule (formule \textit{atomiche}, cioè proposizioni) contenenti informazioni riguardo all'ambiente esterno e codificate in un certo \textbf{linguaggio formale}.
Si possono definire alcune primitive per l'interazione con la KB:
\begin{itemize}
	\item \textsc{\textbf{Tell}}: aggiungi una nuova proposizione alla KB;
	\item \textsc{\textbf{Ask}}: richiedi informazioni dalla KB;
	\item \textsc{\textbf{Retract}}: elimina informazioni dalla KB.
\end{itemize}

La KB si basa sui fatti che già conosce per ricavare inferenze o \textbf{deduzioni logiche} $\alpha$, della forma:
$$ \text{KB} \models \alpha $$

L'agente logico si interfaccia con la KB attraverso le primitive sopra definita, e implementa effettivamente un ciclo \textsc{Tell}-\textsc{Ask}-\textsc{Tell} del tipo:

\begin{algorithm}
\caption{Agente logico}
\begin{algorithmic}
	\STATE \textbf{Input:} le percezioni correnti % input
	\STATE \textbf{Output:} la prossima azione da eseguire % output
	% body
	\STATE \textsc{Tell}(percezioni correnti)
	\STATE prossima azione $\leftarrow$ \textsc{ASK}(KB)
	\STATE \textsc{TELL}(prossima azione)
	\RETURN prossima azione
\end{algorithmic}
\end{algorithm}

Ovvero, l'agente invia le sue percezioni correnti alla KB, e richiede la prossima azione da eseguire.
Invia poi l'azione scelta alla KB (così che possa diventare parte delle informazioni note), e la restituisce.

\subsubsection{Differenza fra KB e DB}
Una KB potrebbe sembrare simile ad un comune database: la differenza è che il database si occupa solo di ricavare fatti specifici, senza possibilità di deduzione di alcun tipo.
La KB è invece progettata per mantenere una rappresentazione strutturata dei fatti, specifici o generali, come riferiti a oggetti reali, e permettere quindi inferenze su quei fatti. 

\subsection{Logica proposizionale}


\end{document}
