
\documentclass[a4paper,11pt]{article}
\usepackage[a4paper, margin=8em]{geometry}

% usa i pacchetti per la scrittura in italiano
\usepackage[french,italian]{babel}
\usepackage[T1]{fontenc}
\usepackage[utf8]{inputenc}
\frenchspacing 

% usa i pacchetti per la formattazione matematica
\usepackage{amsmath, amssymb, amsthm, amsfonts}

% usa altri pacchetti
\usepackage{gensymb}
\usepackage{hyperref}
\usepackage{standalone}

% imposta il titolo
\title{Appunti Ricerca Operativa}
\author{Luca Seggiani}
\date{2024}

% disegni
\usepackage{pgfplots}
\pgfplotsset{width=10cm,compat=1.9}

% imposta lo stile
% usa helvetica
\usepackage[scaled]{helvet}
% usa palatino
\usepackage{palatino}
% usa un font monospazio guardabile
\usepackage{lmodern}

\renewcommand{\rmdefault}{ppl}
\renewcommand{\sfdefault}{phv}
\renewcommand{\ttdefault}{lmtt}

% disponi il titolo
\makeatletter
\renewcommand{\maketitle} {
	\begin{center} 
		\begin{minipage}[t]{.8\textwidth}
			\textsf{\huge\bfseries \@title} 
		\end{minipage}%
		\begin{minipage}[t]{.2\textwidth}
			\raggedleft \vspace{-1.65em}
			\textsf{\small \@author} \vfill
			\textsf{\small \@date}
		\end{minipage}
		\par
	\end{center}

	\thispagestyle{empty}
	\pagestyle{fancy}
}
\makeatother

% disponi teoremi
\usepackage{tcolorbox}
\newtcolorbox[auto counter, number within=section]{theorem}[2][]{%
	colback=blue!10, 
	colframe=blue!40!black, 
	sharp corners=northwest,
	fonttitle=\sffamily\bfseries, 
	title=Teorema~\thetcbcounter: #2, 
	#1
}

% disponi definizioni
\newtcolorbox[auto counter, number within=section]{definition}[2][]{%
	colback=red!10,
	colframe=red!40!black,
	sharp corners=northwest,
	fonttitle=\sffamily\bfseries,
	title=Definizione~\thetcbcounter: #2,
	#1
}

% disponi problemi
\newtcolorbox[auto counter, number within=section]{problem}[2][]{%
	colback=green!10,
	colframe=green!40!black,
	sharp corners=northwest,
	fonttitle=\sffamily\bfseries,
	title=Problema~\thetcbcounter: #2,
	#1
}

% disponi codice
\usepackage{listings}
\usepackage[table]{xcolor}

\lstdefinestyle{codestyle}{
		backgroundcolor=\color{black!5}, 
		commentstyle=\color{codegreen},
		keywordstyle=\bfseries\color{magenta},
		numberstyle=\sffamily\tiny\color{black!60},
		stringstyle=\color{green!50!black},
		basicstyle=\ttfamily\footnotesize,
		breakatwhitespace=false,         
		breaklines=true,                 
		captionpos=b,                    
		keepspaces=true,                 
		numbers=left,                    
		numbersep=5pt,                  
		showspaces=false,                
		showstringspaces=false,
		showtabs=false,                  
		tabsize=2
}

\lstdefinestyle{shellstyle}{
		backgroundcolor=\color{black!5}, 
		basicstyle=\ttfamily\footnotesize\color{black}, 
		commentstyle=\color{black}, 
		keywordstyle=\color{black},
		numberstyle=\color{black!5},
		stringstyle=\color{black}, 
		showspaces=false,
		showstringspaces=false, 
		showtabs=false, 
		tabsize=2, 
		numbers=none, 
		breaklines=true
}

\lstdefinelanguage{javascript}{
	keywords={typeof, new, true, false, catch, function, return, null, catch, switch, var, if, in, while, do, else, case, break},
	keywordstyle=\color{blue}\bfseries,
	ndkeywords={class, export, boolean, throw, implements, import, this},
	ndkeywordstyle=\color{darkgray}\bfseries,
	identifierstyle=\color{black},
	sensitive=false,
	comment=[l]{//},
	morecomment=[s]{/*}{*/},
	commentstyle=\color{purple}\ttfamily,
	stringstyle=\color{red}\ttfamily,
	morestring=[b]',
	morestring=[b]"
}

% disponi sezioni
\usepackage{titlesec}

\titleformat{\section}
	{\sffamily\Large\bfseries} 
	{\thesection}{1em}{} 
\titleformat{\subsection}
	{\sffamily\large\bfseries}   
	{\thesubsection}{1em}{} 
\titleformat{\subsubsection}
	{\sffamily\normalsize\bfseries} 
	{\thesubsubsection}{1em}{}

% disponi alberi
\usepackage{forest}

\forestset{
	rectstyle/.style={
		for tree={rectangle,draw,font=\large\sffamily}
	},
	roundstyle/.style={
		for tree={circle,draw,font=\large}
	}
}

% disponi algoritmi
\usepackage{algorithm}
\usepackage{algorithmic}
\makeatletter
\renewcommand{\ALG@name}{Algoritmo}
\makeatother

% disponi numeri di pagina
\usepackage{fancyhdr}
\fancyhf{} 
\fancyfoot[L]{\sffamily{\thepage}}

\makeatletter
\fancyhead[L]{\raisebox{1ex}[0pt][0pt]{\sffamily{\@title \ \@date}}} 
\fancyhead[R]{\raisebox{1ex}[0pt][0pt]{\sffamily{\@author}}}
\makeatother

\begin{document}

% sezione (data)
\section{Lezione del 30-09-24}

% stili pagina
\thispagestyle{empty}
\pagestyle{fancy}

% testo
\subsection{Calcolo dei vertici}
Troviamo adesso il modo di calcolare i vertici del poliedro di un problema LP in forma primale standard.

Avevamo già dato la definizione di vertice come punto non ottenibile come combinazione convessa degli altri punti del poliedro (definizione 4.6).
Questa definizione è corretta ma poco utile per il calcolo procedurale.
Dimostriamo quindi un teorema utile.

Innanzitutto, assumiamo che in generale, per un problema LP $ \mathcal{P} $ con $n$ variabili decisionali e $m$ vincoli, si ha che $n < m$.
A questo punto, possiamo dire:
\begin{definition}{Soluzione di base}
	Sia dato un problema LP $\mathcal{P}$ con $n < m$.
	Sia $B \subseteq \{ 1, ..., m \}$ un sottoinsieme di indici di riga tale che $\mathrm{card}(B) = n$.
	A questo punto, sia $A_B$ la sottomatrice di $A$ con con righe indicate da $B$, e $b_B$ il sottovettore colonna di $B$ con righe indicate da $B$, con det $\det{A_B} \neq 0$.
	Allora la soluzione di:
	$$
	A_Bx = b_B
	$$
	è detta soluzione di base di $\mathcal{P}$.
\end{definition}
da qui possiamo dimostrare il teorema:
\begin{theorem}{Caratterizzazione dei vertici}
	Un punto $x$ di un poliedro $P$ è un vertice di $P$ se e solo se è una soluzione di base ammissibile, ovvero:
	$$ 
	x \in \mathrm{vert}(P) \Leftrightarrow \text{$x$ è soluzione di base}
	$$
\end{theorem}

Riflettiamo un'attimo su questo risultato: se la definizione di vertice era effettivamente sufficiente a dichiarare \textit{quali} erano i vertici, un teorema di caratterizzazione come quello sopra riportato ci fornisce una procedura per calcolarli tutti a partire da $P$.
In questo, definizione e teorema di caratterizzazione sono effettivamente intercambiabili, ovvero:
$$
\text{definizione} \Leftrightarrow \text{teorema di caratterizzazione}
$$

\end{document}
